\documentclass[serif]{sfuthesis}

%% Fill in below, the values here are fine for a physics masters (just put the name of the thesis and your name and update the date and previous degrees info).

%% No quotes in the title, do NOT end it with a period
\title{How Variance in Initiation Factors Affects the Replication Profile of Budding Yeast DNA}
\thesistype{Thesis}
\author{Mike Chomitz}
\previousdegrees{%
	B.Sc., Trent University, 2012}
\degree{Master of Science}
\discipline{Physics}
\department{Department of Physics}
\faculty{Faculty of Science}
\copyrightyear{2015}
\semester{Fall 2015}
\date{UPDATE THIS}

%% Choose words or phrases that will help people locate your thesis via search tools (library catalogues, Google, etc.)
%% Keywords appear under the abstract and must be entered into the Thesis Registration System.
\keywords{DNA replication kinetics; multiple initiator model; phantom nuclei; Poisson process}


\committee{%
	\chair{Dr.\ Some Body}{POSITION IN DEPARTMENT}
	\member{Dr.\ John Bechhoeffer}{Senior Supervisor\\Professor}
	\member{Dr.\ Malcolm Kennet}{Supervisor\\Associate Professor}
	\member{Dr.\ Some One}{Internal Examiner\\POSITION IN DEPARTMENT}
%% add or remove \members as need
}

%  PACKAGES AND CUSTOMIZATIONS  %%%%%%%%%%%%%%%%%%%%%%%%%%%%%%%%%%%%%%%%%
%
%  You don't need to call the following packages, which are already called
%  in the sfuthesis class file:
%  - etoolbox
%  - setspace
%  - enumitem
%  - pdfpages
%  - tocloft
%  - appendix
%  - cmbright or lmodern
%
%  If you call one of the above packages (or one of their dependencies)
%  with different options, you may get a "Option clash" LaTeX error.
%  If you get this error, you can fix it by removing your copy of
%  \usepackage and just passing the options you need by adding
%
%  \PassOptionsToPackage{<options>}{<package>}
%
%  before \documentclass{sfuthesis}.
%

\usepackage{amsmath,amssymb,amsthm}
\usepackage{fixltx2e}





%  FRONTMATTER  %%%%%%%%%%%%%%%%%%%%%%%%%%%%%%%%%%%%%%%%%%%%%%%%%%%%%%%%%
%
%  Title page, committee page, copyright declaration, abstract,
%  dedication, acknowledgements, table of contents, etc.
%

\begin{document}

\frontmatter
\maketitle %% The details of this command are in sfuthesis.cls. Line 299
\makecommittee %% The details of this command are in sfuthesis.cls. Line 336
\makecopyrightdeclaration %% This page should stay as is, except it needs to be updated to the latest version of the declaration of partial copywrite.
				%% To update, just download the latest version, name it declaration_of_partial_copyright_licence.pdf and make sure it is saved in the
				%% Same folder as this file.

\begin{abstract}
	This is a blank document from which you can start writing your thesis.
\end{abstract}


\begin{dedication} % optional
	For anyone reading this, may you find what you're looking for.
\end{dedication}


\begin{acknowledgements} % optional
	UPDATE THIS
\end{acknowledgements}

%% making automatically managed pages: the table of contents, and lists of tables and figures.
%% To change the table of contents, check the sfuthesis.cls file. The code for the table of contents starts at line 493.
\addtoToC{Table of Contents}\tableofcontents\clearpage
\addtoToC{List of Tables}\listoftables\clearpage
\addtoToC{List of Figures}\listoffigures





%  MAIN MATTER  %%%%%%%%%%%%%%%%%%%%%%%%%%%%%%%%%%%%%%%%%%%%%%%%%%%%%%%%%
%
%  Start writing your thesis --- or start \include ing chapters --- here.
%

\mainmatter%

\chapter{Introduction}

Ubiquitous to all life is the timely and accurate replication of DNA.
In simple cells (called prokaryotes), the process used to replicate the genome (called the ``replication program'') is well understood.
Starting a sequence-defined location (the ``origin''), the double stranded DNA of the prokaryote genome separates into two single-stranded segments.
From the origins, replicates bidirectionally as the single-stranded segments grow and are used as templates for the creation of two copies of the original genome.
The boundaries between replicated and non-replicated DNA, called ``forks,'' continue to propogate throught the genome until it has been entirely separated and replicated.
Prokaryotic organisms have such small genomes that replication can be completed using a single origin.\cite{MolecularCellBiology}

More complex organisms (called eukaryotes), have genomes which are aproximately 1000 times longer and have forks that propogate about 10 times slower than prokaryotes [FIND SOURCE].
Therefore, a single origin cannot fully replicate eukaryotic DNA quickly enough to sustain life.
To overcome this limitation, eukaryotic DNA is replicated using a parallel process, with many origins firing throughout the genome [FIND SOURCE].

Using many origins in the replication program creates several non-obvious issues that must be addressed for the program to work effectively.
One such issue is the existance of separate replicated regions during the replication process.
When the forks of two neighbouring replicated regions meet, the two coalesce into a single, larger replicated region.
Another hurdle the program must overcome is the need to coordinate multiple origin activations events that are driven by stochastic processes.
In other words, the program must be able to control the replication program such that the stochastic events that drive replication do not cause errors that can potentially harm the organism.\cite{eukaryotereview}

%The replication of DNA is a key step in the reproduction and continuation of all known life.  Advances in experimentally techniques, noteably genome sequencing, have helped scientists observe the replication program\cite{Bechhoefer2012374}\cite{DeepSeq}, the process used to replicate the entire genome completely and quickly enough to sustain life.  However, as with all science, increased knowledge from observations leads to new questions. Specifically, this research addresses questions about what mechanisms exists to control the timing of the program.

%The introduction starts by presenting a simplified description of DNA replication. The simple picture is meant to introduce some terms that will be used throughout. After the simple picture, we will look in more detail at parts of the process important to the research presented.  This more focussed description will develop more vocabulary, and provide some background for the research. The discussion will then move from the replication process to experimental methods used in measuring DNA replication and how those experiments can be tied into theory.  Finally, a brief outline of the multiple initiator model for DNA replication will be given, and the precise question this thesis tries to answer will be asked.

%	\section{Simple Picture}\label{SimplePic}

%	Simple organisms are called prokaryotic.  There are two kingdoms of prokaryotic organisms: Bacteria and archaea. Because they are simple, prokaryotes have relatively small genomes, and therefore a simple replication process. Replication starts at a specific location on the genome, called the ``origin of replication'' (or just ``origin'').  From the origin, replication propogates bidirectionally along the genome.  The moving boundaries between replicated and unreplicated regions are called ``forks''.  Prokaryotic organisms have such small genomes that replication can be completed using a single origin.\cite{MolecularCellBiology}

%	More complex organisms are called eukaryotic. Examples of eukaryotic organisms include yeast, flies, frogs, and humans. These organisms have genomes aproximately 1000 times longer, and forks that propagate about 10 times slower than prokaryotes [FIND A SOURCE]. Therefore, a single origin cannot fully replicate eukaryotic DNA fast enough to maintain the reproduction cycle.  To overcome this limitation, eukaryotic DNA is replicated using a parallel process, with many origins firing throughout the genome\cite{}.  With multiple origins firing, there is one more action to describe: ``coalescence''.  When two origins fire with no origin between them, they create two growing regions of replicated DNA separated by a shrinking region of non-replicated DNA.  The two forks surrounding the non-replicated region move toward each other, and when they meet they coalesce, combining the two separate replicated regions into one larger region.\cite{eukaryotereview}

%	\section{Finer Detail}

%		With that introduction, some vocabulary has been developed. The process can be expanded upon to increase the resolution of the picture.  First we will look at the life cycle of a cell, and go into more detail about some of its key stages.  Then we will discuss the specific organization of origins on the genome and the mysteries that we investigated regarding the timing of those origins' firing.  For a full definitions of exactly which protiens take part in the replication process and some exporation of how they interaction, see \cite{PurifiedProteins}.

	\section{The Cell Cycle}
	
	The cell cycle defines the steps taken during cellulare reproduction.
	There are many ways to define the differenst steps, or phases, taken during the cell cycle, depending on the depth and resolution needed.
	For the purpose of communicating this research, the cell cycle will be divided into four steps: The first gap (G1) phase, the synthesis (S) phase, the second gap (G2) phase, and the mitosis (M) phase.
	The first two phases of the cell cycle (G1 and S) are described by key processes for DNA replication.
	The third phase of the cell cycle (G2) primarily acts as a buffer to ensure complete DNA replication.
	The forth phase of the cell cycle (M) encompasses the physical division of the cell into two daughter cells.
	The G2 and M phases are very important for robust cellular reproduction, but do not play a large role in DNA replication, and will therefore remain undiscussed.
	
		\subsection{The G1 Phase}
		
		As with any cycle, the cell cycle begins at the end of the previous cell cycle.
		The G1 phase begins early in the life of a daughter cell, after the mother cell has divided in the previous M phase.
		During this time, the cell grows and, more importantly, an important chemical process called licensing is carried out to prepare for replication during S phase.
		
		Licensing occurs at the origin recognition complex (ORC).
		The ORC is made up of a single group of six protiens that bind to the DNA at an origin.
		Two additional protiens (Cdc6 and Cdt1) assist the ORC in recruiting minichromosome maintenance (MCM) 2-7 hexamer rings onto the DNA.
		After licensing, the resulting set of protiens associated with the origin is call the pre-replication complex (pre-RC).
		Licensing is surpressed during the S and G2 phases by cyclin-dependant kinases.\cite{MolecularCellBiology}
		
		\subsection{The S Phase}
		
		The second phase in the cell cycle is the S phase.
		After lisencing is completed in the G1 phase, the S phase encompases the division of the genome into two identical copies.
		There are three main processes that happen during the S phase: Initiation, elongation, and coalescense.
		
		
		
%	The cell cycle is composed of four phases which broadly describe reproduction for all cell types.  They are the \textbf{G1} (gap-1) \textbf{phase}, the \textbf{S} (synthesis) \textbf{phase}, the \textbf{G2} (gap 2) \textbf{phase}, and the \textbf{M} (mitosis) \textbf{phase}.  The G1 phase consists of cell growth chemical preperation for DNA replication, this phase will be discussed more later.  The S phase is defined as the time during which the DNA is replicated.  The S phase will be duscussed more later.  After the S phase, the G2 phase consists of continued cell growth and chemical preperation for cell division.  Finally, the M phase is the time the cell spends going through division into two daughter cells.  The G2 phase and the M phase do not play a major role in the research presented and are presented here to complete the cell cycle, but will not be discussed further.

%			\subsubsection{The G1 Phase}
			
%			The G1 phase starts immediately after cell division. In addition to cell growth, an important chemical process occurs during this phase which directly affect DNA replication.  The total chemical process is called \textbf{licensing}, is made up of several steps, and results in the construction of the pre-replicative complex (pre-RC).  First, the origin recognition complex (ORC), composed of 6 protiens, binds to the DNA.  With the help of additional protiens, the ORC can recruit minichromosome maintanance (MCM) 2-7 hexamer rings onto the genome.  Two of these rings oriented to face away from each other form a potential initiator.  Each pre-RC makes up one origin, as presented in section \ref{SimplePic}.
			
%			\subsubsection{The S Phase}
			
%			After the G\textsubscript{1} phase completes, the S Phase begins.  The defining event of the S phase is DNA replication.  Replication starts when an origin is activated.  When an origin is activated, the pre-RC is disassembled. With the help of additional protiens, the MCM2-7 ringsundergo a conformational change. They move as helicases along the genome, separating the two strands of DNA.  The helicases become the forks, as described in section \ref{SimplePic}.  After the two strands have been separated, additional protiens act on the separated single-stranded DNA to form two identical dsDNA strands.  When two forks moving toward each other meet, the two helicases disassemble, and the replicated regions coalesce.
			
%		\subsection{Origin Organization}

%		Now that we have seen what an origin is, and the role it plays in DNA replication, lets explore how origins are organized.  One could imagine that an effective approach for origin organization would place all origins equidistant along the genome and fire them simultaneously at the start of the S phase.  However, the real picture is somewhat more complex.  In terms of firing-times, inside the cell, signals are delivered primarily diffusively, so it isn't possible to signal every origin to activate simultaneously.  Therefore, origins can (and do) fire throughout the S phase, untill there are no more origins in unreplicated regions of the DNA.  The spacial patterning of origins is similarly complex.
		
		






%  BACK MATTER  %%%%%%%%%%%%%%%%%%%%%%%%%%%%%%%%%%%%%%%%%%%%%%%%%%%%%%%%%
%
%  References and appendices. Appendices come after the bibliography and
%  should be in the order that they are referred to in the text.
%

\backmatter%
	\addtoToC{Bibliography}
	\bibliographystyle{plain}
	%% Change the name ``references'' to the name of your .bib file (in the same folder as this file).
	\bibliography{references}

%% To add additional appendicies, merely add \chapters
\begin{appendices} % optional
	\chapter{Code}
	\section{the first code}
	the first peice of code is this
	\subsection{the first function}
	the first function in the first code is that
	
\end{appendices}

% You can choose to add an index to your thesis as well. To do so, make sure you \addtoToC{Index}. You'll also need to look up how to include items in the index.
\end{document}
