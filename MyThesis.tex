\documentclass[serif]{sfuthesis}

%% Fill in below, the values here are fine for a physics masters (just put the name of the thesis and your name and update the date and previous degrees info).

%% No quotes in the title, do NOT end it with a period
\title{How Variance in Initiation Factors Affects the Replication Profile of Budding Yeast DNA}
\thesistype{Thesis}
\author{Mike Chomitz}
\previousdegrees{%
	B.Sc., Trent University, 2012}
\degree{Master of Science}
\discipline{Physics}
\department{Department of Physics}
\faculty{Faculty of Science}
\copyrightyear{2015}
\semester{Fall 2015}
\date{UPDATE THIS}

%% Choose words or phrases that will help people locate your thesis via search tools (library catalogues, Google, etc.)
%% Keywords appear under the abstract and must be entered into the Thesis Registration System.
\keywords{DNA replication kinetics; multiple initiator model; phantom nuclei; Poisson process}


\committee{%
	\chair{Dr.\ Some Body}{POSITION IN DEPARTMENT}
	\member{Dr.\ John Bechhoefer}{Senior Supervisor\\Professor}
	\member{Dr.\ Malcolm Kennet}{Supervisor\\Associate Professor}
	\member{Dr.\ Some One}{Internal Examiner\\POSITION IN DEPARTMENT}
%% add or remove \members as need
}

%  PACKAGES AND CUSTOMIZATIONS  %%%%%%%%%%%%%%%%%%%%%%%%%%%%%%%%%%%%%%%%%
%
%  You don't need to call the following packages, which are already called
%  in the sfuthesis class file:
%  - etoolbox
%  - setspace
%  - enumitem
%  - pdfpages
%  - tocloft
%  - appendix
%  - cmbright or lmodern
%
%  If you call one of the above packages (or one of their dependencies)
%  with different options, you may get a "Option clash" LaTeX error.
%  If you get this error, you can fix it by removing your copy of
%  \usepackage and just passing the options you need by adding
%
%  \PassOptionsToPackage{<options>}{<package>}
%
%  before \documentclass{sfuthesis}.
%

\usepackage{amsmath,amssymb,amsthm}
\usepackage{fixltx2e}
\usepackage[rightcaption]{sidecap}



%  FRONTMATTER  %%%%%%%%%%%%%%%%%%%%%%%%%%%%%%%%%%%%%%%%%%%%%%%%%%%%%%%%%
%
%  Title page, committee page, copyright declaration, abstract,
%  dedication, acknowledgements, table of contents, etc.
%

\begin{document}

\frontmatter
\maketitle %% The details of this command are in sfuthesis.cls. Line 299
\makecommittee %% The details of this command are in sfuthesis.cls. Line 336
\makecopyrightdeclaration %% This page should stay as is, except it needs to be updated to the latest version of the declaration of partial copywrite.
				%% To update, just download the latest version, name it declaration_of_partial_copyright_licence.pdf and make sure it is saved in the
				%% Same folder as this file.

\begin{abstract}
	DNA replication starts at many sites (origins) throughout eukaryotic DNA.
	To fully understand the replication program in these higher organisms, one needs to have a strong understanding of the behaviour of these origins.
	In \emph{Saccharomyces cerevisiae} (budding yeast), the spatial organization of the origins is well understood: they are confined to specific sites on the genome called autonomously replicating regions.
	However, when analyzed temporally, the origins in budding yeast are not as rigidly defined.
	The origins are generally categorized as early-firing and late-firing, but the mean fire times of origins exist on a spectrum.
	Past analysis of the budding yeast replication program motivated the creation of the Multiple Initiator Model (MIM).
	
	The MIM is a powerful analytic model that can predict the replicated fraction of a genome given a set of parameters describing the genome.
	Based on an observed correlation between the mean firing time ($t_{1/2}$) and the spread in firing times $(t_\text{width}$) for each origin, the MIM is based on a simple biological hypothesis:
	These parameters are defined for each origin by the number of initiators loaded at that origin ($n$).
	The MIM makes a simplifying assumption that the number of initiators is, on average, large; therefore the size of fluctuations between cells will be small.
	However, recent experimental observations contradict this assumption.

	In three main investigations, we measure the effect of small $n$ on the efficacy of the MIM.
	For these investigations we developed the MIM simulator, a modular program that simulates the replication process in an input genome.
	Our first, preliminary measurements show that for a single origin there is a strong negative change in the accuracy of the MIM as $n$ decreases, which has a growth proportional to $n^{-1}$.
	However, in this naive investigation we use a metric that does not scale to more than on origin.
	Therefore, our second investigation also measured a single origin, with an updated metric for defining the error in the MIM.
	With the updated metric, we again show that for a single origin the MIM does not perform as well for small $n$ as it does for large $n$.
	For our third investigation, we expand the genome from one origin to many to represent Chromosome I of budding yeast.
	We generate two sets of parameters by fitting the MIM to experimental sequencing data, constrained to produce high $n$ and low $n$ results.
	We showed with our simulations of Chromosome I that, unlike the single-origin case, the MIM is equally effective for high and low $n$.
	We then investigate the MIM fit to simulations of small sub-sequences of Chromosome I and show that origins in close proximity cooperate to mitigate the effects of small $n$ on the MIM. 
\end{abstract}


\begin{dedication} % optional
	For anyone reading this, may you find what you're looking for.
\end{dedication}


\begin{acknowledgements} % optional
	UPDATE THIS
\end{acknowledgements}

%% making automatically managed pages: the table of contents, and lists of tables and figures.
%% To change the table of contents, check the sfuthesis.cls file. The code for the table of contents starts at line 493.
\addtoToC{Table of Contents}\tableofcontents\clearpage
\addtoToC{List of Tables}\listoftables\clearpage
\addtoToC{List of Figures}\listoffigures





%  MAIN MATTER  %%%%%%%%%%%%%%%%%%%%%%%%%%%%%%%%%%%%%%%%%%%%%%%%%%%%%%%%%
%
%  Start writing your thesis --- or start \include ing chapters --- here.
%

\mainmatter%

\chapter{Introduction}
\label{ch:Introduction'}

The timely and accurate replication of DNA is a crucial step in the continuation of all life.
In simple cells (called prokaryotes), the process used to replicate the genome (called the ``replication program'') is well understood.
Starting at a sequence-defined location (the ``origin''), the double-stranded DNA (dsDNA) of the prokaryote genome separates into two single-stranded DNA (ssDNA) segments.
Complex biological machinery travel bidirectionally from the origin, separating the dsDNA into growing ssDNA segments which are used as templates for the creation of two copies of the original genome.
The machinery between replicated and non-replicated DNA, called ``forks,'' continue to propagate through the genome until it has been entirely separated and replicated.
Prokaryotic organisms have such small genomes that replication can be completed using a single origin \cite{MolecularCellBiology}.
With this replication program, an \emph{E.~coli} bacterium can replicate its entire genome in about 40 minutes\footnote{Appendix \ref{ap:EColi}}.

More complex organisms (called eukaryotes), have genomes that are approximately 1000 times longer and have forks that propagate about 10 times slower than prokaryotes.
The replication forks from a single origin would take about $10^4$ time as long to fully separate the dsDNA for replication.
Many human cells replicate in about 24 hours, during 8 of those hours, the DNA is replicated \cite{CellMolApproach}.
This is much less than $10^4$ times as long as the 40 minute replication time of \emph{E.~coli}.
Therefore, a single origin cannot be soley responsible for the replication of eukaryotic DNA.
Eukaryotic DNA is replicated using a parallel process, with many origins along the genome that fire stochastically.

Using many origins in the replication program creates several non-obvious issues that must be addressed for the program to work effectively.
One such issue is the existence of separate replicated regions during the replication process.
When the forks of two neighbouring replicated regions meet, the two coalesce into a single, larger replicated region.
Another hurdle the program must overcome is the need to coordinate multiple origin activation events that are driven by stochastic processes.
In other words, the program must be able to control the replication program such that the stochastic events that drive replication do not cause errors that can potentially harm the organism. \cite{eukaryotereview}


	\section{The Cell Cycle}
	\label{sec:CellCycle}
	
	The cell cycle defines the steps taken during cellular reproduction.
	The cell cycle can be divided into four phases (see Figure \ref{fig:CellCycle}) \cite{CellMolApproach,MolecularBiology}:
	The first gap (G1) phase, the synthesis (S) phase, the second gap (G2) phase, and the mitosis (M) phase.
	The first two phases of the cell cycle (G1 and S) contain key processes for DNA replication.
	The third phase of the cell cycle (G2) primarily acts as a buffer to ensure complete DNA replication.
	During the fourth phase of the cell cycle (M) the cell physical divides into two daughter cells.
	
	\begin{SCfigure}[1][thb]
		\includegraphics[width=0.45\textwidth]{images/CellCycle.png}
		\caption[Cell Cycle]{\label{fig:CellCycle} The complete cell cycle is made up of four phases: 
			The Mitosis (M) phase, when a mother cell separates into two daughter cells. 
			The first Gap (G1) phase, when the daughter cell undergoes growth and chemical preparation for DNA replication.
			The Synthesis (S) phase, when the DNA is replicated.
			And the second Gap (G2) phase which acts as a buffer to ensure complete replication before the M phase.}
	\end{SCfigure}
	
		\subsection{The G1 Phase}
		\label{subsec:G1Phase}
		
		As with any cycle, the cell cycle begins at the end of the previous cell cycle.
		The G1 phase begins early in the life of a daughter cell, after the mother cell has divided in the previous M phase.
		During this time, the cell grows and, more importantly, an important chemical process called licensing is carried out to prepare for replication during S phase.
		
		Licensing occurs at the origin recognition complex (ORC).
		The ORC is made up of a single group of six proteins that bind to the DNA at an origin.
		Two additional proteins (Cdc6 and Cdt1) assist the ORC in recruiting minichromosome maintenance (Mcm) 2-7 hexamer rings onto the DNA.
		Loaded hexamers form pairs oriented away from each other, each such pair will later be referred to as potential initiators, or just initiators.
		After licensing, the resulting set of proteins associated with the origin is call the pre-replication complex (pre-RC).
		Licensing is suppressed during the S and G2 phases by cyclin-dependent kinases, this effectively limits the cell cycle to a single replication event. \cite{MolecularCellBiology}
		
		\subsection{The S Phase}
		\label{subsec:SPhase}
		
		The second phase in the cell cycle is the S phase.
		After licensing is completed in the G1 phase, the S phase encompasses the division of the genome into two identical copies.
		There are three main processes that happen during the S phase:
		Initiation, elongation, and coalescence.
		
		An origin initiates (or activates, or fires) during S phase when five other proteins bind to each in a pair of Mcm2-7 rings:
		Cdc45 and the tetrameric GINS complex.
		The total system of proteins is called the CMG complex (\emph{c}dc45, \emph{M}cm2-7, \emph{G}INS complex) and composes a helicase which traverses the genome during the S phase.
		As an origin fires, the pre-RC disassembles and the activated pair of helicases unwind and separate the double helix of the double-stranded DNA (dsDNA) into two complimentary single-stranded DNA (ssDNA) chains. \cite{GINSComplex}
		
		After an origin has been initiated, there is a small region of ssDNA bounded on either side by the CMG complex helicases.
		The points where the dsDNA is separated into two ssDNA chains are called replication forks (or just ``forks'').
		Elongation is the process by which the replication forks, with the help of the biological machinery stored in the CMG complex, propagate bidirectionally from the origin, separating the dsDNA.
		As the forks propagate, DNA polymerases bind with the ssDNA between them.
		DNA polymerases use the ssDNA as a template and backbone for synthesizing dsDNA; essentially it adds the missing half back onto the separated strand.
		DNA polymerase can only propagate in the 3' direction.
		This poses a problem, because the two strands in dsDNA are oriented in opposing directions, so the polymerase can only smoothly traverse one of the ssDNA chains (the ``leading strand'') at each fork.
		On the other strand (the ``lagging strand''), the polymerase ``stutters:''
		It replicates a small region in the direction opposite to that of fork propagation until it hits a replicated region, leapfrogs over and past that region in the direction of fork propagation, and repeats.
		These small fragments are called Okizaki fragments.
		On the lagging strand, the Okizaki fragments are connected by DNA ligases.
		
		Finally, when two forks meet, coalescence occurs:
		The helicases disassemble, and the two regions of dsDNA are connected by DNA ligase. \cite{MolecularCellBiology}
		
		For more information on the intricacies of this process, see \cite{OriginsReview}, and for full definitions of those proteins that take part in the replication process and how they interact, see \cite{PurifiedProteins}.
		
	Figure \ref{fig:EarlyReplicationConcept} shows a distilled version of these processes. It effectively communicates the important parts of the cell cycle and removes many of the fine details that can be ignored.
	
	\begin{figure}[tbh]
		\begin{center}
			\includegraphics[width=\textwidth]{images/EarlyReplicationConcept.png}
		\end{center}
			\caption[Events During Replication]{\label{fig:EarlyReplicationConcept} During replication many processes encompassing many proteins and protein structures are carried out.
				This graphic illustrates a digested version of the G1 and S phases of the cell cycle, with only those parts that are necessary to understand the model present.
				\textbf{A} In the G1 phase, origins are located and licensed when the ORC recruits pairs of MCM2-7 hexamers onto the dsDNA.
					During S phase, pairs of MCM2-7 hexamers are activated by the addition of the Cdc45 protein.
					After activation, the resulting structures become the replicative forks which traverse the DNA unwinding the dsDNA allowing the DNA to be replicated.
				\textbf{B} Due to the existence of multiple origins in eukaryotes, there are several other events that happen during the S phase.
					At the start of S phase, there are several origins licensed along the genome (top of image).
					As time progresses (down), origins fire independently and replicative forks propagate along the genome.
					It is common for some number of origins to be passively replicated, that is, they can be replicated by the replicative fork from a neighbouring origin before firing themselves.
					At the end of the S phase, two identical and complete sets of dsDNA will be present (bottom of image).}
	\end{figure}
	
	
	\section{Origins of Replication}
	\label{sec:Origins}
	
	When prokaryotic DNA is replicated, a single origin is sufficient for a competent replicative program.
	In this case the origin is located at a sequence specific site, and there is no need to worry about the firing-time except to ensure it is early enough for complete replication.
	However, eukaryotic DNA requires many origins of replication for a competent replicative program.
	The number of origins in the genome varies by species, from fewer than 800 in budding yeast \cite{OriDB}, to about 100 000 in humans \cite{OriginsReview}.
	When multiple origins exist on the genome, there are more questions one can ask about their behaviour:
	\begin{itemize}
		\item \emph{What determines the locations of the origins?}
		\item \emph{What controls the timing of when an origin fires?}
	\end{itemize}
	
		\subsection{Origin Locations}
		\label{subsec:OriginLocations}
		
		The question ``\emph{What determines the locations of the origins},'' can be difficult to answer.
		Depending on what organism your discussing, the answer can change significantly.
		In \emph{Saccharomyces civiseai} (budding yeast) origins are tightly bound in sequences between 11 and 17 base pairs (bp) in length and are effectively localized \cite{ScottsPaper}.
		In \emph{Schizosaccharomyces pombe} (fission yeast) the origins are loosely bounded in 100 to 200 bp sequences \cite{OriginsReview}.
		The region of potential licensing grows to about 200 kilo base pairs (kb) in the human genome \cite{HumanGenome}.
		In \emph{Xenopus laevis} (African clawed frog) embryos, the origins are placed completely stochastically, with no sequence affinity at all \cite{FrogEmbryo}.
		
		For the research presented, \emph{S. cerevisiae} (budding yeast) was considered.
		Budding yeast is an ideal model species because, unlike the other examples provided, the origins of \emph{S. cerevisiae} are localized.
		In each cell cycle, origins of budding yeast may only be licensed in very narrow windows (aproximately 11 base pairs (bp)).
		These windows are defined by specific sequences in the genome, called \emph{a}utonomously \emph{r}eplicating \emph{s}equences (ARS) elements, have been identified and cataloged\footnote{an online database can be found at http://cerevisiae.oridb.org/} \cite{OriDB}.
		Thus, by choosing budding yeast as the model species, the potential stochasticity in origin locations has conveniently been removed from consideration.
		
		\subsection{Origin Firing-Times}
		\label{subsec:OriginTimes}
		
		Origin initiation is a chemical process, and therefore is governed in part by the diffusive flow of chemicals within the cell.
		This means the initiation process must be stochastic at some level.
		Previous studies of the firing-times of individual origins in budding yeast have found evidence that the process has some driving force \cite{ScottsPaper,StochasticTermination}.
		Both studies measured the average firing-time and the spread in firing-times for each origin in budding yeast and discovered a correlation between them.
		Essentially, origins that tend to fire early have narrowly defined fire-times, while those that tend to fire late have loosely defined firing times.
		This trend implies the existence of a mechanism that strongly controls the fire-time of origins at the start of S phase, but loses its potency as S phase progresses.
		
		A theory for what physical components of the replicative machinery create this timing mechanism will be discussed in detail later [\textbf{Section number or name?}].
		
		
	\section{Modeling Replication}
	\label{sec:Modeling}
	
	To recap, DNA replication begins at origins which, in budding yeast, are localized spatially but whose firing-times exhibit stochasticity.
	Once an origin has fired, replication forks traverse the DNA bidirectionally, enclosing a growing region of replicated DNA between them.
	When two regions of replicated DNA meet, they coalesce into one larger region.
	
	This process can very easily be mapped to a crystallization process in 1 dimension (Figure \ref{fig:CrystalVsReplication}):
	Crystallization starts when the crystal nucleates at nucleation sites, which map to origins of replication.
	From nucleation sites, the crystal grows bidirectionally and the crystal domain is surrounded by boundaries that can be mapped to the replicative forks.
	Finally, when two crystal regions meet they coalesce into a larger region, which matches the coalescence of neighbouring regions of replicated DNA.
	This mapping of DNA replication to crystal growth is quite convenient.
	
	\begin{figure}[tbh]
		\begin{center}
			\includegraphics[width=\textwidth]{Images/CrystalVsReplication.png}
		\end{center}
			\caption[Comparing Crystallization with Replication]{\label{fig:CrystalVsReplication} A comparison between the 1 dimensional KJMA crystallization model and the described replication model.
				On the left is the 1 dimensional crystallization process, with nucleations shown as single circles and the crystals propagating at circular boundaries.
				On the right is the replication process as described, see Figure \ref{EarlyReplicationConcept} for a full description.
				It is possible to take advantage of the striking similarity between the two processes.}
	\end{figure}
	
	Early in the 20th century the KJMA model was developed as a stochastic model that describes crystallization growth in three dimensions.
	Since its inception, the KJMA model has been used for many studies that range from phase transition kinetics \cite{AlloyPhaseTransitions} to Renyi's car-parking problem in one dimension \cite{CarParking}.
	With the wealth of work done in the KJMA, it shouldn't be surprising that in 2005 S.~Jun, H.~Zhang, and J.~Bechhoefer successfully synthesised a KJMA-like model of DNA replication \cite{KJMA1,KJMA2}.
	In the work of 2005, a formalism was developed which can be used to infer the replication profile of a genome given a set of parameters that describe the speed of replication forks and the origins' locations in time and space.
	
	One of the quantities that can be inferred using the KJMA formalism is the replicated fraction, $f(x,t)$.
	The replicated fraction can be interpreted as the probability that the genome at position $x$ has been replicated by the time $t$ after the start of S phase.
	The replicated fraction is an important quantity that will be discussed more later.
	
	
	\section{Experiments Measuring Replication}
	\label{sec:Experiments}
	
	So far, the mechanisms of DNA replication and a quick introduction to modeling that process have been discussed.
	The question still remains:
	\emph{How can DNA replication be observed experimentally?}
	As it turns out, there are several techniques available to scientists that answer this question, including flow cytometry\cite{DeepSeq}, DNA combing\cite{DNACombing}, microarray\cite{McCuneMicroArray}, and sequencing experiments\cite{DeepSeq}.
	All of these techniques differ to some degree, but there are some ways to easily categorize them by way of their scope.
	The scope of an experiment can be broken into two parameters: Time and space.
	
	In terms of time, there are a two approaches that experiments use to look at time.
	The first approach is to \emph{arrest} the cell cycle.
	This usually entails using a chemical bath to stop the cells from moving from one phase to another (for experiments related to DNA replication, this is generally just prior to entering the S phase).
	The main drawbacks to this approach is it is susceptible to systematic error and it is difficult to arrest many species of eukaryotic organisms.
	After arresting the cell cycle, another chemical bath can be used to force the entire population of cells to enter the S phase synchronously.
	The second approach is to forgo arresting the cell cycle and pull samples from an asynchronous population.
	The main drawback to this method is it either creates time-average data or it relies on multiple techniques to infer the timing information.
	For this research, the first scope was investigated.
	
	Spatially, there exists a spectrum of approaches, spanning between two extremes: Perfect basepair resolution and no spacial information.
	At one end of the spectrum, some experiments calculate one value over the entire genome. 
	For example a common flow cytometry technique called florescence activated cell sorting (FACS), measures only the amount of DNA in the cell.
	Techniques like FACS provide only limited information but are simple and fast.
	At the other end of the spectrum, some techniques are able to spatially resolve to small windows (\cite{DeepSeq} uses sequencing experiments to a resolution of 1 kb).
	Experiments with this level of resolution are quite complex but provide tremendous insight toward understanding DNA replication.
	For this research, the second scope was investigated.
	
	Due to their impact on the research, microarray experiments and sequencing experiments will be discussed in more detail. Both experiments were designed to maximize spacial resolution, and both can use either temporal scope.
	
		\subsection{Microarray Experiments}
		\label{subsec:Microarray}
		
		Microarray experiments are high-throughput experiments that measure entire populations of cells simultaneously.
		They start with a microarray chip\footnote{actually, many chips} and a population of cells.
		The DNA of the population is extracted and hybridized with the chip, which provides a measure of the replicated fraction, $f$.
		Depending on the temporal scope of the experiment, the measured replicated fraction could be time-averaged, $f(x)$, or in the case of an arrested population, it could be the replicated fraction for a specific time, $t_i$ after the start of S phase, $f(x,t=t_i)$.
		
		Because this technique measures entire populations, microarray experiments are unable to provide any information about the variability between individual cells.
		More importantly, microarrays suffer from artifacts and limitations that researchers must account for.
		For example, in \cite{McCuneMicroArray} the measured replicated fraction didn't traverse the full spectrum of possible values; instead of values between $0$ and $1$, a range from $0.1$ to $0.9$ was observed.
		
		\subsection{Sequencing Experiments}
		\label{subsec:Sequencing}
		
		A sequencing experiment is any experiment that can identify the sequence of base pairs contained in the input segment of DNA.
		To measure the replicated fraction using sequencing, one must start with with the fully maped genome of the organism in question.
		The DNA from a population of cells to be measured is harvested and broken into segments about 50 bp in length \cite{StochasticTermination}.
		Each segment can be sequenced and compared to the previously mapped genome.
		The normalized histogram of reads over the genome then provides the replicated fraction:
		Regions that have not replicated in any cell are measured once and regions that have replicated in every cell are measured twice.
		
		Except for the actual process of measuring how many of each segment of DNA is present in the sample, sequencing experiments and microarray experiments are very similar.
		The process of arresting cells, or not, is the same for both, as is the broad strokes of analyzing the output.
		However, sequencing experiments do not required any clever data-processing to remove artifacts like those discussed already \cite{EndOfMicroarray}.
		Recent advances lowering the cost of sequencing have seen a transition from microarray experiments to sequencing experiments for measuring DNA replication.
		
	\section{Thesis map \textbf{UPDATE}}
	\label{sec:Map}

	Here I will give a brief outline of the thesis and how to traverse it. For now this will serve as a plan for the outline, but after writing this will be more of a roadmap for readers than for me.
	
	\textbf{Chapter 2 - Motivation}
	should introduce the mathematics of the model more clearly.
	A discussion of the replicated fraction and how to calculate it.
	It should outline Scott's Sigmoidal Model and the observed correlation between $t_{1/2}$ and $t_{width}$.
	It will then introduce the Multiple Initiator Model, its assumptions, and its results.
	Finally, it will discuss Nick's 2014 experiment measuring loaded MCM and the potential problem it points to.
	
	\textbf{Chapter 3 - Method}
	should talk about the different tools I used to try to understand the problem and its solution.
	Phantom Nuclei Simulation.
	Monte Carlo (compare to analytic, pros and cons).
	Igor global fitting.
	
	\textbf{Chapter 4 - Results}
	talk about results (GET THESE!)
	
	\textbf{Chapter 5 - Conclusions}
	discuss the implications of my results.
	where should research go from here?
		
		
		
		
		
		
		
		
		
		

\chapter{Motivation}
\label{ch:Motivation}

Previous research into DNA replication has investigated the timing of origin initiation in great detail~\cite{ScottsPaper,StochasticTermination,Bechhoefer2012374,deMouraModel1,deMouraModel2}.
In particular, Yang \emph{et al.} fit the replication fraction of an origin, $f(x=x_o,t)$, to a Sigmoidal Model to better understand origin firing-time~\cite{ScottsPaper} [\textbf{Section reference}].
The results (discussed in detail below) were confirmed using a different model by Hawkins \emph{et al.} in 2013~\cite{StochasticTermination}.

The work of Yang \emph{et al.} lead to the development of a second analytical model called the Multiple Initiator Model (MIM)~\cite{ScottsPaper} [\textbf{Section reference}].
The MIM made some reasonable assumptions to simplify the math involved.
This simple model can then be used to analyze experimental data and draw conclusions about the physical mechanisms controlling DNA replication.

However, recent work performed in N. Rhind's lab [\textbf{Source this, somehow}] have shown that one of the assumptions used in the MIM may not be representative of the truth. [\textbf{Section reference}]
The purpose of this research is to discover the impact of this new information on the MIM.

The following sections of this chapter will expand on this story, filling in the details of the math behind the models, and explicitly stating the assumptions and measurements made.


	\section{Replicated Fraction}
	\label{sec:ReplicatedFraction}
	
	It was mentioned briefly in Chapter~\ref{ch:Introduction} that the replicated fraction,$f$, can be calculated from both theoretical models and experiments.
	This makes it a valuable quantity because it acts as a bridge between the two.
	
	There are two equivalent ways to describe the replicated fraction as a function of time and space, $f(x,t)$.
	The first is to describe it as a quantity of a single cell.
	In the single cell case, the replicated fraction at $x$ and $t$ is the probability that the genome at position, $x$, in the genome has replicated a time, $t$, after the start of S phase.
	The second description uses a population of cells.
	From a population of cells, the replicated fraction at $x$ and $t$ is the fraction of the cells in the population that have replicated at position, $x$, in the genome a time, $t$, after the start of S phase.
	It is not difficult to see the equivalency between these two descriptions, but it is important to make it clear that both definitions are true.
	
	
		\subsection{Qualities of the Replicated Fraction}
		\label{subsec:QualitiesReplicatedFraction}
		
		Before diving into the mathematical formulae that describe the replicated fraction of the KJMA-like model of DNA replication, it is valuable to build some intuition.
		Experimentally, the replicated fraction is measured spatially in windows about 1 kb wide, and temporally in steps of 5 minutes~\cite{StochasticTermination}.
		Budding yeast's genome is over $12000$kb long so, visually, the data is well resolved spatially.
		However, Budding yeast completely replicates it's DNA in less than 90 minutes~\cite{DeepSeq}, and in fact most experiments stop performing thorough measurements after about 50 minutes~\cite{DeepSeq,StochasticTermination,McCuneMicroArray}.
		Therefore, the data is not visually well resolved temporally.
		The good news is, this amount of temporal resolution is enough to build an intuitive understanding of the replicated fraction, and to analyze mathematically.
		
		Figure~\ref{fig:ReplicatedFractionExample} shows an example set of replicated fraction data.
		The data comes from measurements done on chromosome IV of budding yeast by M{\"u}ller \emph{et al.}~\cite{DeepSeq}.
		The perceptive reader may notice a few things:
		There are gaps in the spatial data.
		The replicated fraction ranges lower than zero and higher than 1.
		Some regions of the genome replicate faster than others.
		
		The gaps exist because of a limitation of the sequencing experiment used to gather this data.
		Sequencing experiments match small chunks of DNA to the fully mapped genome (Section~\ref{subsec:Sequencing})
		There are segments of the budding yeast genome that contain identical sequences [\textbf{FIND SOURCE}].
		When a chunk of DNA that contains only a sequence that is repeated and nothing that can uniquely identify it, that sequence is omitted from the results because there is no way to know where it came from.
		
		The replicated fraction has a larger range than is theoretically possible.
		This is not because some regions are not replicated or replicated doubly, but because of inaccuracies in the experimental procedure.
		
		The most important observation is the fact that some regions of the genome start replicating much earlier than others.
		This can be seen in the peaks in Figure~\ref{fig:ReplicatedFractionExample}, for example at $x \approx 900$.
		
		\begin{figure}[tbh]
			\begin{center}
				\includegraphics[width=\textwidth]{Images/CHR4Exp.png}
			\end{center}
				\caption[Budding yeast chromosome IV replicated fraction]{\label{fig:ReplicatedFractionExample} Replicated fraction of chromosome IV of budding yeast measured by M{\"u}ller \emph{et al.}~\cite{DeepSeq}.
				}
		\end{figure}


	\section{The Sigmoidal Model}
	\label{sec:SigmoidalModel}
	
	

\chapter{Methods}
\label{ch:Methods}

In this chapter, we outline the tools we used in our investigation of the impact of variability in the initiation factor on the MIM.
The primary tool that we developed for our investigation is the MIM simulator, a Monte Carlo program that simulates DNA replication.
A great deal of effort went into the details of the simulation program to ensure it efficiently produces meaningful results:
We ensured the randomly generated numbers were distributed properly.
We adopted the phantom-nuclei algorithm, an efficient way to simulate the replicated fraction~\cite{KJMA1}.
We used multiple programming languages to increase performance.
We simulated measurements consistent with current experimental results~\cite{StochasticTermination}.

Except when noted, all computations were performed in IGOR Pro Version 6.3.6.4.


	\section{The MIM Simulator}
	\label{sec:MIMSimulator}
	
	The MIM simulator takes as inputs a set of parameters nearly identical to those defined by the MIM.
	There are four global inputs: the elapsed time since the start of S phase $t_\text{sim}$, the speed of replicative forks $v$, the median firing time $t_{1/2}$, and $r$, which defines the width of the cumulative firing time distribution.
	There are also two local parameters per origin: the position $x_i$ and the average number of initiators $n_i$.
	This set of parameters is not identical to those outlined in Sec.~\ref{sec:MIM} because, as we will describe in Sec.~\ref{sec:Noise}, noise was not treated the same way.
	The simulator uses these parameters to generate the replicated fraction, $f(x,t=t_\text{sim})$, over the entire genome.
	The simulation does this over several sets of parameters for which only $t_\text{sim}$ changes by steps of 5 minutes, thereby efficiently creating data comparable to those from sequencing experiments.
	
	The MIM simulator has three modules (see Fig.~\ref{fig:ProgramStructure}):
	The preparation module sets the randomly distributed parameters.
	The phantom-nuclei module uses those parameters to calculate $f(x,t=t_\text{sim})$.
	The housekeeping module tracks progress, calls the preparation and phantom-nuclei modules, and analyzes the results.
	Note that while the preparation and the phantom-nuclei modules both simulate only a single cell at a time, the housekeeping module loops over many cells to find the average behaviour of a population.
	These three modules will be discussed in more detail below.
		
	\begin{figure}[tbh!]
		\begin{center}
			\includegraphics[width=.8\textwidth]{Images/ProgramStructure.pdf}
		\end{center}
			\caption[MIM Simulator Program Structure]{\label{fig:ProgramStructure} 
				Flow chart illustrating the MIM simulator structure.
				The program contains three modules: 
				The housekeeping module loops over every cell in the population being simulated, calls the preparation and phantom nuclei modules, adds noise to the results and performs analysis.
				The preparation module generates two sets of random data, the number of initiators at each origin and the firing times of each initiator.
				The phantom-nuclei module pre-processes the data passed to it, and calculates the replicated fraction for each time step using the phantom nuclei algorithm, which is broken into three steps.
			}
	\end{figure}
	
	
		\subsection{The Preparation Module}
		\label{subsec:PrepModule}
		
		The preparation module is a Monte Carlo program, one whose output depends on random numbers~\cite{CompPhys}.
		For the preparation module of the MIM simulator,  we need two sets of random numbers:
		First, the program requires a set of absolute numbers of initiators, $\{N_i\}$, for all origins $i$.
		Second, the program requires a set of firing times, $\{t_{i,j}\}$, for each initiator $j$ loaded at each origin $i$.
		For both sets, we took care to ensure that the generated values were properly distributed to match MIM theory (Sec.~\ref{sec:MIM}).
		The preparation module is analogous to the licensing process undertaken in the G1 phase of the cell cycle (Sec.~\ref{subsec:G1Phase}).
			
		The first task of the preparation module is to randomly generate $\{N_i\}$, the set of absolute numbers of initiators at all origins $i$ for the cell being simulated.
		Thus, the first choice we made in creating the simulator was how the values for $N_i$ should be distributed, given their average $n_i$.
		In Sec.~\ref{subsec:MIMBasics}, we mentioned that a simple hypothesis is that initiators are loaded onto an origin as a Poisson process; if this is the case, the number of initiators should be Poisson distributed.
		This simple model, which assumes initiators are loaded at a constant rate and loading events are not correlated, is easily implemented using built-in IGOR functions.
		However, we are unaware of any experiments that have measured the distribution of the number of initiators over different cell cycles.
		
		In discussion with collaborators, another hypothetical distribution was considered.
		Several studies have shown that Histone activity\footnote{
		Histones are large, octameric proteins that play a role in organizing DNA into its three-dimensional structure in the cell~\cite{MolecularCellBiology}.}
		is correlated with origin locations~\cite{Histone,Histone2,Histone3}; origins tend to be in open, easily accessible regions of the DNA
		We hypothesis that the accessibility also affects the rate of initiator loading, and that the first initiator at an origin may load much faster than additional initiators.
		One way to implement qualitatively this idea would be to enforce a minimum probability for loading at least one MCM2-7 pair.
		(In an extreme case, this probability would be one.)
		
		We chose to use the Poisson distribution for setting the number of initiators at an origin for two reasons:
		First, without experimental evidence to motivate the selection of a complex model, the simple model is preferred.
		Second, since we are testing the efficacy of the MIM, which assumes constant $n$, the Poisson distribution represents a worst-case scenario:
		It includes the possibility of loading zero initiators, and having zero initiators leads to the largest perturbation from the assumption made in the MIM.
		Therefore, the preparation module selects the number of initiators at origin $i$ from a Poisson distribution defined by the average $n_i$.
		
		The second task of the preparation module is to assign a firing time to each initiator on the genome.
		This is different from assigning a firing time to each origin: If there are $k$ origins, then the number of initiators is given by $\sum\nolimits_{i=0}^k N_i = K$.
		Therefore, $K$ randomly generated firing times are required.
		The MIM dictates the desired firing time distribution of an initiator, which we derive from the cumulative firing time probability shown in Eq.~\ref{eq:CPDInitiator} (and again in Eq.~\ref{eq:CPDInitiator2}).
		\begin{equation} \label{eq:CPDInitiator2}
			\Phi_0(t) = \frac{1}{1+\left(\frac{t^*_{1/2}}{t}\right)^{r^*}}\text{ ,}
		\end{equation}
		where $t^*_{1/2}$ and $r^*$ are global parameters defining, for a single initiator, the median firing time and the spread in firing times respectively. 
		Recognizing that $\Phi_0$ goes from zero (when $t=0$), to one (when $t \rightarrow \infty$), we can use inverse transform sampling~\cite{NumRec} to randomly generate firing times that reproduce the desired cumulative firing time probability.
		If we generate $u$, a uniformly distributed number between zero and one we can transform that to be distributed as desired with
		\begin{equation}
			F(u) = \frac{t^*_{1/2}}{\left(\frac{1}{u}-1\right)^\frac{1}{r^*}} \text{ ,}
		\end{equation}
		where $F(u)$ is the firing time.
		This will produce random numbers that exhibit a cumulative probability distribution given by Eq.~\ref{eq:CPDInitiator2}.
		A histogram of $10^5$ samples from the transformation coincided satisfactorily with the cumulative fire-time distribution, implying that the method is sound.
		After all $K$ firing times are generated, the time of the first-to-fire initiator at each origin is kept because each origin can only fire once; thus, the firing time of the origin $i$ is given by the firing time of the earliest initiator $\text{min}\{t_j\}_i$.
		
		
		\subsection{The Phantom-Nuclei module}
		\label{subsec:PhanNuc}
		
		Based on work done by S.~Jun~\emph{et~al.}, the phantom-nuclei algorithm we used in the simulation is a powerful tool for calculating replicative data from a set of parameters describing the origins of replication in the KJMA formalism~\cite{KJMA1}.
		Figure~\ref{fig:phantom} illustrates the key features of the phantom-nuclei method.
		There are three major steps in our phantom nuclei module: pre-processing the parameters, simulating replication, and compiling the replicated fraction.
		In taking these three steps, the phantom nuclei module quickly calculates the regions on the genome of a single cell which have been replicated.
		These steps have been separated for the sake of clarity; however, there is some overlap between them in our implementation to increase performance.
		
		\begin{figure}[tbh]
			\begin{center}
				\includegraphics[width=\textwidth]{Images/PhantomNuclei.png}
			\end{center}
				\caption[Schematic of the Phantom Nuclei Algorithm]{\label{fig:phantom} Schematic of the Phantom Nuclei algorithm.
				Only the active origins (Black circles) are considered during simulations.
				Open circles correspond to passively replicated origins (``phantom'' nuclei).
				The algorithm outputs the replicated fraction, which is one in replicated regions (ellipses at top) or zero in unreplicated regions (black line at top).
				}
		\end{figure}
			
		The strength of the phantom nuclei algorithm is that it pre-processes the origin data it receives.
		To reduce the amount of work needed to fully simulate the replication process, the program removes origins that are passively replicated (``phantom'' nuclei) from the simulation.
		As we mentioned above, we designed the simulator to loop through many values of $t_\text{sim}$.
		The algorithm starts by calculating the state of replication at the highest value for $t_\text{sim}$, $t_\text{sim}^\text{(max)}$.
		We start at $t_\text{sim}^\text{(max)}$ because that is when every meaningful event will have occurred: origins have fired or not, and every passively replicated origin can be identified.
		
		When pre-processing, the program calculates the positions $\{x_i^\text{(L)}\}$ of the left forks and $\{x_i^\text{(R)}\}$ of the right forks originating from all origins $i$.
		Calculating these positions is done via simple kinematics:
		\begin{equation} \label{eq:findforks}
			x_i^{\binom{\text{R}}{\text{L}}} = x_i \pm v \times \left( t_\text{sim} - t_i\right) \text{ ,}
		\end{equation}
		where the right fork is given by the sum and the left fork by the difference, and where the bracketed term calculates the time since the origin fired.
		As a part of pre-processing, any origins for which $t_i > t_\text{sim}^\text{(max)}$ are immediately removed from the simulation, as they will not contribute to the replicated fraction.
		Once the algorithm calculates the set of fork locations, the forks from each pair of neighbouring origins are analyzed to determine which origins are passively replicated.
		Any phantom nuclei are removed from the simulation (open circles in Fig.~\ref{fig:phantom}).
		Pre-processing is finished when only active origins (black circles in Fig.~\ref{fig:phantom}) are left in the simulation.
		Pre-processing is computationally expensive, but for complex genomes will dramatically decrease the calculations needed for the second step, and the number of calculations needed for this process on simple genomes is small.
		
		The second step of the phantom nuclei algorithm is the largest and is used at every time step.
		During the simulation step, the algorithm performs three major calculations:
		First, it selects which origins will fire by comparing their firing times to the current value of $t_\text{sim}$; only origins with $t_i < t_\text{sim}$ will fire.
		Second, using Eq.~\ref{eq:findforks}, the algorithm calculates two sets of fork positions ($\{x_i^\text{(L)}\}$ and $\{x_i^\text{(R)}\}$) from the origins selected in the first step.
		These two sets of fork data are used to define replicated regions on the genome.
		Third, it analyzes the replicated regions defined by the two sets of fork data, and identifies where replicated regions overlap (i.e., coalescence has occurred).
		Any overlapping regions are combined.
		This step is analogous to the S phase of the cell cycle, including initiation (selecting cells that fire before $t_\text{sim}$), elongation (calculating fork positions), and coalescence (combining overlapping regions), as described in Sec.~\ref{subsec:SPhase}.
		
		Immediately after any overlapping replicated regions are coalesced, the algorithm compiles the replicated fraction.
		Therefore, the replicated fraction is compiled at every time step in the simulation.
		To compile the replicated fraction, the algorithm simply loops through the replicated regions defined by $\{x_i^\text{(L)}\}$ and $\{x_i^\text{(R)}\}$ and sets the replicated fraction for the cell to one inside those regions and zero outside.
		Although this process may sound straightforward, we were unable to do it without nested loops that significantly slowed the simulation process.
		For this reason, this step was written both in IGOR and in C++.
		When we simulated large data sets early in our work (Sec.~\ref{subsec:earlywork}), we called the C++ function as an external program.
		Using this external function gave an 8-fold increase in performance.
		
		
		\subsection{The Housekeeping Module}
		\label{subsec:Housing}
		
		The simulation described above calculates the replicated fraction on the entire genome of a single cell.
		However, we are investigating sequencing data that are acquired by averaging over a large population.
		Therefore, the housekeeping module is designed to loop over a population calling the preparation and phantom nuclei modules for each cell.
		The resulting data are then averaged.
		
		In addition, the housekeeping module can analyze and alter the simulated replicated fraction $f_\text{sim}$.
		In our research, we fit the MIM parameters to $f_\text{sim}$; we added Gaussian noise to reflect current experimental measurements; and we calculated the difference between $f_\text{sim}$ and experimentally measured $f$.
		We discuss these procedures and their results in Ch.~\ref{ch:Results}.
		
		\begin{figure}[tbh]
			\begin{center}
				\includegraphics[width=\textwidth]{Images/SimulatedDataExamples.pdf}
			\end{center}
				\caption[Simulated Replicated Fraction for Chromosome IV]{\label{fig:SimulatedExample}
					Example output of Chromosome IV from the MIM simulator.
					x-axis is the position in the genome.
					y-axis is the replicated fraction.\\
					\textbf{A.} Data averaged over 100 cells, no artificial noise.\\
					\textbf{B.} Data averaged over 100 cells, Gaussian noise added according to the procedure outlined in Sec.~\ref{subsec:AddingNoise}.
					Parameters for the simulation were taken from~\cite{ScottsPaper} supplementary data.
				}
		\end{figure}	
		
		
		\subsection{Qualities of the MIM Simulator}
		\label{subsec:QualitiesofMIMSimulator}
		
		The MIM simulator is a powerful tool for generating the replicated fraction of a population of cells with known $\{n_i\}$.
		The Monte Carlo process used in the MIM Simulator calculates the replicated fraction as the average of a population of cells.
		Therefore, the larger the population, the better the averaging and the more confident we are in the data.
		It may seem simple to use the MIM simulator instead of the analytical MIM shown in Sec.~\ref{sec:MIM}.
		However, simulating the replicated fraction to the accuracy needed for a fit takes many thousands of single-cell measurements to average over, and this is computationally expensive.
		By contrast, a single calculation with the analytical MIM will produce the desired fit function.
		Thus, the MIM simulator is not a good replacement for the analytical MIM; rather, it is a too we use to measure the efficacy of the analytical MIM in the small-$n$ regime.
		
		One of the strengths of our program is its modular structure: It is simple to change the probability distribution of $\{N_i\}$ (currently Poisson distributed) or $\{t_j\}_i$ (currently distributed as described above).
		Additionally, doing new analysis is simply a matter of creating a new function that the housekeeping module can call.
		
		Figure~\ref{fig:SimulatedExample} shows two examples of the replicated fraction of Chromosome IV\footnote{
		Figure~\ref{fig:ReplicatedFractionExample} shows the replicated fraction of the same chromosome measured with DNA sequencing~\cite{StochasticTermination}.}
		generated by the MIM Simulator.
		Both simulations were over a population of 100 cells.
		Figure~\ref{fig:SimulatedExample}A shows data output from the program as described so far.
		Below, we describe how and why we generated the noisy data presented in Fig.~\ref{fig:SimulatedExample}B
		
		
	\section{Analyzing Noise in the Data}
	\label{sec:Noise}
	
	For our initial investigations, we chose to do simulations of many simple artificial cells.
	However, as we discuss in Sec.~\ref{subsec:BiasedFits}, this was not a good choice because this method does not scale well to complex cells.
	Because of this problem, we created simulations that were limited in population size and accuracy to reflect current experimental standards.
	We expected that limiting the population size would increase the noise enough to make a good comparison with experiment.
	However, as we show below, generating noisy data was not as simple as limiting our population size.
	
	
		\subsection{Estimating Experimental Noise}
		\label{subsec:SequencingNoise}
		
		Here, we analyze data from a sequencing experiment investigating the replicated fraction of budding yeast performed by Hawkins~\emph{et~al.} in 2013~\cite{StochasticTermination}.
		In their experiment, Hawkins~\emph{et~al.} used DNA sequencing to calculate the replicated fraction of two strains of budding yeast: wild-type budding yeast and a mutant with three origins of replication removed.
		Figure~\ref{fig:ReplicatedFractionExample} shows their results for Chromosome IV of the wild-type genome.
		Notice that the noise in the experiment leads to replication fraction estimates that lie outside the possible range between zero and one.
		Our goal was to create simulated replicated fraction data that closely resembles data from sequencing experiments.
		To do this, we need to include noise in our data commensurate with that seen experimentally and must therefore estimate experimental noise.
		
		\begin{figure}[tbh]
			\begin{center}
				\includegraphics[width=\textwidth]{Images/WTvsMutDifference.pdf}
			\end{center}
				\caption[Estimating Experimental Noise: Mean Point-By-Point Difference]{\label{fig:MeanDifference} Mean point-by-point difference between wild-type and mutant replicated fractions for each chromosome at each time step.
					Each set of axis is for a different time after the start of S phase (labeled).
					y-axis shows the mean difference.
					x-axis is the chromosome label.
					Note that no data are shown for Chromosomes VI, VII, and X, as they were not analyzed due to their mutations.
					Data derived from~\cite{StochasticTermination} supplementary data.
				}
		\end{figure}
		
		Following the process used by Yang~\emph{et al.}~(\cite{ScottsPaper} supplementary material), we analyzed the experimental data to estimate the uncertainty in the measured replicated fraction.
		Ideally, we would estimate the noise distribution for each data point by analyzing data from an experiment that has been repeated many times.
		Unfortunately, Hawkins~\emph{et~al.} did not publish any repetitions of their data set.
		Therefore, we worked with two measurements we assume to be in close agreement: the wild-type budding yeast and the mutant budding yeast measurements reported in~\cite{StochasticTemination}.
		With the mutation only removing three origins, we assumed that the replication profiles between the wild-type and mutant measurements would be the same except on the chromosomes with missing origins (Chromosomes VI, VII, and X).
		Thus, we compared the remaining 13 of the total 16 budding yeast chromosomes.
		To estimate the distribution of fluctuations, we considered how the differences between the experiments, calculated point-by-point, were distributed.
		Figure~\ref{fig:MeanDifference} shows the mean difference for each chromosome ($x$-axis) at each time step (separate axis, labeled).
		Since the differences vary in time, they are analyzed at each time step separately.
		Within each time point the fluctuations are much more stable, except for a downward trend in Chromosome III.
		Thus, in addition to the three chromosomes that were mutated, Chromosome III was removed from our analysis.
		
		\begin{figure}[tbh]
			\begin{center}
				\includegraphics[width=\textwidth]{Images/WTvsMutHistograms.pdf}
			\end{center}
				\caption[Estimating Experimental Noise: Point-By-Point Difference Distributions]{\label{fig:HistDifference} Histograms of the point-by-point difference between wild-type and mutant data and Gaussian fits.
					Each set of axis is for a different time after the start of S phase (labeled).
					y-axis shows the normalized distribution.
					x-axis shows difference.
					Grey circles are calculated from experiment~\cite{StochasticTermination}.
					Black lines show the best Gaussian fit
					Note that no data are shown for Chromosomes III, VI, VII, and X.
					Data derived from supplementary data from~\cite{StochasticTermination}.
				}
		\end{figure}
		
		After removing the data from the four chromosomes mentioned, we compiled histograms for the six time steps measured.
		These histograms (shown in Fig.~\ref{fig:HistDifference}) estimate the probability distribution between the two noisy measurements.
		To properly duplicate the noise of a single experiment, we need the distribution of a single noisy measurement.
		From elementary properties of the variance, two independent random variables $A$ and $B$ have $\text{Var}[A-B] = \text{Var}[A] + \text{Var}[B]$.
		We assume that the two measurements are equally noisy, which implies that the standard deviations of the differences are $\sqrt{2}$ times larger than the standard deviation of a single measurement.
		Our estimates of the noise are shown as open circles in Fig.~\ref{fig:Noise}.
		
		\begin{figure}[tbh]
			\begin{center}
				\includegraphics[width=.8\textwidth]{Images/SigmaEstimate.pdf}
			\end{center}
				\caption[Scatter Plot of Estimated Simulation and Experimental Noise]{\label{fig:Noise} 
					Scatter plot of estimated $\sigma$ vs time since the start of S phase for experimental data and simulation data.
					Open circles show experimental estimates.
					Black circles show simulation estimates.
					Crosses show calculated values for $\sigma_\text{add}$ (Eq.~\ref{eq:AddingNoise})
					Squares show estimates from sequencing simulations.
					Dots show estimates from simulations with added Gaussian noise.
				}
		\end{figure}
		
		There are three features of the histograms in Fig.~\ref{fig:HistDifference} that should be discussed.
		First, unlike the microarray data that Yang~\emph{et al.} analyzed, the histograms extracted from sequencing data are Gaussian distributed.
		This implies that the data sequencing experiments are better suited to analysis with the MIM, since the MIM approach assumes Gaussian-distributed noise~\cite{ScottsPaper}.
		Second, the standard deviation evolves as time progresses.
		This is expected: Early in the replication program and late in the replication program, many of the cells will be mostly unreplicated and mostly replicated  respectively.
		Therefore, we expect that the noise will be diminished at early time and late time.
		Third, the mean of the Gaussian fits evolve dramatically as time progresses.
		We believe this is due to a global systematic error in the data, potentially the reported time since the start of S phase, or a possible global effect of the mutation that removes origins from Chromosomes VI, VII, and X.
		
		\begin{figure}[tbh]
			\begin{center}
				\includegraphics[width=.8\textwidth]{Images/Correlation.pdf}
			\end{center}
				\caption[Autocorrelation of Experiment and Simulations]{\label{fig:correlation}
					Autocorrelation of experimental data and data from three simulations of the full budding yeast genome.
					\textbf{A.} Autocorrelation of simulated data of a population of 100 cells.
					\textbf{B.} Same as \textbf{A}, with artificial noise added as described in Sec.~\ref{subsec:AddingNoise}.
					\textbf{C.} Autocorrelation of experimental data  over the full genome (calculated from~\cite{StochasticTermination} supplementary data).
					\textbf{D.} Autocorrelation of simulated data of a population of 1000 cells, taking only one tenth of the genome per cell (100-fold coverage).
				}
		\end{figure}
		
		In addition to measuring the distribution of the point-by-point differences in experimental data, we measured the correlation length.
		Figure~\ref{fig:correlation}C shows the autocorrelation of the differences after the mean difference had been subtracted.
		We observe two features in the autocorrelation function.
		There is a delta function at $\Delta x=0$, implying that the noise between neighbouring data points is large.
		However, there is also a non-negligible tail, implying long-range order in the experimental data.
		
		
		\subsection{Estimating Simulation Noise}
		\label{subsec:SimulationNoise}
		
		Now that we have estimated noise level in current sequencing experiments, we would like to use those data to ensure our simulated replicated fraction has noise commensurate with experimental data.
		Two steps were taken to make this happen: the first based on experimental procedures, the second by artificially adding Gaussian noise.
		
		The first step taken to make our simulated data similar to experimental data was to limit the size of the population of simulated cells.
		As we mentioned above, the Monte Carlo program operates by taking the average of many cells, which is very similar to sequencing experimental techniques.
		In their experiment, Hawkins~\emph{et al.} extracted 10--25 million 50 bp sequences~\cite{StochasticTermination}.
		Over the genome of 12 Mb, that is equivalent to 50-to 100-fold coverage per base.
		Therefore, we limited our simulations to a population of 100 cells to get equivalent coverage.
		
		\begin{figure}[tbh]
			\begin{center}
				\includegraphics[width=\textwidth]{Images/SimNoise.pdf}
			\end{center}
				\caption[Estimating Simulation Noise: Point-By-Point Difference Distributions]{\label{fig:SimNoise} Histograms of the point-by-point difference between two sets of simulated data.
					Each set of axis is for a different time after the start of S phase (labeled).
					y-axis shows the normalized distribution.
					x-axis shows difference.
					Grey circles are calculated from experiment.
					Black line at $t=15$ shows the best Laplace distribution fit.
					Black line at $t=30$ shows the best Gaussian fit.
				}
		\end{figure}
		
		We generated two simulated replicated fraction functions over the entire genome from 100-cell populations (parameters were set using results from the MIM~\cite{ScottsPaper}).
		These two functions were used to estimate the noise in the simulation, $\sigma_\text{sim}$, using the same process as outlined in Sec.~\ref{subsec:SequencingNoise}.
		Figure~\ref{fig:SimNoise} shows the distributions of the difference at each time step.
		Interestingly, there is an evolution in the noise from near-Laplace distributed at early time, to near Gaussian, and back to near-Laplace distributed.
		Figure~\ref{fig:Noise} shows our estimation of the noise in simulation, measured using built-in IGOR Pro tools that report the standard deviation of a set of data.
		
		Is  noise in simulations distributed differently than the noise in experiment?
		To address this concern, we qualitatively investigated a possible source of the noise.
		We know that the greatest uncertainty in replicated fraction will coincide with the presence of forks of replication:
		While regions that replicate early and regions that replicate late will simulate a replicated fraction of primarily ones and primarily zero respectively, regions that are in the process of replicating will return both values.
		Therefore, we measured  the average number of replicated regions across the genome over simulation time (Fig.~\ref{fig:NumberIslands}).
		Except when a fork has hit the end of the chromosome, the number of forks is twice the number of replicated regions.
		We observed a peak in the number of replicated regions, and hence forks, at 30 minutes after the start of S phase.
		This time coincides with the time at which the distribution of the simulated noise is most Gaussian (Fig.~\ref{fig:SimNoise}).
		Thus, noise in simulation is Laplace distributed when the number of forks is small, and adding more forks makes the distribution more Gaussian.
		
		\begin{figure}[tbh]
			\begin{center}
				\includegraphics[width=.8\textwidth]{Images/NumIslands.pdf}
			\end{center}
			\caption[Number of Replicated Regions in Simulation]{\label{fig:NumberIslands}
				Histogram of the number of replicated regions per cell in the simulation.
				x-axis shows $t_\text{sim}$ in minutes.
			}
		\end{figure}
		
		
		\subsection{Adding Gaussian Noise to the MIM Simulator}
		\label{subsec:AddingNoise}
		
		We changed two features of the noise in simulation to better produce noise commensurate with experiments.
		First, as shown in Fig.~\ref{fig:Noise}, the statistical noise that arises from the random sampling of the Monte Carlo process is not large enough to match the noise we estimated for the experiment.
		Second, as shown in Fig.~\ref{fig:correlation}A and C, the experimental data has a short-range disorder that is lacking in the simulation over a population of 100 cells.
		Therefore, we added extra noise to the simulated data to match the levels we found in Sec.~\ref{subsec:SequencingNoise}.
		To add the noise, we used our estimate of the uncertainty from the simulation, $\sigma_\text{sim}$, then calculated the amount of Gaussian noise we had to add, $\sigma_\text{add}$, such that the resulting uncertainty matched the desired values:
		\begin{equation} \label{eq:AddingNoise}
			\sigma_\text{add} = \sqrt{{\sigma_t}^2 - {\sigma_\text{sim}}^2} \text{ ,}
		\end{equation}
		where $\sigma_t$ is the experimental noise calculated for the simulated time $t$ from experimental data (Sec.~\ref{subsec:SequencingNoise}).
		The crosses in Fig.~\ref{fig:Noise} show the resulting values of $\sigma_\text{add}$ that were used for our simulations.
		
		\begin{figure}[tbh]
			\begin{center}
				\includegraphics[width=\textwidth]{Images/NoisySimNoise.pdf}
			\end{center}
				\caption[Simulation Point-By-Point Difference Distributions With Artificial Noise]{\label{fig:NoisySimNoise} Histograms of the point-by-point difference between two sets of simulated data with artificially added Gaussian noise.
					Each set of axis is for a different time after the start of S phase (labeled).
					y-axis shows the normalized distribution.
					x-axis shows difference.
					Grey circles are calculated from experiment.
					Black lines show the best Gaussian distribution fit.
				}
		\end{figure}
		
		Figure~\ref{fig:correlation}B shows the autocorrelation function for the simulated data with artificial noise.
		With the addition of the noise, we have acquired the delta function at $x=0$ but lost much of the long-range order.
		To understand the effect that creates the long-range order in experiment, and potentially improve our simulation, we tried a second approach to creating noise.
		This new approach, called the ``sequencing simulation,'' simulates 1000 cells, records only one tenth of the data, and does not add any artificial noise.
		With this method, the simulation is closer to sequencing experiments which pull 50 kb sequences from an effectively infinite population.
		Analysis of the point-by-point difference shows a similar evolution from Laplace distributed noise to Gaussian (figure not shown).
		The estimated standard deviations, shown as black squares in Fig.~\ref{fig:Noise}, are closer to the experimental estimates than our initial simulation but do not coincide.
		However, as shown in Fig.~\ref{fig:correlation}D, the correlation length is shorter in this case than in the simulation with added noise.
		
		In the end, we chose to artificially add noise to a simulation of 100 cells.
		Adding artificial noise has three main benefits:
		First, it effectively increases the uncertainty in the simulated data to match that seen experimentally.
		Second, the artificial noise pushes the differences between simulations closer to the Gaussian distribution we observe in experimental data.
		Third, it is about 10 times faster than simulating 1000 cells and taking one tenth of the data.
		Figure~\ref{fig:SimulatedExample} shows a comparison of simulated data before and after noise has been added artificially.
		Figure~\ref{fig:NoisySimNoise} shows the distribution of noise, measured as outlined in Sec.~\ref{subsec:SequencingNoise}.
		These distributions are a much better match to those shown in Fig.~\ref{fig:HistDifference} with the addition of Gaussian noise.
		The estimated standard deviations from the simulations with artificial noise (shown as dots in Fig.~\ref{fig:Noise}) are within one percent of those estimated from experimental data.
		
		To better include noise in simulations, the two features discussed above need to be addressed.
		We believe a better understanding of the experimental procedure and its sources of error would help with both of these.
		Given that the analysis above shows adding Gaussian noise makes the simulation noise match experimental noise much more closely, we believe the presented method is an effective first approach to incorporating noise in the MIM simulator.











































\chapter{Results}
\label{ch:Results}

In this chapter, we outline our investigations into the effect of small $n$ on the Multiple Initiator Model.
Using the MIM simulator described in Ch.~\ref{ch:Methods}, we performed four major investigations.

Our preliminary investigation was a single-origin comparison between the analytical MIM and the MIM Simulator.
We defined a parameter that measures the difference between the replicated fraction from simulation and from the MIM.
In Sec.~\ref{subsec:earlywork}, our so-called ``difference parameter'' shows that small $n$ does create a disagreement between the analytical MIM and the MIM simulator proportional to $n^{-1}$.
The presence of this error and its large tail motivated further study into how small $n$ affects the MIM.

In our second investigation, our goal was to develop a new metric for measuring the error in the MIM at low $n$.
The difference parameter does not scale to more than one origin.
Section~\ref{subsec:BiasedFits} outlines the new method, which consists of simulating the replicated fraction for a single origin of fixed $n$, followed by using the MIM to find the value of $n$ that best fits the simulated data.
The results of this investigation show that our first approach, while qualitatively in agreement, overestimates the difference between the MIM and the simulation.

Third, we progressed to simulating and fitting the more complex Chromosome I.
For this investigation, we started by fitting parameters with the MIM to data from DNA sequencing~\cite{StochasticTermination}.
We fit two sets of parameters by fixing $t_{1/2}$ as high and low
In this way, we forced the MIM to produce small and high $n$ respectively.
Using the parameters from the fits, we then simulated the replication of Chromosome I.
By calculating the root-mean-squared difference between simulated data and experimental data we showed that the two simulations are effectively indistinguishable from each other.
Further, analysis of the fit parameters shows that the small-$n$ values are proportional to the large-$n$ values.

The surprising results from our third investigation motivated additional work that we use to explain why the MIM is inaccurate for a single origin but produces good chromosome-wide results.
We expanded our single-origin analysis to a genome with two origins and more and show that multiple origins interact to reduce the inaccuracy in the MIM.

	\section{Single-Origin Investigations}
	\label{sec:SingleOrigin}
	
	Our early work consisted of single-origin simulations intended to motivate further study.
	In 2014, we received the results discussed in Sec.~\ref{sec:ExperimentsMIM} from N.~Rhind's lab~\cite{Rhind}.
	As mentioned, these results called into question the assumption that $n$ is large enough to ignore variations in $N$.
	Our first goal, therefore, was to discover whether or not simulations of small $n$ agreed with the MIM predictions for small $n$.
	In this naive investigation, discussed further below, we developed the ``difference parameter,'' a metric to measure the difference between the simulated and predicted $f(x,t,n)$ of a single origin.
	The investigation shows that the difference parameter decreases as $n^{-1}$ and motivated the deeper research presented in this thesis.
	
	The single-origin investigations that followed our preliminary work consisted of simulating $f(n)$ and fitting the MIM parameters to the result.
	Thus our metric charged from the difference parameter to a comparison between the simulated $n_\text{sim}$ and the fitted $n_\text{fit}$.
	With these investigations, we refined the simulation program to run more efficiently, and to create data similar to that measured in sequencing experiments.
	We show that our new metric reveals a qualitatively similar behaviour for the MIM; the difference between $n_\text{sim}$ and  $n_\text{fit}$ goes approximately as $n^{-1/2}$.
	We attribute the change in the rate to the change in how the metric is defined.
		
	\begin{figure}[tbh!]
		\begin{center}
			\includegraphics[width=\textwidth]{Images/DifferenceParameterGraph.pdf}
		\end{center}
			\caption[Schematic of the Difference Parameter Calculations]{\label{fig:DifferenceParameter} Schematic of the difference parameter calculations.
				Coloured lines represent the difference parameter for different values of $n$.
				Triangles show the parameter values in the corresponding inset graphs.
				\textbf{A.} Replication fraction simulation and theory curves for $n=10$; $DP=0.037$.
				\textbf{B.} Replication fraction simulation and theory curves for $n=5$; $DP = 0.062$.
				\textbf{C.} Replication fraction simulation and theory curves for $n=1$; $DP = 0.264$.
				Also illustrated in \textbf{C} is the value $\Delta P(x)$ at the peak.
				Note that $\Delta P(x)$ is defined over the entire domain.
				}
	\end{figure}
	
	
		\subsection{The Difference Parameter}
		\label{subsec:earlywork}
		
		When starting this project, we performed a quick investigation into the difference between $f_\text{sim}(n)$ and $f_\text{MIM}(n)$ for a single origin.
		To generate $f_\text{sim}(n)$, the MIM simulator calculated the average replicated fraction for about $10^6$ sequences, producing data with very little statistical error (in contrast to the noisy data shown in Sec~\ref{sec:Noise}.
		The global parameters used for the simulation were set equal to those measured previously by fitting the MIM to microarray measurements of budding yeast~\cite{ScottsPaper}.
		The difference parameter was then defined as
		\begin{equation} \label{DifferenceParameter}
			DP = \max_{x} {\frac {\Delta P(x)} {P}} \text{ ,}
		\end{equation}
		where $\Delta P(x)$ is the difference between $f_\text{sim}(n)$ and $f_\text{MIM}(n)$ (illustrated in Fig.~\ref{fig:DifferenceParameter}), and $P$ is the peak value of $f_\text{MIM}(n)$.
		
		Figure~\ref{fig:DifferenceParameter} shows the analysis process of the difference parameter for a single origin.
		First, $f_\text{sim}(n)$ and $f_\text{MIM}(n)$ were calculated over several time steps and $n$ ranging from one to 128, example replicated fractions are shown in the insets of Fig.~\ref{fig:DifferenceParameter}.
		From these data, we calculated $DP(n,t)$, shown in the main graph of Fig.~\ref{fig:DifferenceParameter}.
		Note the value for $DP$ ``saturates'' as $t$ increases, we call this value the ``saturated difference parameter.''
		Figure~\ref{fig:SaturatedDifferenceParameter} shows\footnote{
		The reader may notice a change in the $n$ values displayed in the graphics. Figure~\ref{fig:DifferenceParameter} the saturated difference parameter as a function of $n$ is illustrative and contains old data; accurate, but not useful.
		After producing that graph we used a slightly different set of parameters in the simulation and changed the range of $n$ simulated when producing Fig.~\ref{fig:SaturatedDifferenceParameter}}.
		This initial investigation showed the saturated difference parameter decreases as $n^{-1}$.
		
		\begin{figure}[tbh]
			\begin{center}
				\includegraphics[width=0.8\textwidth]{Images/SaturatedDifferenceParameterGraph.pdf}
			\end{center}
				\caption[Saturated Difference Parameter vs. $n$]{\label{fig:SaturatedDifferenceParameter} Saturated difference parameter vs. $n$.
				The large uncertainty in low $n$ arises because the time-steps not going far enough to accurately measure the saturated difference parameter.
				The line is proportional to $n^{-1}$ ($\text{[Saturated Difference Parameter]} \approx \frac{0.3}{n}$).
				}
		\end{figure}
		
		From our initial investigation, we concluded that the difference parameter grows quickly with decreasing $n$.
		Therefore, we suspected that the MIM will not produce accurate results in the case that $n$ is small.
		These results motivated in-depth research into the effect of small $n$ on the MIM.
		
		
		\subsection{Biased Fits}
		\label{subsec:BiasedFits}
		
		The definition of the difference parameter does not scale to more than one origin.
		The saturated difference parameter is only measurable when the theory curve has a peak value near one.
		Additionally, the difference parameter is defined as a single value for the whole simulated genome; therefore, we cannot infer anything about more than one origin.
		Thus, we developed a second investigation that measures how the parameters fitted with the MIM to a simulation of small $n$ may be biased.
		
		\begin{figure}[tbh]
			\begin{center}
				\includegraphics[width=0.47\textwidth]{Images/LargePopBias.pdf}
			\end{center}
			\caption[Bias in MIM fit on Large-Population Simulations]{\label{fig:LargePopulation} Scatter plot of $n_\text{sim}$ vs. $n_\text{fit}$ for simulations of a large population.
			Red circles show the data. Dashed line shows unity.
			}
		\end{figure}
		
		The process of this investigation was to calculate $f_\text{sim}(n)$ followed that by fitting the parameters of the MIM to the result.
		Thus, we have two parameters: $n_\text{sim}$, the value for $n$ input to the simulation, and $n_\text{fit}$, the value for $n$ that results from the MIM fit.
		Initially, we performed these measurements using simulations of large populations sets of sequences (about $10^6$ and more 100 kb sequences).
		Figure~\ref{fig:LargePopPopulation} shows the preliminary results from our biased fit investigation on large-population simulations.
		The graph is a scatter plot of $n_\text{sim}$ vs. $n_\text{MIM}$, the dashed line shows unity.
		As we expected, the biased is relatively large for low $n_\text{sim}$, but decreases as $n_\text{sim}$ grows.
		However, even using a C++ module to increase performance, these simulations were slow and were far more accurate than the current experimental standard (Sec.~\ref{sec:Noise}).
		Therefore, we limited our simulations as described in the previous chapter; in this way we simulated data comparable to those generated experimentally.
		
		\begin{figure}[tbh]
			\begin{center}
				\includegraphics[width=0.8\textwidth]{Images/NoisyBias.pdf}
			\end{center}
				\caption[Bias in MIM Fit to Noisy Data]{\label{fig:NoisyBias} Scatter plot of $n_\text{sim}$ vs. $n_\text{fit}$ for noisy simulations of a single origin.
				Red dots show data from 50 simulations at each value $n_\text{sim}$.
				Blue circles with error bars show the mean and standard deviation of the mean.
				Dashed line shows unity.
				\textbf{Inset.} Scatter plot of the percent difference ($[n_\text{sim} - n_\text{fit}]/n_\text{sim}$) vs. $n_\text{sim}$.\\
				$\text{[Dashed line]} = 0.315/\sqrt{n_\text{sim}}$.
				}
		\end{figure}
		
		In Sec.~\ref{sec:Noise} we outlined the process used to generate noisy data.
		We used the MIM Simulator to generate $f_\text{sim}(n_\text{sim})$ with noise and used the MIM to fit the parameters to that data.
		Figure~\ref{fig:NoisyBias} is a scatter plot of $n_{sim}$ vs. the resulting $n_\text{fit}$.
		In this case, because of the increased noise in the simulated data, we performed this procedure fifty times per $n_\text{sim}$ value (red dots).
		The blue circles show the mean for each value of $n_\text{sim}$.
		Here, again, we see that the bias is largest for small $n_\text{sim}$, and decreases $n_\text{sim}$ grows.
		In the inset to Fig.~\ref{fig:NoisyBias} we show the percent difference given by $[n_\text{sim} - n_\text{fit}]/n_\text{sim}$.
		The dashed line shows $0.315/\sqrt{n_\text{sim}}$, but that trend is presented only for comparison with the difference parameter; there is either no simple fit or we are underestimating the error.
		
		The data shown in Fig.~\ref{fig:NoisyBias}, which comes from fitting the MIM to artificially noisy simulated data, shows the behaviour we expect:
		The MIM works poorly when $n$ is small, and better with increasing $n$.
		This indicates that the MIM simulator is producing good data, that is comparable to both experimental data and the analytical MIM.
		With these results assuring us the program is sound, we continued our research exploring the same process on genomes with multiple origins.
		
		
	\section{Simulations of Chromosome I}
	\label{sec:ChromosomeI}
	
	Our results from single-origin simulations indicate that the MIM does not perform well in the small-$n$ regime for a single origin.
	However, in eukaryotes, origins are not alone: as we discussed in Ch.~\ref{ch:Introduction}, replication in eukaryotes starts at many origins.
	In this section, we investigate the replicated fraction of Chromosome I and do not observe the same reduction in accuracy.
	Following our single-origin investigation, we simulated the replicated fraction of Chromosome I of budding yeast.
	For this investigation, we used the same simulation method as described above for the noisy single-origin analysis, except that the genome size and origin parameters were chosen to represent Chromosome I.
	
	\begin{table}[tbh]
		\begin{center}
			\begin{tabular}{| r | c | c | c | c | c | c |}	
				\hline
				Position (kb)	&	38.3	&	72.6	&	124.2	&	155.6	&	174	&	216	\\	\hline
				Small $n$	&	1.65	&	1.93	&	2.3	&	2.5	&	5.7	&	1.5	\\
				Large $n$	&	10.2	&	12	&	14	&	16	&	36	&	9	\\	\hline
			\end{tabular}
		\end{center}
		
		\caption[High and low $n$ fit values for Chromosome I]{\label{tab:LargeAndSmallN}
			High and low fitted values for $n$ on Chromosome I.
			Top row shows the fixed positions of the six fitted origins.
			Center row shows the values of $n$ when $t_{1/2}=40$.
			Bottom row shows $n$ from a fit with $t_{1/2}=90$.
			Small $n$ vs Large $n$ is plotted in Fig.~\ref{fig:ChrISims}
		}
	\end{table}
		
	\begin{figure}[tbh]
		\begin{center}
			\includegraphics[width=\textwidth]{Images/ChrIExpAndSim.pdf}
		\end{center}
			\caption[Experimental and Simulated Replicated Fraction of Chromosome I]{\label{fig:ThisFigure} 
				The replicated fraction of Chromosome I from experimental data~(\cite{StochasticTermination} supplementary data), and simulations with high $n$ and low $n$.
			}
	\end{figure} 

	The origin parameters were set by fitting the MIM to the replicated fraction for wild-type budding yeast reported by Hawkins~\emph{et al.}~\cite{StochasticTermination}.
	To test the effect of small $n$, the fitted parameter $t_{1/2}$ was fixed at a high value (90 minutes) to produce high values for $n$, and, at a low value (40 minutes) to produce low values for $n$.
	To be sure that any effects we observed were due only to the fitted values of $n$, the origin positions were fitted once, then held constant for the second fit.
	The resulting values for $n$ at each origin can be seen in Tab.~\ref{tab:LargeAndSmallN}.
		
	\begin{figure}[tbh]
		\begin{center}
			\includegraphics[width=\textwidth]{Images/RMSDiff.pdf}
		\end{center}
			\caption[Root Mean Square Difference Between Simulations and Experimental Data]{\label{fig:RMSDiff} 
				Analysis of $\Delta f_\text{rms}$.
				\textbf{A.} Plot of $\Delta f_\text{rms}(x)$ for six time-steps between high $n$ simulations and experimental data.
				The gradient goes from light grey (15 min) to black (40 min).
				\textbf{B.} Same as \textbf{A} between low $n$ simulations and experimental data.
				\textbf{C.} $\Delta f_\text{rms}$ averages over the genome vs time since the start of S phase.
			}
	\end{figure} 
		
	\begin{figure}[tbh!]
		\begin{center}
			\includegraphics[width=\textwidth]{Images/LowVsHighAndPercDiffVsX.pdf}
		\end{center}
			\caption[Low-$n$ Fit Values vs. High-$n$ Fit Values and Percentage Difference Over the Genome]{\label{fig:ChrISims}
				\textbf{A.} Scatter plot of Low-$n$ fit values vs. high-$n$ fit values.
				The line shows the best linear fit, $n_\text{low} \approx 0.16\times n_\text{high}$.
				\textbf{B.} Scatter plot of the percentage difference shown in Figs.~\ref{fig:ChrIFitVsSim}B and D vs the positions of the origins.
				Dashed line is zero.
				Labels show $n_\text{sim}$.
			}
	\end{figure} 
	
	We simulated the two sets of parameters that resulted from the high-$n$ and low-$n$ fits.
	The resulting replicated fractions are shown in Fig.~\ref{fig:ThisFigure}.
	To quantitatively measure the quality of the two fits, we calculated the root mean squared difference $\Delta f_\text{rms}$ between each replicated fraction and the experimental data.
	We simulated each set of parameters fifty times and averaged $\Delta f_\text{rms}$ at each time step over the fifty simulations.
	Figures~\ref{fig:RMSDiff}A and B show $\Delta f_\text{rms}(x,t)$ for high and low $n$ respectively and Fig.~\ref{fig:RMSDiff}C shows the average value for each time step, $\Delta f_\text{rms}(t)$.
	Surprisingly, and in contrast to the results presented so far, these data imply that the quality of the MIM fits is nearly identical\footnote{
	We observe a trend for high-$n$ simulation to be slightly more accurate than low-$n$ simulations; however, the error bars show the standard deviation of the mean of 50 simulations. Thus, the trend is well within the noise of a single measurement.}
	for large and small $n$.
		
	Further, we were interested in the relationship between the large-$n$ values from the fit and the small-$n$ values from the fit.
	The MIM does not claim that $n$ is the absolute number of initiators on an origin; rather, the fitted parameter should be proportional to the absolute number of initiators.
	If this is true, the values for low-$n$ and high-$n$ should be linearly related.
	Our suspicion, based on the results of our single-origin investigation, was that the relationship would not be linear, but go approximately as $n^{-1/2}$.
	However, our results from fitting to Chromosome I seem to contradict this suspicion.
	Indeed, Fig.~\ref{fig:ChrISims}A shows that the relationship is linear: $n_\text{small} \propto n_\text{large}$.
		
	\begin{figure}[tbh!]
		\begin{center}
			\includegraphics[width=\textwidth]{Images/ChrIFitVsSim.pdf}
		\end{center}
			\caption[Chromosome I $n_\text{fit}$ vs. $n_\text{sim}$ and Percentage Difference]{\label{fig:ChrIFitVsSim}
				$n_\text{fit}$ vs. $n_\text{sim}$ and percentage difference for high $n$ and low $n$ simulations of Chromosome I.
				\textbf{A.}~Scatter plot of $n_\text{fit}$ vs. $n_\text{sim}$ for high $n$.
				Dashed line is unity.
				\textbf{B.}  Scatter plot of the percentage difference between $n_\text{fit}$ and $n_\text{sim}$ vs. $n_\text{sim}$ for high $n$.
				Dashed line is zero.
				\textbf{C - D.} Same as \textbf{A} and \textbf{B} respectively for low $n$.
			}
	\end{figure} 
	
	Using the same technique as outline in Sec.\ref{subsec:BiasedFits}, we used the MIM to calculate $n_\text{fit}$ values for each origin that we simulated on Chromosome I.
	For the six origins we simulated, we calculated the percentage difference between $n_\text{fit}$ and $n_\text{sim}$ (shown in Figs.~\ref{fig:ChrIFitVsSim} B and D).
	The trend we observe in the single origin case of decreasing percentage difference with increasing $n$ is no longer present.
	In the case of Chromosome I, our measurements indicate that the MIM is approximately equally effective for both high-$n$ and low-$n$.
	
	We suspected that the spatial organization of the origins in Chromosome I played a role in the observed equality in the fit.
	Therefore, Fig.~\ref{fig:ChrISims}B shows the percentage difference as a function of location within the genome.
	We do not observe any meaningful pattern in this plot.
	
	
	\section{Neighbouring Origins Reduce the effect of Small $n$}
	\label{sec: NearOrigins}
	
	The results shown in Sec.~\ref{sec:ChromosomeI} appear to contradict the results from the previous, single-origin investigations.
	Therefore, we ask the question: \emph{What is different between the single-origin simulations and the simulations of Chromosome I?}
	The immediate answer is that the number of origins has changed from one to six.
	Perhaps then, multiple origins cooperate somehow to reduce the effect of small $n$ on the predictive power of the MIM.
	Here we show our investigations of genomes with more than one origin.
	
		\subsection{Two-Origin Investigation}
		\label{subsec:TwoOrigins}
		
		To test our hypothesis that multiple origins cooperate to increase the efficacy of the MIM in the small-$n$ regime, we expanded our single-origin investigation to a genome with two origins.
		With the addition of a second origin, there is a new consideration: By what distance should the two origins be separated?
		The obvious maximum distance the origins should be separated is $2vt_\text{sim}^\text{(max)}$, because if they are any further apart, their replicated regions will never overlap.
		For our simulations, that is 180 kb.
		However, it is very rare for an origin to fire at the start of S phase, so placing the two origins 180 kb apart is not wise.
		We looked to the fitted locations of origins on Chromosome I shown in Tab.~\ref{tab:LargeAndSmallN} as a guide.
		Here, we see the two closest origins are 18.4 kb separated, and the two furthest origins are 54.2 kb separated.
		Therefore, we simulated two origins with equal $n_\text{sim}$ spaced both 18.4 kb apart and 52.4 kb apart.
		
	\begin{figure}[tbh]
		\begin{center}
			\includegraphics[width=\textwidth]{Images/TwoOriginPercDiff.pdf}
		\end{center}
			\caption[Scatter Plots of Single-and Two-Origin Percent Difference]{\label{fig:TwoOrigins} 
				Scatter plots of single-and two-origin percent difference between $n_\text{fit}$ and $n_\text{sim}$.
				\textbf{A.} Percent difference vs. $n_\text{sim}$ for a single origin.
				\textbf{B.} Same as \textbf{A} for two near origins (18.4 kb separation)
				\textbf{C.} Same as \textbf{A} for two distant origins (54.2 kb separation)
			}
	\end{figure} 
		
		Figure~\ref{fig:TwoOrigins} shows the percent difference between $n_\text{sim}$ and $n_\text{fit}$ resulting from these simulations for $n$ ranging from 2 to 64 Compared to single-origins simulations.
		Contrary to our hypothesis, these results show that two origins are fitted less accurately than single origins.
		
		
		\subsection{Multiple-Origin Investigation}
		
		The results so far have been surprisingly contradictory:
		In the single-origin case, the MIM fits get progressively worse as $n$ decreases.
		In the many-origin case of Chromosome I, the MIM fits are equally accurate for both high $n$ and low $n$.
		In the two-origin case, the MIM fits are less accurate than the single-origin case for low $n$.
		
	\begin{figure}[tbh]
		\begin{center}
			\includegraphics[width=0.8\textwidth]{Images/ChromosomeI.png}
		\end{center}
			\caption[Sketch of Sub-sequences of Chromosome I]{\label{fig:ChrISubSeq} 
				Sketch of sub-sequences of Chromosome I used in simulations.
				Vertical grey lines denote chosen edge points for sub-sequences.
				Circles represent origins.
				Horizontal black lines represent sub-sequences (number of origins labeled and the full chromosome (labeled).
			}
	\end{figure} 

		We would like to see a transition from the inaccurate 2-origin system to the accurate 6-origin case.
		To accomplish this, we chose to simulate sub-sequences of Chromosome I containing 2 origins, 3 origins, and 4 origins.
		Figure~\ref{fig:ChrISubSeq} illustrates how the sub-sequences were selected, with divisions occurring directly in the middle\footnote{
		Due to a calculation error, the second division was 4 kb off the midpoint. This should not have a strong impact on the results.}
		of neighbouring origins.
		There may be better selection criteria for where the endpoints of sub-sequences should fall;
		for example, the fifth origin has higher $n$ than both of its neighbours, and therefore it will have a larger region of influence than them.
		However, as the results will show, in addition to its simplicity, this method is effective.
		
	\begin{figure}[tbh]
		\begin{center}
			\includegraphics[width=0.8\textwidth]{Images/MultipleOrigins.pdf}
		\end{center}
			\caption[Scatter Plots of $n_\text{fit}$ vs. $n_\text{sim}$ for Two, Three and Four Origins]{\label{fig:IncreasingOrigins} 
				Scatter Plots of $n_\text{fit}$ vs. $n_\text{sim}$ for two, three and four origins for high $n$ and low $n$.
				Open circles show low-$n$ data.
				Black circles show high-$n$ data.
				Dotted lines in right-most graphs show a single sub-sequence of four origins.
				Dashed lines show unity.
			}
	\end{figure} 
		
		We simulated the sub-sequences of Chromosome I using the $n$ values shown in Tab.~\ref{tab:LargeAndSmallN}.
		Figure~\ref{fig:IncreasingOrigins} shows a scatter plot of $n_\text{fit}$ vs. $n\text{sim}$ for two-origin (left), three-origin (middle), and four-origin (right) sub-sequences (labeled with a star in Fig.~\ref{fig:ChrISubSeq}) of Chromosome I in the high-$n$ (black circles) and low-$n$ (open circles) regimes.
		The two dotted lines show the four origins on a single sub-sequence in order.
		In this data we observe two trends:
		First, as the number of origins in the sequence increases, the MIM fits grow in accuracy.
		While between high $n$ and low $n$, the accuracy is still better in the high-$n$ regime, we suspect that as $n$ grows the difference will become negligible.
		Second, as illustrated by the dotted lines in Fig.~\ref{fig:IncreasingOrigins}, we observe that origins in the center of the sequence (surrounded by origins) are fitted more accurately than origins on the edges, with a single neighbour only.
		These two trends appear to work together to make the MIM accurate for all values of $n$ on sequences with many origins.
		







































\chapter{Conclusions}
\label{ch:Conclusions}

In this thesis, we used the MIM simulator to explore the effect of variation in initiation factors on the efficacy of the analytical MIM.
We presented several investigations ranging in complexity from a single origin to Chromosome I of budding yeast.
From our investigations we concluded that the inferences made with the MIM remain accurate in the case that the number of initiators is lower than first assumed.

Our research was motivated by a recent experiment measuring low numbers of loaded initiators which contradicts the assumption made in the analytical MIM.
Naive interpretations of the MIM suggest that the MIM should fail when the number of initiators is low.
We started our investigations by analyzing simple single-origin genomes.
The results of these single-origin studies confirmed our suspicions:
Inferences made with the MIM become less accurate as the number of initiators decreases.
In contrast, simulations of Chromosome I of budding yeast showed that these inferences are accurate when the number of initiators is low.
To understand this contradiction, we continued our research by simulating sequences with multiple origins.
From these simulations we observed two trends that explain the transition from inaccuracy for single origins to accuracy for several origins.
The first trend is that the overall accuracy of MIM inferences increases with the number of origins.
The second trend is that the accuracy is greater for origins in the middle of a cluster of origins than origins at the edges (i.e., origins with two neighbours are handled better than origins with only one).
These two trends contribute to our conclusion that inferences made with the MIM are equally accurate for any number of initiators.

Our conclusion is that inferences made with the MIM maintain their accuracy for small numbers of initiators when there are many origins.
We have provided a qualitative analysis of multiple-origin simulations that shows that as the number of origins increases, so too does the accuracy.
The positive outcome of this research is that we now have increased confidence in the MIM approach to analyzing DNA replication data from experiments.

	\section{Future Considerations}
	
		\subsection{Quantitative Analysis}
		
		It is apparent that the next step in this research is to perform a quantitative analysis of how the contribution of multiple origins act to mitigate the inaccuracy in inferences made by the MIM due to fluctuations in initiation factors.
		Such research could explore two features of our results:
		first, a quantification of the accuracy as a function of number of origins and number of initiators;
		second, an exploration of how origin spacing and relative initiation factors between origins in a sequence affect inferences made by the MIM.
		A successful investigation along these lines would help future researchers quantify the accuracy of their measurements made with the MIM for a given genome.
		
		
		\subsection{Analysis of Other Organisms}
		
		The research presented here was based on the model organism \emph{Saccharomyces cerevisiae}.
		This was a deliberate choice: because the origins of replication in \emph{S.~cerevisiae} are confined to known locations, the complexity of the replication process is significantly reduced.
		However, as we discussed in Ch.~\ref{ch:Introduction}, \emph{S.~cerevisiae} is a special case, and it is far more common for origins to be located diffusely in a region.
		Therefore, research expanding the MIM to address stochastically located origins would dramatically increase the number of organisms it can analyze.
		
		
		\subsection{Toward a Biological Research Tool}
		
		We believe that in the long-term  the culmination of this research will be the development of a tool for biological research.
		If the studies described above of multiple origins and stochastically located origins are successful, the MIM could form the basis of a research tool for DNA replication.
		This tool would be used by researchers performing studies of DNA replication to quickly make inferences about the number of initiators loaded on the DNA.
		
		
		\subsection{Comments About the Simulator}
		
		In our investigations we created a modular program that simulates DNA replication.
		This simulation makes use of the KJMA framework discussed in Sec.~\ref{subsec:KJMA}.
		As we mentioned in our discussion of the KJMA, it is a mathematical framework that has a broad range of applications.
		Thus, the MIM simulator can address problems described by the KJMA in one dimension with the addition or replacement of modules within the program.
		Therefore, there are many new and peripherally related avenues of research that can be investigated with use of our MIM simulator.
		



%  BACK MATTER  %%%%%%%%%%%%%%%%%%%%%%%%%%%%%%%%%%%%%%%%%%%%%%%%%%%%%%%%%
%
%  References and appendices. Appendices come after the bibliography and
%  should be in the order that they are referred to in the text.
%

\backmatter%
	\addtoToC{Bibliography}
	\bibliographystyle{bibstyle/Thesis-Bib}%unsrt} %% default style is {plain}, but it outputs the list alphabetically by first author, unsrt will order the list by citation order
	%% Change the name ``references'' to the name of your .bib file (in the same folder as this file).
	\bibliography{references}

%% To add additional appendicies, merely add \chapters
%\begin{appendices} % optional

%\chapter{The replication time of \emph{E. coli}}
\label{ap:EColi}

To estimate the time required for complete replication of \emph{E. coli} DNA some information is required.
First, the length of the genome is $L = 4600 kb$.
Second, the replication forks propagate at an average speed of $v = 1000 bp/s = 1 kb/s$.
Third, the DNA is organized in a loop, so there is a single origin site and a single termination site exactly opposite. \cite{EColi}

With this information, it is a quick calculation to estimate the time it takes for a single fork to traverse the genome, $t_1$:
$$
	t_1 = {{L}\over{v}} = {{4600 kb}\over{1 kb/s}} = 4600 s.
$$
However, two forks are used to replicate \emph{E. coli} DNA, so the time required to fully replicate the genome, $t$, is:
$$
	t = t_1/2 = 2300 s,
$$
which is just over 38 minutes.

Therefore, the time required for \emph{E. coli} to fully replicate its genome is about 40 minutes.

%\end{appendices}

% You can choose to add an index to your thesis as well. To do so, make sure you \addtoToC{Index}. You'll also need to look up how to include items in the index.
\end{document}
