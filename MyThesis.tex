\documentclass[serif]{sfuthesis}

%% Fill in below, the values here are fine for a physics masters (just put the name of the thesis and your name and update the date and previous degrees info).

%% No quotes in the title, do NOT end it with a period
\title{Variance in Initiation Factors Does Not Strongly Affect the Replication Profile of Budding Yeast DNA}
\thesistype{Thesis}
\author{Mike Chomitz}
\previousdegrees{%
	B.Sc., Trent University, 2012}
\degree{Master of Science}
\discipline{Physics}
\department{Department of Physics}
\faculty{Faculty of Science}
\copyrightyear{2015}
\semester{Fall 2015}
\date{August 19, 2015}

%% Choose words or phrases that will help people locate your thesis via search tools (library catalogues, Google, etc.)
%% Keywords appear under the abstract and must be entered into the Thesis Registration System.
\keywords{DNA replication kinetics; multiple initiator model; phantom nuclei; Poisson process}


\committee{%
	\chair{Dr.\ Some Body}{POSITION IN DEPARTMENT}
	\member{Dr.\ John Bechhoefer}{Senior Supervisor\\Professor}
	\member{Dr.\ Malcolm Kennet}{Supervisor\\Associate Professor}
	\member{Dr.\ Eldon Emberly}{Internal Examiner\\Associate Professor}
%% add or remove \members as need
}

%  PACKAGES AND CUSTOMIZATIONS  %%%%%%%%%%%%%%%%%%%%%%%%%%%%%%%%%%%%%%%%%
%
%  You don't need to call the following packages, which are already called
%  in the sfuthesis class file:
%  - etoolbox
%  - setspace
%  - enumitem
%  - pdfpages
%  - tocloft
%  - appendix
%  - cmbright or lmodern
%
%  If you call one of the above packages (or one of their dependencies)
%  with different options, you may get a "Option clash" LaTeX error.
%  If you get this error, you can fix it by removing your copy of
%  \usepackage and just passing the options you need by adding
%
%  \PassOptionsToPackage{<options>}{<package>}
%
%  before \documentclass{sfuthesis}.
%

\usepackage{amsmath,amssymb,amsthm}
\usepackage{fixltx2e}
\usepackage[rightcaption]{sidecap}



%  FRONTMATTER  %%%%%%%%%%%%%%%%%%%%%%%%%%%%%%%%%%%%%%%%%%%%%%%%%%%%%%%%%
%
%  Title page, committee page, copyright declaration, abstract,
%  dedication, acknowledgements, table of contents, etc.
%

\begin{document}

\frontmatter
\maketitle %% The details of this command are in sfuthesis.cls. Line 299
\makecommittee %% The details of this command are in sfuthesis.cls. Line 336
\makecopyrightdeclaration %% This page should stay as is, except it needs to be updated to the latest version of the declaration of partial copywrite.
				%% To update, just download the latest version, name it declaration_of_partial_copyright_licence.pdf and make sure it is saved in the
				%% Same folder as this file.

\begin{abstract}
	DNA replication starts at many sites (origins) throughout eukaryotic DNA.
	To fully understand the replication program in higher organisms, one needs to understand the behaviour of these origins.
	In \emph{Saccharomyces cerevisiae} (budding yeast), the spatial organization of the origins is simple: The origins are confined to known specific sites on the genome.
	The temporal behaviour of origins in budding yeast is more complex: They fire stochastically with a broad distribution of firing times.
	
	Several key proteins take part in the DNA replication process.
	The MCM2-7 hexamer in particular forms a helicase that unwinds DNA locally, allowing access for other proteins to replicate the separated DNA.
	Past analysis of the budding yeast replication program suggested the Multiple Initiator Model (MIM), which hypothesizes that the number of these MCMs loaded at an origin predicts the firing time of that origin.
	Part of the MIM formalism assumes that the number of loaded MCMs is large; in this case the relative fluctuations between cells will be small and are ignored.
	However, a recent experiment measuring the number of loaded MCMs has revealed that the number is low, and thus, cell-to-cell fluctuations may be larger than expected.
	The purpose of this thesis is to investigate the impact of large relative fluctuations in the number of MCMs on the MIM.
	To measure this effect, we built the ``MIM simulator,'' a modular program that simulates the replication process.
	Although a naive argument suggests that the MIM should fail when the number of MCMs is low, the impact of these fluctuations is mitigated by the contributions from neighbouring origins.
	We conclude that inferences made with the MIM remain accurate in the case that the number of MCMs is lower than first assumed.
\end{abstract}


\begin{dedication} % optional
	For anyone reading this, may you find what you're looking for.
\end{dedication}


\begin{acknowledgements} % optional
	UPDATE THIS
\end{acknowledgements}

%% making automatically managed pages: the table of contents, and lists of tables and figures.
%% To change the table of contents, check the sfuthesis.cls file. The code for the table of contents starts at line 493.
\addtoToC{Table of Contents}\tableofcontents\clearpage
\addtoToC{List of Tables}\listoftables\clearpage
\addtoToC{List of Figures}\listoffigures





%  MAIN MATTER  %%%%%%%%%%%%%%%%%%%%%%%%%%%%%%%%%%%%%%%%%%%%%%%%%%%%%%%%%
%
%  Start writing your thesis --- or start \include ing chapters --- here.
%

\mainmatter%

\chapter{Introduction}
\label{ch:Introduction}

The timely and accurate replication of DNA is critical for maintaining genetic integrity in cellular life.
In simple cells (``prokaryotes''), the process used to replicate the genome (the ``replication program'') is well understood.
Starting at a sequence-defined location (the ``origin''), the double-stranded DNA (dsDNA) of the prokaryotic genome separates into two single-stranded DNA (ssDNA) segments.
Complex biological machinery travels bidirectionally from the origin, separating the dsDNA into growing ssDNA segments, which are used as templates for the creation of two copies of the original genome.
The machinery between separated and non-separated DNA (``forks'') continues to propagate through the genome until it has been entirely separated and replicated.
Prokaryotic organisms have such small genomes that replication can be completed using a single origin~\cite{MolecularCellBiology}.
With this replication program, an \emph{E.~coli} bacterium can replicate its entire genome in about 40 minutes\footnote{
Under fast-growth conditions, the time to replicate the genome, $T=\frac{L}{2v}=\frac{4600\text{ kb}}{2\times1\text{ kb/s}} = 2300\text{ s}$.~\cite{EColi}}.

More complex organisms (``eukaryotes'') have genomes that are approximately 1000 times longer and have forks that propagate about 10 times slower than prokaryotes.
For example, compare the human genome, about 3000 Mb~\cite{HumanGenomeLength} with mean fork speed 1.5 kb/min~\cite{HumanForks}, to \emph{E. coli}, about 4600 kb long with fork speed 1 kb/s~\cite{EColi}.
The replication forks from a single origin would take nearly 4 years to completely replicate the entire human genome.
In many human cells, DNA replicates in about 8 hours~\cite{CellMolApproach}.
This is much less than the four years that would be needed if there were only a single active origin of replication.
Therefore, a single origin cannot be solely responsible for the replication of eukaryotic DNA.
Eukaryotic DNA is thus replicated using a parallel process, with many origins along~\cite{eukaryotereview}.
A considerable body of experimental evidence suggests that the timing (initiation) of origins is stochastic~\cite{}\textbf{FIND SOURCES}

Using many origins in the replication program creates several non-obvious issues that must be addressed for the program to work effectively.
One issue is the existence of separate replicated regions during the replication process.
When the forks of two neighbouring replicated regions meet, the two coalesce into a single, larger replicated region.
Another is is the need to coordinate multiple origin-activation events that are driven by stochastic processes.
The replication program describes the process by which the DNA is replicated using many origins of replication.
These origins are somehow able to coordinate their stochastic initiation times such that it is rare for any part of the DNA to be left unreplicated by the end of S phase~\cite{eukaryotereview}.


	\section{The Cell Cycle}
	\label{sec:CellCycle}
	
	The cell cycle defines the steps taken during cellular reproduction and can be divided into four phases (see Fig.~\ref{fig:CellCycle})~\cite{MolecularCellBiology,CellMolApproach}:
	the first gap (G1) phase, the synthesis (S) phase, the second gap (G2) phase, and the mitosis (M) phase.
	The G1 phase of the cell cycle contain key processes that prepare the DNA for replication.
	During S phase (the second phase in the cell cycle) the DNA is replicated.
	The third phase of the cell cycle (G2) primarily acts as a buffer to ensure complete DNA replication.
	During the fourth phase of the cell cycle (M), the cell physically divides into two daughter cells.
	
	\begin{figure}[thb]
		\begin{center}
			\includegraphics[width=0.45\textwidth]{images/CellCycle.png}
		\end{center}
			\caption[The Four Phases of the Cell Cycle]{\label{fig:CellCycle} The cell cycle has four phases: 
				Mitosis (M), when a mother cell separates into two daughter cells; 
				the first Gap (G1), when the daughter cell undergoes growth and chemical preparation for DNA replication;
				Synthesis (S), when the DNA is replicated;
				and the second Gap (G2), which acts as a buffer to ensure complete replication before the M phase.}
	\end{figure}
	
	
		\subsection{The G1 Phase}
		\label{subsec:G1Phase}
		
		The G1 phase begins early in the life of a daughter cell, after the mother cell has divided in the preceding M phase.
		During this time, the cell grows and, more important, ``licensing'' is carried out to prepare for replication during S phase.
		
		Licensing occurs at the origin recognition complex (ORC)~\cite{DNAInitiation}, as seen in Fig.~\ref{fig:EarlyReplicationConcept}A.
		The ORC is made up of a single group of six proteins that bind to the DNA at an origin~\cite{ORC}.
		Two additional proteins (Cdc6 and Cdt1; left out of Fig.~\ref{fig:EarlyReplicationConcept}A) assist the ORC in recruiting minichromosome maintenance (MCM) 2-7 hexamer rings onto the DNA~\cite{DNARepInitiation}.
		Loaded hexamers form pairs oriented away from each other~\cite{MCMPairs}.
		Each such pair will later be referred to as ``potential initiators'', or just ``initiators''.
		After licensing, the resulting set of proteins associated with the origin is called the pre-replication complex (pre-RC).
		Licensing is suppressed during the S and G2 phases by cyclin-dependent kinases.
		This effectively limits the cell cycle to a single replication event~\cite{MolecularCellBiology}.
		
		
		\subsection{S Phase}
		\label{subsec:SPhase}
		
		The second phase in the cell cycle is S phase.
		After licensing is completed in the G1 phase, the copying of the genome occurs during S phase.
		There are three main processes that happen during S phase: initiation, elongation, and coalescence, shown in Fig.~\ref{fig:EarlyReplicationConcept}B.
		
		An origin initiates (or ``activates'', or ``fires'') during S phase when five other proteins bind to each in a pair of MCM2-7 rings:
		Cdc45 and the tetrameric GINS complex (only Cdc45 is shown in Fig.~\ref{fig:EarlyReplicationConcept}A).
		The total system of proteins is called the CMG complex (Cdc45, MCM2-7, GINS complex) and comprises a helicase that traverses the genome during S phase.
		As an origin fires, the pre-RC disassembles and the activated pair of helicases unwind, separating the double helix of the dsDNA into two complementary ssDNA chains~\cite{GINSComplex}.
		
		After an origin has been initiated, there is a small region of ssDNA bounded on either side by the CMG complex helicases.
		The locations where the dsDNA is separated into two ssDNA chains are called replication forks (or just ``forks'').
		Elongation is the process by which the replication forks, with the help of the biological machinery stored in the CMG complex~\cite{PurifiedProteins}, propagate bidirectionally from the origin, separating the dsDNA.
		As the forks propagate, DNA polymerases bind with the ssDNA between them.
		DNA polymerases use the ssDNA as a template and backbone for synthesizing dsDNA; essentially, it adds the missing half back onto the separated strand.
		DNA polymerase can only propagate in the $3'$ direction.
		This poses a problem:
		Because the two strands in dsDNA are oriented in opposing directions, the polymerase can smoothly traverse only one of the ssDNA chains (the ``leading strand'') at each fork.
		On the other strand (the ``lagging strand''), the polymerase ``stutters'':
		It replicates a small region in the direction opposite to that of fork propagation until it hits a region that has already been replicated.
		The DNA polymerase then leapfrogs over and past the region it just replicated in the direction of fork propagation and repeats.
		These small fragments are called Okizaki fragments.
		On the lagging strand, the Okizaki fragments are connected by DNA ligases~\cite{MolecularCellBiology, CellMolApproach, OriginsReview}.
		
		Finally, when two forks meet, coalescence occurs:
		The helicases disassemble, and the two regions of dsDNA are connected by DNA ligase~\cite{MolecularCellBiology}.
	
	\begin{figure}[tbh]
		\begin{center}
			\includegraphics[width=\textwidth]{images/EarlyReplicationConcept.png}
		\end{center}
			\caption[Schematic of G1 and S Phases]{\label{fig:EarlyReplicationConcept} Simplified schematic of the G1 and S phases of the cell cycle, containing only those parts that are necessary to understand the model presented in Ch.~[\textbf{CH 3}].\\
				\textbf{A.} In the G1 phase, origins are located and licensed when the ORC recruits pairs of MCM2-7 hexamers onto the dsDNA.
					During S phase, pairs of Mcm2-7 hexamers are activated by the Cdc45 protein.
					After activation, the resulting structures become replicative forks that traverse the DNA unwinding the dsDNA allowing the DNA to be replicated.\\
				\textbf{B.} More detailed view of S phase.
					At the start of S phase, several origins are licensed along the genome (top of image).
					As time progresses (down), origins fire independently (initiate), and replicative forks propagate along the genome (elongation).
					It is common for some origins to be passively replicated; that is, they can be replicated by the replicative fork from a neighbouring origin before firing themselves.
					At the end of the S phase, two identical and complete sets of dsDNA will be present (bottom of image).}
	\end{figure}
	
	
	\section{Origins of Replication}
	\label{sec:Origins}
	
	When prokaryotic DNA is replicated, a single origin suffices for a competent (i.e., timely and accurate) replication program.
	In this case, the origin is located at a sequence-specific site, and the firing time need only be early enough for complete replication.
	However, eukaryotic DNA requires many origins of replication for a competent replication program.
	The number of origins in the genome varies by species, from fewer than 800 in budding yeast~\cite{OriDB} to about 100 000 in humans~\cite{OriginsReview}.
	When multiple origins exist on the genome, we can ask the following questions:
	\begin{itemize}
		\item \emph{What determines the locations of the origins?}\\*
		\item \emph{What controls the timing of origin firing?}
	\end{itemize}
	
	
		\subsection{Origin Locations}
		\label{subsec:OriginLocations}
		
		Depending on the organism, the factors determining origin locations vary considerably.
		In \emph{Saccharomyces cerevisiae} (budding yeast), origins are tightly bound in sequences between 11 and 17 base pairs (bp) in length and are effectively localized~\cite{ScottsPaper}.
		In \emph{Schizosaccharomyces pombe} (fission yeast), the origins are loosely associated with sequences between 100 and 200 bp long~\cite{OriginsReview}.
		The region of potential licensing grows to about 200 kilobase pairs (kb) in the human genome~\cite{HumanGenome}.
		In \emph{Xenopus laevis} (African clawed frog) embryos, the origins are placed stochastically, with no sequence affinity at all~\cite{FrogEmbryo}.
		
		
		\subsection{Origin Firing Times}
		\label{subsec:OriginTimes}
		
		In favourable environments, prokaryotic organisms exhibit exponential growth, and their replication program can be quite complex.
		During exponential growth, the cell cycles overlap, and more than one S phase can be active simultaneously~\cite{ExponentialGrowth}.
		However, this phenomenon is outside the scope of this thesis.
		
		In eukaryotes, the  need to initiate multiple origins leads to interesting timing dynamics.
		The origins do not all initiate simultaneously, with origin-initiation events occurring throughout S phase~\cite{DNAInitiation}.
		The mechanism that controls the relative timing of different origin initiation events is still a matter of some debate~\cite{ScottsPaper,Bechhoefer2012374,deMouraModel2,deMouraModel1} and is the topic of this thesis.
		
		
		\subsection{Budding Yeast}
		\label{subsec:BuddingYeast}
		
		In this thesis, we focus on \emph{S. cerevisiae} (budding yeast).
		Budding yeast is a useful model species because, unlike the other eukaryotic examples we discussed, the origins of \emph{S. cerevisiae} are localized.
		In each cell cycle, origins of budding yeast may be licensed only in very narrow regions on the genome.
		These regions are defined by specific sequences in the genome, called autonomously replicating sequences (ARS elements), that have been identified and cataloged\footnote{An online database can be found at http://cerevisiae.oridb.org/}~\cite{OriDB}.
		Thus, the advantage of choosing budding yeast as the model species is that the potential stochasticity in origin locations has conveniently been removed from consideration.
		
		Previous studies of the firing times of individual origins in budding yeast have found a correlation between the median firing times and the width of the firing time distributions~\cite{ScottsPaper,StochasticTermination}.
		Both studies measured the average firing time and the spread in firing times for each origin in budding yeast and discovered a correlation between them (see Fig.~\ref{fig:SigmoidalModel2}).
		Essentially, origins that tend to fire early have narrowly defined firing times, while those that tend to fire late have loosely defined firing times.
		This trend implies the existence of a mechanism that strongly controls the firing time of origins at the start of S phase but loses its potency as S phase progresses.
		
		One theory that explains this observation is the Multiple Initiator Model, which supposes that the number of MCM2-7 hexamers loaded on an origin will affect the timing width and median for that origin.
		The MIM will be discussed in detail below in Sec.~\ref{sec:MIM}.
		
		
	\section{Modelling Replication}
	\label{sec:Modelling}
	
	To recap, DNA replication begins at origins which, in budding yeast, are localized spatially but whose firing times exhibit stochasticity.
	Once an origin has fired, replication forks traverse the DNA bidirectionally, enclosing a growing region of replicated DNA between them.
	When two regions of replicated DNA meet, they coalesce into a single, larger region.
	
	This process can be mapped to a crystallization process in one dimension (Fig.~\ref{fig:CrystalVsReplication}):
	Crystallization starts when the crystal nucleates at nucleation sites, which map to origins of replication.
	From nucleation sites, the crystal grows bidirectionally, and the crystal domain is surrounded by boundaries that can be mapped to the replicative forks.
	Finally, when two crystal regions meet, they coalesce into a larger region, which matches the coalescence of neighbouring regions of replicated DNA.
	This mapping of DNA replication to crystal growth means that one can easily adapt well-developed stochastic models from crystal growth dynamics to describe DNA replication kinetics.
	
	\begin{figure}[tbh]
		\begin{center}
			\includegraphics[width=\textwidth]{Images/CrystalVsReplication.png}
		\end{center}
			\caption[Comparison Between 1-D Crystallization and DNA Replication]{\label{fig:CrystalVsReplication} Comparison between the one-dimensional KJMA crystallization model and DNA replication.
				At left is the one-dimensional crystallization process, with round markers representing nucleation sites, and arrows representing crystal boundaries (pointing in the direction of propagation).
				Thick lines represent crystal regions, and thin lines liquid regions.
				Right: DNA replication process, as described in Fig.~\ref{fig:EarlyReplicationConcept}.}
	\end{figure}
	
	Early in the 20\textsuperscript{th} century, Kolmogorov~\cite{Kolmogorov}, Johnson and Mehl~\cite{JohnsonAndMehl}, and Avrami~\cite{AvramiI,AvramiII,AvramiIII} independently developed a stochastic model to describe crystallization growth in three dimensions.
	Since its inception, the ``KJMA model'' has been used for many studies that range from phase transition kinetics~\cite{AlloyPhaseTransitions} to R{\'e}nyi's car-parking problem in one dimension~\cite{CarParking}.
	In 2002, J.~Herrick~\emph{et~al.} introduced a KJMA-like model of DNA replication to analyze experiments on \emph{X.~laevis} embryo extracts~\cite{KJMA2002}.
	In 2005, the KJMA-like model was expanded, and a formalism which can be used to infer the replication program of a genome given a set of parameters that describe the speed of replication forks and the origins' locations in time and space~\cite{KJMA1, KJMA2}.
	
	One of the quantities that can be inferred using the KJMA formalism is the replicated fraction, $f(x,t)$.
	The replicated fraction can be interpreted as the probability that the genome at position $x$ has been replicated by a time $t$ after the start of S phase.
	The replicated fraction is an important quantity that will be discussed more in Sec.~\ref{sec:ReplicatedFraction}.
	
	
	\section{Experiments Measuring Replication}
	\label{sec:ExperimentsBasics}
	
	The theory describing DNA replication has been driven by experimental results.
	In this section, we ask, \emph{How can DNA replication be observed experimentally?}
	There are several techniques to help answer this question, including flow cytometry~\cite{DeepSeq}, DNA combing~\cite{DNACombing}, microarray~\cite{MicroarrayReview, McCuneMicroArray}, and sequencing experiments~\cite{StochasticTermination,DeepSeq}.
	When designing an experimental procedure to measure DNA replication, there are two main considerations to take into account: the temporal scope and the scope of the desired measurement.
	
	There are two experimental approaches to study the time course of replication.
	The first is to synchronize the cell cycle.
	A common way to do this is by arresting the cell cycle~\cite{CellCycleSynch}.
	Arresting usually entails using a chemical bath to stop the cells from progressing from one phase to another (for experiments related to DNA replication, this is generally just prior to entering the S phase).
	After arresting the cell cycle, another chemical bath can be used to force the entire population of cells to enter the S phase synchronously.
	The main drawbacks to this approach are that  it is difficult to arrest many species of eukaryotic organisms and that, although arresting will stop a cell from moving through the cell cycle, it will not stop a cell from growing.
	Cells that have been arrested may have altered replication profiles as a result~\cite{CellCycleSynch}.
	The second approach is to forego arresting the cell cycle and pull samples from an asynchronous population.
	The main drawback to this method is that it either creates time-averaged data or it relies on multiple techniques to infer the timing information.
	For this research, the first strategy was investigated.
	
	Spatially, approaches range between two extremes: perfect resolution down to the scale of individual base pairs and no spatial information.
	At one extreme, some experiments infer quantities that are averaged over the entire genome. 
	For example, a common flow-cytometry technique called fluorescence-activated cell sorting (FACS)~\cite{DeepSeq, SequencingReview} measures only the total amount of DNA in the cell.
	Techniques such as FACS provide only limited information but are simple and fast.
	At the other extreme, techniques such as DNA sequencing and DNA microarrays can spatially resolve windows as small as 1 kb~\cite{DeepSeq}.
	Experiments with this level of resolution are quite complex but provide tremendous insight toward understanding DNA replication.
	For this research, highly resolved DNA sequencing data were investigated.
	
	Because of their impact on the research presented in this thesis, microarray and sequencing experiments will be discussed in more detail.
	Both experiments were designed to maximize spatial resolution, and both can use either a synchronous or an asynchronous population.
	
	
		\subsection{Microarray Experiments}
		\label{subsec:Microarray}
		
		Microarray experiments are high-throughput experiments that count the genome of entire populations of cells simultaneously~\cite{MicroarrayReview}.
		They start with a microarray chip\footnote{actually, many chips} and a population of cells.
		The DNA of the population is extracted and hybridized with the chip, and the number of hybridizations gives a measure of the replicated fraction, $f$.
		Depending on the temporal scope of the experiment, the measured replicated fraction can be time-averaged, $f(x)$, or in the case of an arrested population, it can be the replicated fraction for a specific time, $t_i$ after the start of S phase, $f(x,t=t_i)$.
		
		Because this technique measures entire populations, microarray experiments do not provide information about cell-to-cell variability.
		More importantly, microarrays suffer from artifacts that can be challenging to overcome.
		For example, in 2008, McCune~\emph{et~al.} measured replication fractions that spanned only 80\% of the possible values~\cite{McCuneMicroArray}.
		
		
		\subsection{Sequencing Experiments}
		\label{subsec:Sequencing}
		
		A sequencing experiment determines the precise sequence of base pairs contained in the input segment of DNA~\cite{SequencingReview}.
		To measure the replicated fraction using sequencing, one must start with with the fully mapped genome of the organism in question.
		The DNA from a population of cells to be measured is harvested and broken into segments about 50 bp long~\cite{StochasticTermination}.
		Each segment is sequenced and matched to the previously mapped genome.
		The normalized histogram of reads over the genome then provides the replicated fraction:
		Regions that have not replicated are measured once, and regions that have replicated are measured twice.
		
		Except for the actual process of measuring how many of each segment of DNA is present in the sample, sequencing experiments and microarray experiments are very similar.
		The process of arresting cells, or not, is the same for both, as is the broad analysis of the output.
		However, sequencing experiments do not require any clever data-processing to remove artifacts such as those present in microarray experiments.
		Recent advances lowering the cost of sequencing have seen a transition from microarray experiments to sequencing experiments for measuring DNA replication~\cite{EndOfMicroarray}.
		
		
	\section{Reading This Thesis}
	\label{sec:Map}

	Here, we give a brief outline of the thesis.
	
	\textbf{Chapter 2 - Motivation}
	Here we summarize the mathematical details of the KJMA-like model that describes DNA replication, and how to use it to calculate the replicated fraction from a theoretical model.
	We then discuss the development of the Sigmoidal Model the correlation between origin firing time width and median that it measured.
	This correlation motivated the creation of the Multiple Initiator Mode (MIM)l, which we describe.
	Finally, we introduce the recently experimental work done in N.~Rhind's lab that measured loaded MCM.
	These measurements provided some surprising results that call into question part of the scenario assumed in the MIM.
	With this chapter we hope to motivate our work in measuring the effect of the experimental results from the Rhind Lab on the MIM.
	
	\textbf{Chapter 3 - Method}
	Here we discuss the MIM simulator program which was used in our analysis of the Multiple Initiator Model.
	We start by outlining the simulation process in detail and motivate the assumptions and choices we made.
	Our discussion of the MIM simulator transitions into an analysis of noise; we want our simulation to produce data similar to experiment, including the level of noise.
	We estimate the noise in current cutting-edge sequencing data, and compare that to the noise in our simulation.
	Because the noise in experiment is different from the noise in our simulated data, we motivate the introduction of two methods of increasing the noise in our simulations:
	One method based on experimental limitations, and the other method involved adding Gaussian noise.
	
	\textbf{Chapter 4 - Results}
	Here we outline our three investigations in full.
	We started with preliminary single-origin measurements comparing the analytical MIM to simulated data as a quick investigation.
	After these measurements implied a deeper investigation would prove fruitful, we redesigned our methodology and performed more single-origin measurements.
	Finally, we moved onto a \textbf{chromosome or genome} wide investigation, in which we measured how the fit from MIM changes when it is force to have high or low $n$.
	With our final investigation, we show that \textbf{Finish this statement!}
	
	\textbf{Chapter 5 - Conclusions}
	Here we discuss the implications of our results and suggest new directions for this research.
		
		
		
		
		
		
		
		
		

\chapter{Motivation}
\label{ch:Motivation}

Previous work on quantitative modelling of DNA replication has investigated the timing of origin initiation~\cite{ScottsPaper,deMouraModel1,StochasticTermination,Goldar2009,OriginTimingReview}.
In 2010, Yang~\emph{et~al.} developed the ``Sigmoidal Model,'' which uses three parameters per origin (position, median firing time, and spread in firing time) to describe the replication program of budding yeast~\cite{ScottsPaper}.
After fitting these parameters to microarray data, the authors observed a correlation between the median firing time and spread in firing time.
This result (discussed in detail in Sec.~\ref{sec:SigmoidalModel}) was confirmed by Hawkins~\emph{et~al.} when they used a similar model to analyze sequencing data in 2013~\cite{StochasticTermination}.

The work of Yang~\emph{et~al.} led to the development of a second analytical model, ``the Multiple Initiator Model'' (MIM)~\cite{ScottsPaper}, which we present in Sec.~\ref{sec:MIM}.
The MIM proposes a biological hypothesis that explains the observed correlation between median firing time and spread in firing time.
The benefit of the MIM over the Sigmoidal Model is that the MIM uses only two parameters per origin to define the replicated fraction, effectively removing one third of the parameters from the model.

However, recent work performed in N. Rhind's lab~\cite{Rhind} has shown that one part of the scenario assumed in the MIM may not be biologically realistic (Sec.~\ref{sec:ExperimentsMIM}).
The purpose of this thesis is to explore the impact of these new experimental data on the MIM and show that \textbf{conclusion}.

This chapter will expand on the above story.


	\section{Replicated Fraction}
	\label{sec:ReplicatedFraction}
	
	In Secs.~\ref{sec:Modelling} and~\ref{sec:ExperimentsBasics}, we saw that the replicated fraction, $f$, can be calculated from both theoretical models and experiments.
	The replicated fraction as a function of time and space, $f(x,t)$, can be interpreted two ways:
	as describing either a single cell, or a population of cells.
	In the single-cell case, $f(x,t)$ is interpreted as the probability that the sequence at position $x$ in the genome has replicated by a time $t$ after the start of S phase.
	For a population of cells, $f(x,t)$ represents the fraction of cells in the population that have replicated at position $x$ by a time $t$ after the start of S phase.
	Although the two interpretations of $f$ might seem equivalent, we will see in Sec.~\ref{sec:MIM} that they are subtly different.
	Both definitions lead to a function that has values ranging from zero (no replication has occurred), to one (replication has certainly occurred).
	
	
		\subsection{Qualities of the Replicated Fraction}
		\label{subsec:QualitiesReplicatedFraction}
		
		Before we describe in detail the KJMA-like model of DNA replication, we will build some valuable intuition.
		In DNA sequencing experiments, the replicated fraction is measured spatially in windows about 1 kb wide and temporally in steps of 5 minutes~\cite{StochasticTermination}.
		Since the budding yeast genome has about 800 origins of replication, and is about $12\time10^3$ kb long, origins are, on average, spaced every 15 kb~\cite{OriDB,BuddingYeastSequence}.
		Thus, a spatial resolution of 1kb is narrow enough to uniquely identify and observe many origins individually. 
		However, there are generally no more than ten time points measured experimentally~\cite{StochasticTermination,DeepSeq,McCuneMicroArray}.
		(Indeed, the data analyzed in Sec.~[\textbf{Section Reference}] has only six.)
		Fortunately, this amount of temporal data is enough to infer the important features of the replication program.
		
		Figure~\ref{fig:ReplicatedFractionExample} shows an example set of replicated fraction data.
		The data comes from measurements done on Chromosome IV of budding yeast by Hawkins~\emph{et~al.}~\cite{StochasticTermination}.
		The reader may notice a few features:
		there are gaps in the spatial data;
		the replicated fraction ranges lower than zero and higher than one;
		and some regions of the genome replicate faster than others.
		
		\begin{figure}[tbh]
			\begin{center}
				\includegraphics[width=\textwidth]{Images/CHR4Exp.png}
			\end{center}
				\caption[Budding yeast Chromosome IV replicated fraction]{\label{fig:ReplicatedFractionExample} Example graph of replicated fraction.
					Data from chromosome IV of budding yeast as measured by Hawkins~\emph{et~al.} (\cite{StochasticTermination} supplementary data).
					x-axis represents the spatial organization of the genome as if it had been stretched out straight.
					y-axis is the replicated fraction.
					Six time points.
				}
		\end{figure}
		
		The gaps in Fig.~\ref{fig:ReplicatedFractionExample} exist because of a limitation of the sequencing experiment used to gather this data.
		Sequencing experiments match short sequences of DNA to the fully mapped genome (Sec.~\ref{subsec:Sequencing}).
		The budding yeast genome contains repeated patterns: sequences longer than 50 bp that appear more than once~[\textbf{FIND SOURCE}].
		When a sequence of DNA extracted from one of these patterns is measured, it is not counted because it cannot be uniquely located.
		
		The replicated fraction in Fig.~\ref{fig:ReplicatedFractionExample} has a range that goes below zero and above one.
		This surprising feature results from two assumptions: first, that the measured sequences were evenly distributed spatially; and, second, that all cells have the same average replicated fraction at the time measured, $f(t=t_i)$.
		In~\cite{StochasticTermination}, Hawkins~\emph{et~al.} extracted and measured about $10^7$ sequences.
		However, there is no guarantee that these sequences are evenly distributed, therefore there will be regions of the genome that are sequenced more and regions that are sequenced less.
		Additionally, Hawkins~\emph{et~al.} normalized the measured replicated fraction by setting the average replicated fraction, $f(t=t_i)$, equal to the replicated fraction measured using FACS on the bulk sample.
		This normalization assumes that the measured cells have the same average replicated fraction as the population measured with FACS.
		The error in regions that have been over-extracted or under-extracted can be exaggerated by normalizing incorrectly, leading to values of $f$ greater than one or less than zero.
		
		The most important observation is that some regions of the genome start replicating much earlier than others.
		This can be seen in the peaks in Fig.~\ref{fig:ReplicatedFractionExample}, for example at $x \approx 910$.
		Because replication starts at an origin and propagates outward, peaks in the replicated fraction imply early replication and, hence, the presence of origins.
		Additionally, early origins should create stronger peaks, and late origins should create weaker peaks.
		
		
		\subsection{Calculating Replicated Fraction from the KJMA Formalism}
		\label{subsec:KJMA}
		
		The replication program is defined by the origins through their spatial and temporal organization.
		The speed at which the replicative forks propagate also plays a role in determining the replicative program.
		Based on the work of Jun~\emph{et~al.}~\cite{KJMA1}, here we outline calculation of the replicated fraction from a set of parameters describing the origins of replication and the replicative forks.
		
		We define the rate of initiation, $I(x,t)$, to be the number of origins initiated per time per genome length at an unreplicated position $x$, and time $t$ after the start of S phase.
		Of course, initiation can happen only at origins of replication.
		In budding yeast, origins are localized at known locations~\cite{OriDB}, labeled $x_i$.
		Therefore, we define the rate of initiation at origin $i$ to be $I_i(x,t)=\delta(x-x_i)I_i(t)$, where $\delta(x)$ is the Dirac $\delta$ function.
		Finally, we define the rate of initiation to be $I(x,t) = \sum\limits_i I_i(x,t)$.
		
		\begin{figure}[tbh]
			\begin{center}
				\includegraphics[width=\textwidth]{Images/KJMA.png}
			\end{center}
				\caption[Replicated fraction from KJMA]{\label{fig:KJMA} KJMA approach to calculating $f(x,t)$.
					In order for the point at $(x,t)$ to not have been replicated, there cannot be any initiation events within the shaded triangle.
				}
		\end{figure}
		
		Given the function $I(x,t)$, we can infer the replicated fraction, $f(x,t)$, at a position $x$ a time, $t$, after the start of S phase:
		\begin{equation} \label{eq:RepFromI}
			f\left( x,t\right) = 1 - \prod_\Delta\left[1-I\left( x^\prime,t^\prime\right)\Delta x^\prime\Delta t^\prime\right] \text{ ,}
		\end{equation}
		where the product is over intervals $\Delta x^\prime \Delta t^\prime$ lying within the ``past triangle'' shown in Fig.~\ref{fig:KJMA}.
		In words, Eq.~\ref{eq:RepFromI} says that the probability that the genome at position $x$ has been replicated is one minus the probability that no origin has fired long enough in the past to have a replication fork pass over position $x$.
		In the limit $\Delta x\rightarrow0$ and $\Delta t\rightarrow0$, Eq.~\ref{eq:RepFromI} becomes
		\begin{equation} \label{eq:RepFromIexp}
			f\left( x,t\right) = 1 - \exp\left[-\iint\limits_\Delta dx^\prime dt^\prime I\left( x^\prime,t^\prime\right)\right] \text{ .}
		\end{equation}
		
		Now, it is possible to define a new quantity, $g(\Delta x_p,t)$, that is a local measure of origin firing:
		\begin{equation} \label{eq:LocalOriginFiring}
			g\left(\Delta x_p,t\right) = \int\limits_{x_p}^{x_{p+1}} dx \delta\left( x-x_i\right) \int\limits_o^t dt^\prime I_i\left( t^\prime\right)
		\end{equation}
		over the region $[x_p, x_{p+1})$ of a genome of length, $L$, discretized into $M$ segments;
		\begin{align}
			\Delta x = \frac{L}{M} \qquad\qquad x_p = p\left(\Delta x\right) \qquad\qquad p = 0, 1, 2, \ldots , M-1 \text{ .}
		\end{align}
		$g(\Delta x_p,t)=0$ if there are no origins enclosed in $\Delta x_p$ because initiation will only occur at an origin.
		Thus, we replace the double integral in Eq.~\ref{eq:RepFromIexp} by the function $g(\Delta x_p,t)$ and arrive at
		\begin{equation} \label{eq:RepFromG}
			f\left( x,t\right) = 1 - \exp\left[ - \sum\limits_{p=0}^{M-1}g\left(\Delta x_p,t-\frac{\left| x-x_p \right|}{v}\right)\right] \text{ ,}
		\end{equation}
		where $v$ is the speed of replication forks, $\Delta x_p$ is the $p^\text{th}$ interval, $x_p$ is the $p^\text{th}$ position, and $\left| x-x_p \right|/v$ is the time at the edge of the past triangle in Fig.~\ref{fig:KJMA}.
		
		Recognizing that $g(\Delta x_p,t)$ represents the initiation rate of budding yeast, we can constrain it to better describe the biological system:
		First, we constrain $g$ such that replication cannot happen before the start of S phase; $g(\Delta x_p,t<0)=0$.
		Second, we constrain the initiation rate to be non-negative. Because of the definition in Eq.~\ref{eq:LocalOriginFiring}, this constrains $g$ as well: $\frac{d}{dt}g(\Delta x_p,t)\geq 0$.
		Thus, as a consequence of the first two constraints, $g(\Delta x_p,t)\geq 0$.
		
		Finally, we derive the cumulative initiation probability, $\Phi(x_p,t)$, from $g(\Delta x_p,t)$ using a calculation similar to that used for a Poisson-process~\cite{Spikes}:
		\begin{equation} \label{eq:PhiFromG}
			\Phi\left( x_p,t\right) = 1 - e^{-g\left(\Delta x_p,t\right)} \text{ .}
		\end{equation}
		The cumulative initiation distribution is an important quantity that will be revisited below, in {Sec.~\ref{sec:MIM}.
		Note that $\Phi(x_p,t)$ is a general function that can be defined throughout the genome, but in the case of budding yeast is nonzero only for values of $x_p$ that coincide with origins.


	\section{The Sigmoidal Model}
	\label{sec:SigmoidalModel}
	
	The sigmoidal model is a phenomenological approach to characterizing each origin.
	Developed by S. Yang as part of his PhD thesis, this model assumes that the functional form of $I_i(t)$ is a sigmoidal function that has a range from zero to one and that is defined by three parameters for each origin, $i$, on the genome~\cite{ScottsPaper,ScottsThesis}.
	
	Figure~\ref{fig:SigmoidalModel} shows the preliminary observations that motivated the sigmoidal model.
	First, the replicated fractin at an origin, $f(x=x_i,t)$, was extracted from experimental data of the entire genome, $f(x,t)$ (Fig.~\ref{fig:SigmoidalModel}A).
	The figure shows the analysis of microarray data~\cite{McCuneMicroArray}.
	Second, a sigmoidal curve was fit to $f(x=x_i,t)$ (Fig.~\ref{fig:SigmoidalModel}B).
	This sigmoidal curve is parameterized by the median replication time, $t_{\text{rep}}$, and by the spread of replication times, $t_{\text{width}}$:
	\begin{equation} \label{eq:SigmoidalModel}
		f(t) = {\frac{1}{1+\left({\frac{t_{\text{rep}}}{t}}\right)^r}}\text{ ,}
	\end{equation}
	where $t_{\text{width}}$ is defined by
	\begin{equation} \label{eq:SigmoidalModel2}
		t_{\text{width}} = \left(3^{1/r}-3^{-1/r}\right)t_{\text{rep}}\text{ .}
	\end{equation}
	
	\begin{figure}[tbh]
		\begin{center}
			\includegraphics[width=.82\textwidth]{Images/ScottFig3-1.pdf}
		\end{center}
			\caption[Early analysis of budding yeast]{\label{fig:SigmoidalModel} Schematic of the initial analysis of budding yeast data that led to the creation of the Sigmoidal Model.
				\textbf{A} Sample replicated fraction: smoothed data from microarray measurements of Chromosome I (solid lines)~\cite{McCuneMicroArray}.
				The black triangles indicate the locations of previously identified origins~\cite{OriginLocations}.
				Data from the replicated fraction at a single origin (grey region) at a time were analyzed.
				\textbf{B} Equation~\ref{eq:SigmoidalModel} fitted to the extracted replicated fraction at an origin.
				The fit function has parameters, $t_{\text{rep}}$ and $t_{\text{width}}$, which are shown.
				\textbf{C} Scatter plot of origin parameters from fitting the replicated fraction at every origin reveals a correlation.
				The dashed line shows $t_{\text{width}}=t_{\text{rep}}$.
				Figure reproduced with permission from S.~Yang~\cite{ScottsThesis}}
	\end{figure}
	
	One can see in Fig.~\ref{fig:SigmoidalModel}C the correlation between $t_\text{rep}$ and $t_\text{width}$.
	However, this approach ignores interactions between neighbouring origins and their effects on $f(x=x_i,t)$; these parameters may not describe intrinsic properties of the origins.
	Thus, the correlation may not be a property of the origins themselves but an accident of their geometry.
	
	To better analyze the data, Yang developed the Sigmoidal Model: an analytical method for quickly calculating the replicated fraction over the whole genome, $f(x,t)$.
	The model calculates $f(x,t)$ from a set of origins defined by three parameters ($x_i$, $t_i^{(1/2)}$, and $t^{(\text{w})}_i$), where $t_{1/2}$ and $t_\text{w}$ are defined by Eqs.~\ref{eq:SigmoidalModel} and~\ref{eq:SigmoidalModel2}~\cite{ScottsThesis}.
	With this, the entire set of experimental data was used to characterize every origin simultaneously (not illustrated).
	
	\begin{SCfigure}[1][tbh]
		\includegraphics[width=.47\textwidth]{Images/ScottTFig3-4.pdf}
		\caption[Sigmoidal model]{\label{fig:SigmoidalModel2} Scatter plot of origin parameters from fitting the replicated fraction of the entire genome with the Sigmoidal Model reveals a correlation.
			Solid points are specific origins identified for discussion in Yang's thesis.
			Figure reproduced with permission from S.~Yang~\cite{ScottsThesis}}
	\end{SCfigure}
	
	Figure~\ref{fig:SigmoidalModel2} graphs the intrinsic $t_{1/2}$ vs $t_{\text{w}}$ calculated from fitting the Sigmoidal Model to microarray data.
	Notice the strong correlation between timing width and median observed in the crude analysis of Fig.~\ref{fig:SigmoidalModel}C is present in the intrinsic parameters as well.
	This correlation means that early origins have narrowly defined firing times, while late origins have loosely defined firing times.
	An implication is that there is a mechanism that controls origin firing time that is strong at the start of S phase but weakens as S phase progresses~\cite{ScottsThesis}.
	This observation suggested the Multiple Initiator Model (MIM).
	
	
	
	\section{The Multiple Initiator Model}
	\label{sec:MIM}
	
	The MIM, in its simplest form, assumes that each origin has a given number of potential initiators that may be initiated during S phase~\cite{ScottsThesis}.
	If each of these potential initiators have equally opportunity to fire, then origins with large numbers of initiators should tend to fire earlier than origins with few initiators.
	Effectively, origins with more initiators loaded will tend to fire earlier in S phase than origins with fewer pairs.
	However, it is important to note that other factors, such as chromatin structure (the three-dimensional organization of the genome), can affect the relative firing times of origins~\cite{Chromatin}.
	
		\subsection{MIM Basics}
		\label{subsec:MIMBasics}
		
		One hypothesis for the biological mechanism that makes up a potential initiator is the MCM2-7 hexamer pairs loaded at each origin.
		During licensing, the ORC can load MCM2-7 hexamers in excess~\cite{MultiMCM}.
		A simple hypothesis (that will be discussed in Sec.~\ref{subsec:PrepModule}) is that initiators are loaded as a Poisson process:
		MCM2-7 hexamers are loaded at an origin individually with some probability determined by the affinity of that origin.
		We assume that the average number of initiators loaded at the $i^{\text{th}}$ origin, $n_i$, is a fixed quantity over different cell cycles.
		However, the actual number of initiators, $N_i$, can be different each cell cycle and is here assumed to be Poisson distributed.
		
		During S phase, the initiators are activated by the addition of Cdc45 and the GINS complex.
		The MIM assumes that each initiator has the same cumulative probability of firing as time progresses through S phase, given by
		\begin{equation}\label{eq:CPDInitiator}
			\Phi_0(t) = \frac{1}{1+\left(\frac{t^*_{1/2}}{t}\right)^{r^*}}\text{ ,}
		\end{equation}
		where $t^*_{1/2}$ is the median firing-time for a single initiator and where $r^*$ sets the width of the distribution.
		These variables are global, defining the behaviour of every initiator on the genome.
		From this assumption, the cumulative probability that an origin with $N$ loaded initiators has fired is
		\begin{equation} \label{eq:CPDEffectiveN}
			\Phi_{\text{eff}}(t,N) = 1 - \left[1 - \Phi_0(t)\right]^N\text{ .}
		\end{equation}
		
		From Eq.~\ref{eq:CPDEffectiveN}, the replicated fraction can be inferred from a set of global parameters (fork velocity, time, $t^*_{1/2}$, $r^*$, and two parameters defining the noise in the experimental data) and two parameters per origin (its position on the genome, and the number of initiators it loads).
		We start by calculating the effective cumulative firing time distribution, $\Phi_{i}^{(\text{eff})}(x,t,N_i)$ for each origin, $i$.
		Next, we invert Eq.~\ref{eq:PhiFromG},
		\begin{equation}
			\ln \left( 1- \Phi_{i}^{(\text{eff})}(x,t,N_i)\right) = - g(\Delta x_p, t) \text{ ,}
		\end{equation}
		and sum over every origin in the genome,
		\begin{equation} \label{eq:fitting}
			\sum\limits_{\text{all origins }i}\Phi_{i}^{(\text{eff})}(x,t,N_i) = - \sum\limits_{p=0}^{M-1} g(\Delta x_p,t) \text{ ,}
		\end{equation}
		to calculate the initiation rate for the entire genome.
		Finally, we can replace the exponent in Eq.~\ref{eq:RepFromG} with the left-hand side of Eq.~\ref{eq:fitting}.
		Using this process, Yang fit the parameters listed above to microarray data as part of his PhD research~\cite{ScottsThesis}.
	
		Equation~\ref{eq:CPDEffectiveN} calculates the effective cumulative probability distribution of an origin with $N$ loaded initiators.
		This is a property of a single cell, with a single value for $N$.
		In Sec.~\ref{sec:ReplicatedFraction} we mentioned that there is a subtle difference between the single-cell interpretation and the population interpretation of the replicated fraction.
		If the number of loaded initiators at an origin does not change between cell cycles (i.e. $N=n$), then the two interpretations are equivalent and Eq.~\ref{eq:CPDEffectiveN} applies to large cell populations as effectively as a single cell.
		However, if $N$ varies among cell cycles, the two interpretations diverge.
		Equation~\ref{eq:CPDEffectiveN} then becomes:
		\begin{equation} \label{eq:CPDEffectiven}
			\Phi_{\text{eff}}(t,n) = 1 - \langle\left[1 - \Phi_0(t)\right]^n\rangle\text{ ,}
		\end{equation}
		where $\langle \cdots \rangle$ denotes the ensemble average over $P_N(n)$.
		
		
		\subsection{Accounting for variability in $N$}
		\label{subsec:VariableN}
		
		Given the many factors that affect the ability of the ORC to load MCM2-7 initiators onto DNA~\cite{MultiMCM}, it is reasonable to assume that the number of initiators will vary over cell cycles.
		Thus, we need a way to evaluate Eq.~\ref{eq:CPDEffectiven} to calculate $\Phi_\text{eff}(t,n)$.
		
		We assume that initiators are loaded as a Poisson process.
		This means in a large population of cells, if a particular origin has a mean number of initiators, $n$, the standard deviation of $N$ will be $\sqrt{n}$~\cite{cowan}.
		Therefore, as $n$ grows, the relative fluctuations within the population shrinks as $n^{-1/2}$.
		Thus, for large-enough $n$, we can neglect fluctuations in $n$ and Eq.~\ref{eq:CPDEffectiveN} becomes accurate.
		However, recent experimental evidence suggests that typically, $n$ ranges between one and five, which is not large.
		
	\section{Experimental Measurement of the Number of Initiators}
	\label{sec:ExperimentsMIM}
	
	The MIM makes a strong hypothesis about the physical mechanism controlling origin firing times during S phase.
	In particular, its predictions about the relative number of MCM pairs at a given origin can be checked experimentally.
	Here, we describe recent experiments performed by Das~\emph{et~al.}, that constitute a first attempt to estimate relative and absolute numbers of loaded MCM pairs in budding yeast~\cite{Rhind}.
	
	Das~\emph{et~al.} made several measurements to test the strength of the MIM.
	First, they measured the relative number of initiators loaded at each origin; according to the MIM, origins that fire earlier should have more initiators than those that fire later.
	The relative number of initiators indicates which origins have more initiators, but cannot measure exactly how many initiators there are (i.e. it can say that origin $a$ has twice as many initiators as origin $b$ but cannot differentiate between $n_a=2$, $n_b=1$ and $n_a=200$, $n_b=100$).
	Second, Das~\emph{et~al.} measured the effect of reducing the number of initiators at an origin; according to the MIM, reducing the number of loaded initiators should delay the mean firing time of that origin.
	Third, they measured the average absolute number of initiators loaded on a particular early origin; according to the MIM, an early-firing origin should have many initiators loaded, on average.
		
		\begin{figure}[tbh!]
			\begin{center}
				\includegraphics[width=.8\textwidth]{Images/DasFig1.png}
			\end{center}
				\caption[Relative amounts of loaded MCM]{\label{fig:Das1} Results from ChIP-seq experiments measuring the relative number of loaded MCMs throughout the budding yeast genome.
					\textbf{A} ChIP-seq measurements for Chromosome X from G1 arrested wild-type cells.
					Red histogram shows uniquely located reads, grey is multiply-located reads.
					Circles along x-axis show the locations of identified origins.
					\textbf{B} Scatter plot of ChIP-seq data at origins vs their $n$ values calculated from the MIM.
					Blue origins are believed to be affected by chromatin structure~\cite{Chromatin}; green are not.
					Some origins are labeled with double circles.
					These labels refer to other parts of the data presented by Das~\emph{et al.}, but not presented here.
					ARS1 origin is labeled with text.
					Line represents the best linear fit to green dots ($r=0.56$).
					Figure reproduced with permission from N.~Rhind~\cite{Rhind}}
		\end{figure}
	
		\subsection{Relative Number of Initiators}
		\label{subsec:RelativeNo}
		
		To measure the relative number of initiators, Das~\emph{et~al.} used ChIP-seq in a population of G1-arrested cells.
		ChIP-seq (Chromatin immunoprecipitation followed by sequencing) is a technique that profiles genome-wide DNA-binding proteins, histone modifications or nucleosomes~\cite{ChIP-seq}.
		Das~\emph{et al.} used ChIP-seq to measure the number of MCMs bound to budding yeast DNA.
		In this case, Das~\emph{et~al.} prepared the experiment such that the output provided a measure of the relative number of MCM proteins loaded throughout the genome.
		Figure~\ref{fig:Das1}A shows the relative number of MCM proteins in Chromosome X of budding yeast.
		The peaks align with origins identified in OriDB~\cite{OriDB}.
		Figure~\ref{fig:Das1}B shows the relative number of MCM2-7 hexamers loaded at an origin vs.\ the $n$ value predicted from the MIM in 2010~\cite{ScottsPaper}.
		After ignoring origins that are believed to fire late due to their location in chromatin structure (blue origins in Fig.~\ref{fig:Das1}B)~\cite{Chromatin}, Das~\emph{et~al.} observed a correlation between the number of initiators and the theoretical parameter $n$ calculated by the MIM.
		This measurement confirms the first prediction made by the MIM, that the number of initiators loaded correlates with origin firing times.
		
		\begin{figure}[tbh]
			\begin{center}
				\includegraphics[width=.8\textwidth]{Images/DasFig2.png}
			\end{center}
				\caption[Replication profile of various origins]{\label{fig:Das2} Replication profiles of various origins.
					Many of the origins are outside the scope of this discussion.
					ARS1 and ARS1$\Delta$B2 are shown as the thick black and thick grey curves respectively.
					The removal of the B2 element of ARS1 causes that origin to fire, on average, 13 minutes later in S phase.
					Note that, on the y-axis, $f = [\text{Relative copy number}] - 1$.
					Figure reproduced with permission from N.~Rhind~\cite{Rhind}}
		\end{figure}
		
		
		\subsection{Suppressing the Loading of Initiators}
		\label{subsec:SuppressingInitiators}
		
		To evaluate the effect of suppressing the loading of initiators on the replication program, Das~\emph{et~al.} measured the replication profile of ARS1 and the ARS1-$\Delta$B2 mutant.
		The ARS1 origin is known to be early firing~\cite{OriDB} and should therefore have a relatively high number of initiators.
		Because the B2 element of ARS1 takes part in the recruitment of Mcm2p, the ARS1-$\Delta$B2 mutant (which has the B2 element removed) reduces MCM2-7 loading~\cite{ARS1Mutant}.
		The expectation, based on the MIM, is that the mutant ARS1-$\Delta$B2 will have a later mean firing time because of the reduced number of initiators loaded.
		By measuring the replicated fraction of cells with both wild-type and mutant ARS1 origins independently, Das~\emph{et~al.} saw a marked (13 minute) delay in the average replication timing of ARS1 caused by the $\Delta$B2 mutation (Fig.~\ref{fig:Das2}).
		
		\begin{SCfigure}[1][tbh]
			\includegraphics[width=.47\textwidth]{Images/DasFig3.png}
			\caption[Absolute number of loaded MCMs at ARS1]{\label{fig:Das3} Quantization of data from a western blot experiment that measured the amounts of MCM, ORC and Zif268 present in populations of G1 arrested plasmids and G2 arrested plasmids.
				The left-most column shows that, on average, there are about 3 initiators loaded at ARS1 during the G1 phase.
				Figure reproduced with permission from N.~Rhind~\cite{Rhind}}
		\end{SCfigure}
		
		
		\subsection{Number of Initiators Loaded}
		\label{subsec:NoInitiatorsLoaded}
		
		Das~\emph{et~al.} engineered several special plasmids and used them to measure the average number of initiators loaded at an early firing origin.
		A plasmid is a small loop of dsDNA that is separate from the genome and that is replicated independently~[\textbf{Find Source}].
		On each plasmid was engineered one of a selection of origins (plasmids were separated into populations such that each plasmid in a given population contained the same origin).
		One of the six proteins in the ORC and one of the six proteins in the MCM2-7 hexamer were tagged, so that their relative average numbers could be measured with western blotting.
		In order to calculate the absolute average numbers of MCM2-7 hexamers and ORCs, Das~\emph{et~al.} normalized the measurements using a zinc-finger.
		The normalization was possible because Zif268, the so-called ``zinc-finger,'' binds to a specific 10-bp sequence of DNA with sub-nanomolar affinity~\cite{ZincFingers}.
		By including a single instance of the Zif268 binding sequence in the plasmid, Das~\emph{et~al.} concluded that each plasmid had exactly one Zif268 protein bound to it.
		From a population of G1 phase arrested cells, the engineered plasmids were extracted and the relative number of ORCs, MCM2-7s and zinc-fingers were measured and normalized such that the average number of zinc-fingers was one.
		
		The results of Das~\emph{et~al.} for the ARS1 origin are particularly instructive.
		Figure~\ref{fig:Das3} shows the average number of MCM2-7 hexamers (one initiator is a pair of these) loaded on ARS1, and the average number of loaded ORCs during G1 arrest and G2 arrest.
		The conclusion is that there are $n\approx 3$ MCM pairs on the ARS1 origin during G1 arrest.
		It is important to restate that the MIM does not report absolute $n$ but rather gives relative $n$.
		In other words, the value for $n$ reported by the MIM  is proportional to the number of unique chances an origin has to fire.
		Thus, the fact that the number measured by Das~\emph{et~al.} does not match that predicted by the MIM is not immediately troublesome.
		However, the small number of initiators, means that cell-to-cell variability can be important.
		The question we set out to answer with this research is, thus, \emph{how does cell-to-cell variability in the initiation factor affect the replication program?}












































\chapter{Methods}
\label{ch:Methods}

In this chapter we outline our investigation of the impact of variability in initiation factor on the MIM.
The backbone of our investigation was the development and implementation of a Monte Carlo program that simulates DNA replication.
A great deal of consideration went into the details of the simulation program to ensure it was efficiently producing meaningful results:
We ensured the randomly generated numbers were distributed properly.
We utilized optimized algorithms to track the replicated fraction and used multiple programming languages to increase performance.
We calculated how to adequately produce measurements commensurate with current experimental standards.
We ran simple tests that show that the program produces expected results in well understood regimes.
After verifying the efficacy of the simulator, we investigated the effect of small $n$ on the MIM.
Finally, we concluded that variability in $N$ does not greatly impact the predictive power of the MIM.

Except where special note is made, all computations were performed in IGOR Pro Version 6.3.6.4


	\section{The MIM simulator}
	\label{sec:MIMSimulator}
	
	The MIM simulator takes as inputs an identical set of parameters to those defined by the MIM.
	Namely, there are four global inputs (The elapsed time since the start of S phase $t_\text{sim}$, the speed of replicative forks $v$, the median firing time $t_{1/2}$, and $r$, which defines the width of the cumulative firing time distribution) and two local parameters per origin (the position $x_i$ and the average number of initiators $n_i$).
	The reader may notice that this set of parameters is not identical to those outlined in Sec.~\ref{sec:MIM}, that is because the experimental error was handle slightly differently, as will be described in Sec.~\ref{}.
	The simulator uses these parameters to generate the replicated fraction, $f(x,t=t_\text{sim})$, over the entire genome.
	To increase efficiency, the simulation is able to do this over several sets of parameters that are nearly identical (with $t_\text{sim}$ changing by steps of 5 minutes), thereby creating data comparable to those measured with sequencing experiments.
	
	The MIM simulator can be broken into three modules: the preparation module, that sets the randomly distributed parameters; the phantom nuclei module, that uses those parameters to calculate $f(x,t=t_\text{sim})$; and, the housing module, that tracks progress and executes the commands.
	These three modules will be discussed in more detail below.
	
	
		\subsection{The preparation module}
		\label{subsec:PrepModule}
		
		The preparation module is, fundamentally, a Monte Carlo program.
		A Monte Carlo program can be identified by its use of pseudo-random numbers~[\textbf{find source}] (pseudo-random because computers are unable to generate purely random numbers~[\textbf{find source}].
		In the case of the preparation module of the MIM simulator, there is are two sets of random numbers needed:
		First, the program requires a set of absolute numbers of initiators, ${N_i}$, for all origins $i$.
		Second, the program requires a set of firing times, $t_i,j$, for all initiators $j$ at each origin $i$.
		For both sets, care was taken to ensure that the generated values were properly distributed to match MIM theory (Sec.~\ref{sec:MIM}).
		To connect this with the DNA replication program, the preparation module is analogous to the initiation process undertaken in the G1 phase of the cell cycle (Sec.~\ref{subsec:G1Phase}.
		
		
			\subsubsection{Randomly Generated $N_i$}
			
			The first task the preparation module undertakes is randomly generating $N_i$, the absolute number of initiators at origin $i$ for the cell being simulated.
			Thus, the first choice we made in creating the simulator was how the values for $N_i$ should be distributed, given their average $n_i$.
			In Sec.~\ref{subsec:MIMBasics}, we mentioned that a simple hypothesis is that initiators are loaded onto an origin as a Poisson-process; if this is the case, the number of initiators should be Poisson distributed.
			This hypothesis is nearly as simple as possible (setting $N$ the same for every cell cycle is simpler, but that is the assumption we are testing from previous work), but is not based on observations.
			We are unaware of any experiments that have measured the distribution in the number of initiators between cell cycles.	
			In discussion with collaborators, another hypothetical distributions were considered:
			It seems reasonable that it is very easy to load the first initiator at an origin, so perhaps all origins are guaranteed to have one initiator with additional initiators are loaded as a Poisson-process.
			However, without any experimental evidence to motivate the selection of a complex model, the simple Poison model was used.
			Therefore, the preparation module selects the number of initiators at origin $i$ from a Poisson distribution defined by the average $n_i$.
			The preparation module makes this selection using a built-in IGOR function that will create Poisson-distributed random numbers given the desired average.
			
			
			\subsubsection{Randomly Generated Firing Times}
			
			The second task the preparation module undertakes is assigning a firing time to each initiator on the genome.
			This is different from assigning a firing time to each origin: If there are $k$ origins, then the number of initiators is given by $\sum\nolimits_{i=0}^k N_i = K$.
			Therefore, $K$ randomly generated firing times are required.
			In this case, the MIM dictates the desired firing time distribution of an initiator, which we derive from the cumulative firing time probability shown in Eq.~\ref{CPDInitiator} (Shown again below).
			\begin{equation} \label{CPDInitiator2}
				\Phi_0(t) = \frac{1}{1+\left(\frac{t^*_{1/2}}{t}\right)^{r^*}}\text{ .}
			\end{equation}
			By recognizing that $\Phi_0$ goes from zero (when $t$ is zero), to one (when $t$ is infinite), we can use inverse transform sampling~\textbf{cite} to randomly generate firing times that reproduce the desired cumulative firing time probability.
			Essentially, if we generate $u$, a uniformly distributed number between zero and one, for which IGOR has a built-in function, we can transform that to be distributed as given by Eq.~\ref{CPDInitiator2} with
			\begin{equation}
				F(u) = \frac{t^*_{1/2}}{\left(\frac{1}{u}-1\right)^\frac{1}{r^*}} \text{ ,}
			\end{equation}
			where $F(u)$ is the firing time.
			To test the inverse transformation, a histogram of $10^5$ samples was satisfactorily fit to the cumulative fire-time distribution.
			After all $K$ firing times are generated, the time of the first-to-fire initiator at each origin is kept because each origin can only fire once, thus the firing time of the origin $i$ is given by the firing time of the earliest initiator $t_i$.
			
			
		\subsection{The phantom nuclei module}
		\label{subsec:PhanNuc}
		
		Based on work done by S.~Jun~\emph{et~al.}, the phantom nuclei algorithm we used in the simulation is a powerful tool for calculating replicative data from a set of data describing origins of replication in the KJMA-like formalism~\cite{KJMA1}.
		There are three major components to the phantom nuclei module: the pre-processor, the coalescer, and the compiler.
		With these three components, the phantom nuclei quickly calculates the regions on the genome which have been replicated given the input parameters (the four global parameters, and the positions and firing-times of each initiator).
		These components have been separated for the sake of clearly describing the basic process, however there is some overlap between them in our code to increase performance.
		Figure \textbf{Make/Take this} illustrates the key features of the phantom nuclei method
		
		
			\subsubsection{Pre-processor}
			
			The strength of the phantom nuclei algorithm is in that it pre-processes the origin data it receives in order to remove redundant information and reduce the amount of computations needed to fully simulate the replication process.
			As we mentioned above, we designed the simulator to loop through many values of $t_\text{sim}$.
			The pre-processor calculates the state of replication at the highest value for $t_\text{sim}$ given.
			We start at the last time step because that is when every meaningful event will have occurred: origins have fired, or not, and any coalescence events have taken place.
			
			For each origin, $i$, the pre-processor calculates the position $x_i^\text{(L)}$ of the left fork and and $x_i^\text{(R)}$ of the right fork that propagate from the position $x_i$ at a speed $v$ starting at the time $t_i$.
			Calculating these positions is done with simple kinematics:
			\begin{equation} \label{eq:findforks}
				x_i^{\binom{\text{R}}{\text{L}}} = x_i \pm v \times \left( t_\text{sim} - t_i\right) \text{ ,}
			\end{equation}
			where the right fork is given by the sum, and the left fork by the difference, and where the bracketed term calculates the time since the origin fired.
			As a part of this step, any origins for which $t_i > t_\text{sim}$ are immediately removed from the simulation, as they will not contribute any fork data.
			Once the fork locations for each origin have been calculated, the pre-processor analyzes each pair of neighbouring origins to determine if any origin was passively replicated.
			Any passively replicated origins, or ``phantom nuclei'' are removed from the simulation (black dots in Fig. \textbf{make/take this}).
			The pre-processor is finished after it removes all but the necessary origins from the simulation (empty circles in Fig. \textbf{make/take this}), thereby speeding up the calculation of the replicated fraction.
			
			
			\subsubsection{Coalescer}
			
			The coalescer is the largest part of the phantom nuclei algorithm, and is used at every time-step in the simulation.
			The coalescer makes three major calculations:
			First, it selects which origins will fire by comparing them to the current value of $t_\text{sim}$; only origins with $t_i < t_\text{sim}$ will fire.
			Second, using Eq.~\ref{eq:findforks}, the coalescer creates two sets of fork positions (forks traveling left and forks traveling right) from the origins selected in the first step.
			Third, it analyzes the two sets of forks, and identifies where forks have collided and coalescence has occurred.
			Any forks that have collided are removed from the sets of forks, and the resulting set of forks make up the boundaries between replicated an unreplicated regions
			
			
			\subsubsection{Compiler}
			
			The compiler is used immediately after the coalescer, and is therefore used at every time-step in the simulation.
			The compiler simply loops through the two sets of forks, and sets the replicated fraction for the cell to one between forks, and zero otherwise.
			Although this process may sound straightforward, we were unable to do it without nests loops that significantly slowed the simulation process.
			For this reason, this part was written both in IGOR and in C++, and when we wanted to simulate large data sets early in our work, we called the C++ function as an external program.
			
		
		\subsection{The housing module}
		
		When the compiler finished, so too did the simulator.
		However, in Sec.~\ref{sec:ReplicatedFraction} we showed that the replicated fraction spans all values between zero and one, and in the description above, the simulator only creates replicated a replicated fraction with values of either zero or one.
		The simulation is of a single cell, and knows the replication profile for the single cell exactly, so the replicated fraction can only be zero (not replicated) or one (replicated).
		To see an example of $f(x,t)$ that can be compared to experimental data, we must loop over many cells and calculate the average replicated fraction.
		This is, essentially, exactly what sequencing experiments do: taking large amounts of data from a population and averaging them to find the average behaviour of the population.
		The housing module is designed to call the other two modules a prescribed number of times with the same set of parameters and thereby calculate an average.
		
		In addition to this, the housing module can analyze and alter the simulated replicated fraction $f_\text{sim}$.
		In our research we fit the MIM parameters to $f_\text{sim}$, we added Gaussian noise to reflect current experimental limitations, and we calculated the difference between $f_\text{sim}$ and experimentally measured $f$.













































\chapter{Results}
\label{ch:Results}

\textbf{Make an introductory paragraph(s) to outline the procedures and the results shown here.}

	\section{Single-Origin Investigations}
	\label{sec:SingleOrigin}
	
	Our early work consisted of single-origin simulations to motivate further study.
	Early in 2014, the results discussed in Sec.~\ref{sec:ExperimentsMIM} from N.~Rhind's lab were sent to us~\cite{Rhind}.
	As mentioned, these results called into question the assumption that $n$ is large enough to ignore variations in $N$.
	Our first goal, therefore, was to discover whether simulations of small $n$ agreed with the MIM predictions for small $n$ or not.
	In this investigation, discussed further below, we developed ``the difference parameter,'' a metric to measure the difference between the simulated and predicted $f(n)$ of a single origin with average $n$ initiators.
	The investigation showed the difference parameter decreased as $n^{-1}$ and motivated the deeper research presented in this thesis.
	
	The single-origin investigations that followed our preliminary work consisted of simulating $f(n)$ and fitting the MIM parameters to the result.
	With these investigations we refined the simulation program to run more efficiently, and to create data similar to that measured in sequencing experiments.
		
	\begin{figure}[tbh!]
		\begin{center}
			\includegraphics[width=\textwidth]{Images/DifferenceParameterGraph.pdf}
		\end{center}
			\caption[Difference Parameter]{\label{fig:DifferenceParameter} Schematic of the difference parameter calculations.
				Coloured lines represent the difference parameter for different values of $n$.
				Triangles show the parameter values in the corresponding inset graphs.
				\textbf{A} Replication fraction simulation and theory curves for $n=10$; $DP=0.037$.
				\textbf{B} Replication fraction simulation and theory curves for $n=5$; $DP = 0.062$.
				\textbf{C} Replication fraction simulation and theory curves for $n=1$; $DP = 0.264$.
				Also illustrated in \textbf{C} is the value $\Delta P(x)$ at the peak.
				Note that $\Delta P(x)$ is defined over the entire domain.
				}
	\end{figure}
	
	
		\subsection{The Difference Parameter}
		\label{subsec:earlywork}
		
		When this project started, we performed a quick investigation into the difference between $f_\text{sim}(n)$ and $f_\text{MIM}(n)$ for a single origin.
		To generate $f_\text{sim}(n)$, the MIM simulator calculated the average replicated fraction for about $10^6$ cells, producing data with very little statistical error.
		The global parameters used for the simulation were set equal to those measured previously by fitting the MIM to microarray measurements of budding yeast~\cite{ScottsPaper}.
		To generate $f_\text{MIM}(n)$, the fitting function built into IGOR was used, where the parameters were fixed and set equal to those used in the simulation.
		The difference parameter was then defined as
		\begin{equation} \label{DifferenceParameter}
			DP = \max_{x} {\frac {\Delta P(x)} {P}} \text{ ,}
		\end{equation}
		where $\Delta P(x)$ is the difference between $f_\text{sim}(n)$ and $f_\text{MIM}(n)$ (see Fig.~\ref{fig:DifferenceParameter}), and $P$ is the peak value of $f_\text{MIM}(n)$.
		
		Figure~\ref{fig:DifferenceParameter} shows the analysis process of the difference parameter for a single origin.
		First, $f_\text{sim}(n)$ an $f_\text{MIM}(n)$ were calculated over several time steps and $n$ ranging from one to 128, example replicated fractions are shown in the insets of Fig.~\ref{fig:DifferenceParameter}.
		From these data, we calculated $DP(n,t)$, the main image in Fig.~\ref{fig:DifferenceParameter}.
		Note the value for $DP$ ``saturates'' as $t$ increases, we call this value the ``saturated difference parameter.''
		Figure~\ref{fig:SaturatedDifferenceParameter}\footnote{
		The reader may notice a change in the $n$ values displayed in the graphics. Figure~\ref{fig:DifferenceParameter} is illustrative and contains old data; accurate, but not useful.
		After producing that graph we used a slightly different set of parameters in the simulation and changed the range of $n$ simulated when producing Fig.~\ref{fig:SaturatedDifferenceParameter}}
		shows the saturated difference parameter as a function of $n$.
		This initial investigation showed the saturated difference parameter decreases as $n^{-1}$.
		
		\begin{figure}[tbh]
			\begin{center}
				\includegraphics[width=0.8\textwidth]{Images/SaturatedDifferenceParameterGraph.pdf}
			\end{center}
				\caption[Saturated Difference Parameter]{\label{fig:SaturatedDifferenceParameter} Saturated difference parameter vs. $n$.
				The large uncertainty in low $n$ arises because the time-steps not going far enough to accurately measure the saturated difference parameter.
				The line is proportional to $n^{-1}$ ($\text{[Saturated Difference Parameter]} \approx \frac{0.3}{n}$).
				}
		\end{figure}
		
		From our initial investigation, we concluded that the difference parameter grows quickly with decreasing $n$.
		Therefore, we suspect that the MIM will not produce accurate results in the case that $n$ is small.
		These results motivated further research into the effect of small $n$ on the MIM.
		
		
		\subsection{Biased Fits}
		\label{subsec:BiasedFits}
		
		The definition of the difference parameter does not scale to more than one origin.
		The saturated difference parameter is only measurable when the theory curve has a peak value near one, therefore near-neighbour origins will interfere with each other.
		Additionally, the difference parameter is defined as a single value for the whole simulated genome, therefore we cannot infer anything about more than one origin.
		Thus, we developed a second investigation that measures how the parameters fitted with the MIM to a simulation of small $n$ may be biased.
		
		The process of this investigation was to calculate $f_\text{sim}(n)$ and to follow that by fitting the parameters of the MIM to the result.
		Thus, we have two parameters, $n_\text{sim}$ and $n_\text{fit}$, where $n_\text{sim}$ is the value for $n$ input to the simulation, and $n_\text{MIM}$ is the value for $n$ that results from the MIM fit.
		Initially, we performed these measurements using simulations of large populations of cells (populations of $10^6$ cells and larger).
		However, we discovered that when we performed high-accuracy simulations the fits are not strong.
		Therefore, we changed the simulation as described in Sec.~\ref{sec:Noise} to produce noisy data to represent the current capabilities of sequencing experiments.
		
		\begin{SCfigure}[1][tbh]
			\includegraphics[width=0.47\textwidth]{Images/LargePopBias.pdf}
			\caption[Bias in MIM fit on Large-Population Simulations]{\label{fig:LargePopulation} Scatter plot of $n_\text{sim}$ vs. $n_\text{fit}$ for simulations of a large population.
			Red circles show the data, dashed line shows unity.
			There is a large bias for low $n_\text{sim}$ which decreases as $n_\text{sim}$ grows.
			}
		\end{SCfigure}
		
		Figure~\ref{fig:LargePopulation} shows the preliminary results from our biased fit investigation on large-population simulations.
		The graph is a scatter plot of $n_\text{sim}$ vs. $n_\text{MIM}$, the dashed line shows unity.
		As we expected, the biased is relatively large for low $n_\text{sim}$, but decreases as $n_\text{sim}$ grows.
		Before this line of analysis went any deeper, however, we found that, at this population size, the simulated replicated fraction and the analytical MIM were not compatible.
		When performing the fits, the $\chi^2$ metric indicated that the fits generated were not acceptable.
		Therefore, although the results match our expectation, we had to reconsider our approach to make a strong statement about the MIM.
		
		\begin{figure}[tbh]
			\begin{center}
				\includegraphics[width=0.8\textwidth]{Images/NoisyBias.pdf}
			\end{center}
				\caption[Bias in MIM fit on Noisy Data]{\label{fig:NoisyBias} Scatter plot of $n_\text{sim}$ vs. $n_\text{fit}$ for simulations of noisy data.
				Red dots show data from 50 simulations at each value $n_\text{sim}$.
				Blue circles with error bars show the mean and standard deviation of the mean.
				Dashed line shows unity.
				The bias for low $n_\text{sim}$ decreases with increasing $n_\text{sim}$.
				\textbf{Inset} Scatter plot of the normalized difference ($[n_\text{sim} - n_\text{fit}]/n_\text{sim}$) vs. $n_\text{sim}$.
				$\text{[Dashed line]} = 0.55/n_\text{sim}$.
				$\text{[Dottend line]} = 0.32/\sqrt{n_\text{sim}}$.
				}
		\end{figure}
		
		After discovering that simulating over a large population to produce highly accurate replicated fraction data was not working, we decided to limit our simulations to the accuracy of current experiments.
		In Sec.~\ref{Sec:Noise} we outlined the process used to generate noisy data.
		Figure~\ref{fig:NoisyBias} shows the result of our analysis of noisy simulated data.
		Again, we used the MIM Simulator to generate $f_\text{sim}(n_\text{sim})$ and using a built-in IGOR Pro function, we fit the parameters of the MIM to that data.
		Figure~\ref{fig:NoisyBias} is a scatter plot of $n_{sim}$ vs. the resulting $n_\text{fit}$.
		In this case, because of the increased noise in the simulated data, we performed this procedure fifty times per $n_\text{sim}$ value (red dots).
		The blue data show the mean and standard deviation of the mean for each value of $n_\text{sim}$.
		Here, again, we see that the bias is largest for small $n_\text{sim}$, and decreases $n_\text{sim}$ grows.
		In the inset to Fig.~\ref{fig:NoisyBias} we show the relative difference normalized by $n_\text{sim}$ (i.e. $[n_\text{sim} - n_\text{fit}]/n_\text{sim}$).
		The dashed line shows $0.55/n_\text{sim}$, while the dotted line shows $0.32/\sqrt{n_\text{sim}}$.
		The uncertainty in the data is too great to claim any particular trend, except to say that it is behaving qualitatively as expected.
		
		The data shown in Fig.~\ref{fig:NoisyBias}, which comes from fitting the MIM to artificially noisy simulated data, shows the behaviour we expect.
		Additionally, the fits being produced have normalized $\chi^2$ values of $1\pm 0.1$.
		Therefore, we conclude that the MIM Simulator is producing good data, that is comparable to both experimental data and the analytical MIM.
		
		
	\section{Simulations of Chromosome I}
	\label{sec:ChromosomeI}
	
	Our results from single-origin simulations seem to indicate that the MIM does not perform well in the small-$n$ regime, but multiple origins may work together to reduce that effect.
	Following our single-origin investigation, we simulated the replicated fraction of Chromosome I of budding yeast.
	For this investigation, we used the same simulation method as described for the noisy single-origin analysis, except that the genome size and origin parameters were chosen to represent Chromosome I.
	The origin parameters were set by fitting the MIM to the replicated fraction for wild-type budding yeast reported by Hawkins~\emph{et al.}~\cite{StochasticTermination}.
	To test the effect of small $n$, the fitted parameter $t_{1/2}$ was fixed at a high value (90 minutes) to produce high values for $n$, and it was fixed at a low value (40 minutes) to produce low values for $n$.
	The resulting values for $n$ at each origin can be seen in table \textbf{make table}.
	
	We simulated
	









































\chapter{Conclusions}
\label{ch:Conclusions}

In this thesis, we used the MIM simulator to explore the effect of variation in initiation factors on the efficacy of the analytical MIM.
We presented several investigations ranging in complexity from a single origin to Chromosome I of budding yeast.
From our investigations we concluded that the inferences made with the MIM remain accurate in the case that the number of initiators is lower than first assumed.

Our research was motivated by a recent experiment measuring low numbers of loaded initiators which contradicts the assumption made in the analytical MIM.
Naive interpretations of the MIM suggest that the MIM should fail when the number of initiators is low.
We started our investigations by analyzing simple single-origin genomes.
The results of these single-origin studies confirmed our suspicions:
Inferences made with the MIM become less accurate as the number of initiators decreases.
In contrast, simulations of Chromosome I of budding yeast showed that these inferences are accurate when the number of initiators is low.
To understand this contradiction, we continued our research by simulating sequences with multiple origins.
From these simulations we observed two trends that explain the transition from inaccuracy for single origins to accuracy for several origins.
The first trend is that the overall accuracy of MIM inferences increases with the number of origins.
The second trend is that the accuracy is greater for origins in the middle of a cluster of origins than origins at the edges (i.e., origins with two neighbours are handled better than origins with only one).
These two trends contribute to our conclusion that inferences made with the MIM are equally accurate for any number of initiators.

Our conclusion is that inferences made with the MIM maintain their accuracy for small numbers of initiators when there are many origins.
We have provided a qualitative analysis of multiple-origin simulations that shows that as the number of origins increases, so too does the accuracy.
The positive outcome of this research is that we now have increased confidence in the MIM approach to analyzing DNA replication data from experiments.

	\section{Future Considerations}
	
		\subsection{Quantitative Analysis}
		
		It is apparent that the next step in this research is to perform a quantitative analysis of how the contribution of multiple origins act to mitigate the inaccuracy in inferences made by the MIM due to fluctuations in initiation factors.
		Such research could explore two features of our results:
		first, a quantification of the accuracy as a function of number of origins and number of initiators;
		second, an exploration of how origin spacing and relative initiation factors between origins in a sequence affect inferences made by the MIM.
		A successful investigation along these lines would help future researchers quantify the accuracy of their measurements made with the MIM for a given genome.
		
		
		\subsection{Analysis of Other Organisms}
		
		The research presented here was based on the model organism \emph{Saccharomyces cerevisiae}.
		This was a deliberate choice: because the origins of replication in \emph{S.~cerevisiae} are confined to known locations, the complexity of the replication process is significantly reduced.
		However, as we discussed in Ch.~\ref{ch:Introduction}, \emph{S.~cerevisiae} is a special case, and it is far more common for origins to be located diffusely in a region.
		Therefore, research expanding the MIM to address stochastically located origins would dramatically increase the number of organisms it can analyze.
		
		
		\subsection{Toward a Biological Research Tool}
		
		We believe that in the long-term  the culmination of this research will be the development of a tool for biological research.
		If the studies described above of multiple origins and stochastically located origins are successful, the MIM could form the basis of a research tool for DNA replication.
		This tool would be used by researchers performing studies of DNA replication to quickly make inferences about the number of initiators loaded on the DNA.
		
		
		\subsection{Comments About the Simulator}
		
		In our investigations we created a modular program that simulates DNA replication.
		This simulation makes use of the KJMA framework discussed in Sec.~\ref{subsec:KJMA}.
		As we mentioned in our discussion of the KJMA, it is a mathematical framework that has a broad range of applications.
		Thus, the MIM simulator can address problems described by the KJMA in one dimension with the addition or replacement of modules within the program.
		Therefore, there are many new and peripherally related avenues of research that can be investigated with use of our MIM simulator.
		



%  BACK MATTER  %%%%%%%%%%%%%%%%%%%%%%%%%%%%%%%%%%%%%%%%%%%%%%%%%%%%%%%%%
%
%  References and appendices. Appendices come after the bibliography and
%  should be in the order that they are referred to in the text.
%

\backmatter%
	\addtoToC{Bibliography}
	\bibliographystyle{bibstyle/Thesis-Bib}%unsrt} %% default style is {plain}, but it outputs the list alphabetically by first author, unsrt will order the list by citation order
	%% Change the name ``references'' to the name of your .bib file (in the same folder as this file).
	\bibliography{references}

%% To add additional appendicies, merely add \chapters
%\begin{appendices} % optional

%\chapter{The replication time of \emph{E. coli}}
\label{ap:EColi}

To estimate the time required for complete replication of \emph{E. coli} DNA some information is required.
First, the length of the genome is $L = 4600$ kb.
Second, the replication forks propagate at an average speed of $v = 1000 \text{ bp/s} = 1$ kb/s.
Third, the DNA is organized in a loop, so there is a single origin site and a single termination site exactly opposite. \cite{EColi}

With this information, it is a quick calculation to estimate the time it takes for a single fork to traverse the genome, $t_1$:
\begin{equation}
	t_1 = {{L}\over{v}} = {{4600 \text{ kb}}\over{1 \text{ kb/s}}} = 4600 \text{ s}.
\end{equation}
However, two forks are used to replicate \emph{E. coli} DNA, so the time required to fully replicate the genome, $t$, is:
\begin{equation}
	t = t_1/2 = 2300 \text{ s},
\end{equation}
which is just over 38 minutes.

Therefore, the time required for \emph{E. coli} to fully replicate its genome is about $40$ minutes.

%\end{appendices}

% You can choose to add an index to your thesis as well. To do so, make sure you \addtoToC{Index}. You'll also need to look up how to include items in the index.
\end{document}
