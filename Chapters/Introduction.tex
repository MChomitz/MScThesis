\chapter{Introduction}

Ubiquitous to all life is the timely and accurate replication of DNA.
In simple cells (called prokaryotes), the process used to replicate the genome (called the ``replication program'') is well understood.
Starting a sequence-defined location (the ``origin''), the double stranded DNA of the prokaryote genome separates into two single-stranded segments.
From the origins, replicates bidirectionally as the single-stranded segments grow and are used as templates for the creation of two copies of the original genome.
The boundaries between replicated and non-replicated DNA, called ``forks,'' continue to propogate throught the genome until it has been entirely separated and replicated.
Prokaryotic organisms have such small genomes that replication can be completed using a single origin.\cite{MolecularCellBiology}

More complex organisms (called eukaryotes), have genomes which are aproximately 1000 times longer and have forks that propogate about 10 times slower than prokaryotes.
Therefore, a single origin cannot fully replicate eukaryotic DNA quickly enough to sustain life.
To overcome this limitation, eukaryotic DNA is replicated using a parallel process, with many origins firing throughout the genome.

Using many origins in the replication program creates several non-obvious issues that must be addressed for the program to work effectively.
One such issue is the existance of separate replicated regions during the replication process.
When the forks of two neighbouring replicated regions meet, the two coalesce into a single, larger replicated region.
Another hurdle the program must overcome is the need to coordinate multiple origin activations events that are driven by stochastic processes.
In other words, the program must be able to control the replication program such that the stochastic events that drive replication do not cause errors that can potentially harm the organism.\cite{eukaryotereview}


	\section{The Cell Cycle}
	
	The cell cycle defines the steps taken during cellulare reproduction.
	There are many ways to define the differenst steps, or phases, taken during the cell cycle, depending on the depth and resolution needed.
	For the purpose of communicating this research, the cell cycle will be divided into four steps (see Figure \ref{CellCycle}::
	The first gap (G1) phase, the synthesis (S) phase, the second gap (G2) phase, and the mitosis (M) phase.
	The first two phases of the cell cycle (G1 and S) are described by key processes for DNA replication.
	The third phase of the cell cycle (G2) primarily acts as a buffer to ensure complete DNA replication.
	The forth phase of the cell cycle (M) encompasses the physical division of the cell into two daughter cells.
	The G2 and M phases are very important for robust cellular reproduction, but do not play a large role in DNA replication, and will therefore remain undiscussed.
	
	\begin{SCfigure}[1][thb]
		\includegraphics[width=0.45\textwidth]{images/CellCycle.png}
		\caption[Cell Cycle]{\label{CellCycle} The complete cell cycle is made up of four phases: 
			The Mitosis (M) phase, when a mother cell separates into two daughter cells. 
			The first Gap (G1) phase, when the daughter cell undergoes growth and chemical preperation for DNA replication.
			The Synthesis (S) phase, when the DNA is replicated.
			And the second Gap (G2) phase which acts as a buffer to ensure complete replication before the M phase.}
	\end{SCfigure}
	
		\subsection{The G1 Phase}
		
		As with any cycle, the cell cycle begins at the end of the previous cell cycle.
		The G1 phase begins early in the life of a daughter cell, after the mother cell has divided in the previous M phase.
		During this time, the cell grows and, more importantly, an important chemical process called licensing is carried out to prepare for replication during S phase.
		
		Licensing occurs at the origin recognition complex (ORC).
		The ORC is made up of a single group of six protiens that bind to the DNA at an origin.
		Two additional protiens (Cdc6 and Cdt1) assist the ORC in recruiting minichromosome maintenance (Mcm) 2-7 hexamer rings onto the DNA.
		Loaded hexamers form pairs oreiented away from each other, each such pair will later be referred to as potential initiators, or just initiators.
		After licensing, the resulting set of protiens associated with the origin is call the pre-replication complex (pre-RC).
		Licensing is surpressed during the S and G2 phases by cyclin-dependant kinases, this effectively limits the cell cycle to a single replication event.\cite{MolecularCellBiology}
		
		\subsection{The S Phase}
		
		The second phase in the cell cycle is the S phase.
		After lisencing is completed in the G1 phase, the S phase encompases the division of the genome into two identical copies.
		There are three main processes that happen during the S phase:
		Initiation, elongation, and coalescense.
		
		An origin initiatiates (or activates, or fires) during S phase when five other proteins bind to each in a pair of Mcm2-7 rings:
		Cdc45 and the tetrameric GINS complex.
		The total system of protiens is called the CMG complex (\emph{c}dc45, \emph{M}cm2-7, \emph{G}INS complex) and composes a helicase which traverses the genome during the S phase.
		As an origin fires, the pre-RC disassembles and the activated pair of helicases unwind and separate the double helix of the double-stranded DNA (dsDNA) into two complimentary single-stranded DNA (ssDNA) chains. \cite{GINSComplex}
		
		After an origin has been initiated, there is a small region of ssDNA bounded on either side by the CGM complex helicases.
		The points where the dsDNA is separated into two ssDNA chains are called replication forks (or just ``forks'').
		Elongation is the process by which the replication forks, with the help of the biological machinery stored in the CGM complex, propogate bydirectionally from the origin, separating the dsDNA.
		As the forks propogate, DNA polymerases bind with the ssDNA between them.
		DNA polymerases use the ssDNA as a template and backbone for synthesising dsDNA; essentially it adds the missing half back onto the separated strand.
		DNA polymerase can only propogate in the 3' direction.
		This poses a problem, because the two strands in dsDNA are orriented in opposing directions, so the polymerase can only smoothly traverse one of the ssDNA chains (the ``leading strand'') at each fork.
		On the other strand (the ``lagging strand''), the polymerase ``stutters:''
		It replicates a small region in the direction opposite to that of fork propogation until it hits a replicated region, leapfroggs over and past that region in the direction of fork propogation, and repeats.
		These small fragments are called Okizaki fragments.
		On the lagging strand, the Okizaki fragments are connected by DNA ligases.
		
		Finally, when two forks meet, coalescense occurs:
		The helicases dissasemble, and the two regions of dsDNA are connected by DNA ligase. \cite{MolecularCellBiology}
		
		For more information on the intricacies of this process, see \cite{OriginsReview}, and for full definitions of those protiens that take part in the replication process and how they interact, see \cite{PurifiedProteins}.
		
	Figure \ref{EarlyReplicationConcept} shows a distilled version of these processes. It effectively communicates the important parts of the cell cycle and removes many of the fine details that can be ignored.
	
	\begin{figure}[tbh]
		\begin{center}
			\includegraphics[width=\textwidth]{images/EarlyReplicationConcept.png}
		\end{center}
			\caption[Events During Replication]{\label{EarlyReplicationConcept} During replication many processes encompassing many protiens and protien structures are carried out.
				This graphic illustrates a digested version of the G1 and S phases of the cell cycle, with only those parts that are necessary to understand the model present.
				\textbf{A} In the G1 phase, origins are located and licensed when the ORC recruits pairs of MCM2-7 hexamers onto the dsDNA.
					During S phase, pairs of MCM2-7 hexamers are activated by the addition of the cdc45 protien.
					After activation, the resulting structures become the replicative forks which travers the DNA unwinding the dsDNA allowing the DNA to be replicated.
				\textbf{B} Due to the existance of multiple origins in eukaryotes, there are several other events that happen during the S phase.
					At the start of S phase, there are several origins liscensed along the genome (top of image).
					As time progresses (down), origins fire independantly and replicative forks propogate along the genome.
					It is common for some number of origins to be passively replicated, that is, they can be replicated by the replicative fork from a neighbouring origin before firing themselves.
					At the end of the S phase, two identical and complete sets of dsDNA will be present (bottom of image).}
	\end{figure}
	
	
	\section{Origins of Replication}
	
	When prokaryotic DNA is replicated, a single origin is sufficient for a competent replicative program.
	In this case the origin is located at a sequence specific site, and there is no need to worry about the firing-time except to ensure it is early enough for complete replication.
	However, eukaryotic DNA requires many origins of replication for a competent replicative program.
	The number of origins in the genome varies by species, from fewer than 800 in budding yeast\cite{OriDB}, to about 100 000 in humans \cite{OriginsReview}.
	When multiple origins exist on the genome, there are more questions one can ask about their behaviour:
	\begin{itemize}
		\item \emph{What determines the locations of the origins?}
		\item \emph{What controls the timing of when an origin fires?}
	\end{itemize}
	
		\subsection{Origin Locations}
		
		The question ``\emph{What determines the locations of the origins},'' can be difficult to answer.
		Depending on what organism your discussing, the answer can change significantly.
		In \emph{Saccharomyces civiseai} (budding yeast) origins are tightly bound in sequences between 11 and 17 bp in length and are effectively localized \cite{ScottsPaper}.
		In \emph{Schizosaccharomyces pombe} (fission yeast) the origins are loosely bounded in 100 to 200 bp sequences \cite{OriginsReview}.
		[\textbf{FIND EXAMPLES}].
		
		For the research presented, \emph{S. cerevisiae} (budding yeast) was considered.
		Budding yeast is an ideal model species because, unlike the other examples provided [\textbf{when I find some}], the origins of \emph{S. cerevisiae} are localized.
		In each cell cycle, origins of budding yeast may only be licensed in very narrow windows (aproximately 11 base pairs (bp)).
		These windows are defined by specific sequences in the genome, called \emph{a}utonomously \emph{r}eplicatin \emph{s}equences (ARS) elements, have been identified and catalogued\footnote{an online database can be found at http://cerevisiae.oridb.org/} \cite{OriDB}.
		Thus, by choosing budding yeast as the model species, the potential stochasticity in origin locations has conveniently been removed from consideration.
		
		\subsection{Origin Firing-Times}
		
		Origin initiation is a chemical process, and therefore is governed in part by the diffusive flow of chemicals within the cell.
		This means the initiation process must be stochastic at some level.
		Previous studies of the firing-times of individual origins in budding yeast have found evidence that the process has some driving force. [scott and de Moura]
		Both studies measured the average firing-time and the spread in firing-times for each origin in budding yeast and discovered a correlation between them.
		Essentially, origins that tend to fire early have narrowly defined fire-times, while those that tend to fire late have loosely defined firing times.
		This trend implies the existance of a mechanism that strongly controls the fire-time of origins at the start of S phase, but loses its potency as S phase progresses.
		
		A theory for what physical components of the replicative machinery create this timing mechanism will be discussed in detail later [\textbf{Section number or name?}].
		
		
	\section{Modelling Replication}
	
	To recap, DNA replication begins at origins which, in budding yeast, are localized spacially but whose firing-times exhibit stochasticity.
	Once an origin has fired, replication forks traverse the DNA bidirectionaly, enclosing a growing region of replicated DNA between them.
	When two regions of replicated DNA meet, they coalesce into one larger region.
	
	This process can very easily be mapped to a crystalization process in 1 dimension (Figure \ref{CrystalVsReplication}):
	Crystalization starts when the crystal nucleates at nucleation sites, which map to origins of replication.
	From nucleation sites, the crystal grows bidirectionally and the crystal domain is surrounded by boundaries that can be mapped to the replicative forks.
	Finally, when two crystal regions meet they coalesce into a larger region, which matches the coalescense of neighbouring regions of replicated DNA.
	This mapping of DNA replication to crystal growth is quite convenient.
	
	\begin{figure}[tbh]
		\begin{center}
			\includegraphics[width=\textwidth]{Images/CrystalVsReplication.png}
		\end{center}
			\caption[Comparing Crystalization with Replication]{\label{CrystalVsReplication} A comparison between the 1 dimensional KJMA crystalization model and the described replication model.
				On the left is the 1 dimentional crystalization process, with nucleations shown as single circles and the crystals propogating at circular boundaries.
				On the right is the replication process as described, see Figure \ref{EarlyReplicationConcept} for a full description.
				It is possible to take advantage of the striking similarity between the two processes.}
	\end{figure}
	
	Early in the 20th century the KJMA model was developed as a stochastic model that describes crystalization growth in three dimensions.
	Since its inception, the KJMA model has been used for many studies that range from phase transition kinetics \cite{AlloyPhaseTransitions} to Renyi's car-parking problem in one dimension \cite{CarParking}.
	With the wealth of work done in the KJMA, it shouldn't be surprising that in 2005 S.~Jun, H.~Zhang, and J.~Bechhoefer successfully synthesised a KJM-like model of DNA replication \cite{KJMA1,KJMA2}.
	In the work of 2005, a formalism was developed which can be used to infer the replication profile of a genome given a set of parameters that describe the speed of replication forks and the origins' locations in time and space.
	
	One of the quantities that can be infered using the KJMA formalism is the replicated fraction, $f(x,t)$.
	The replicated fraction can be interpretted as the probability that the genome at position $x$ has been replicated by the time $t$ after the start of S phase.
	The replicated fraction is an important quantity that will be discussed more later.
	
	
	\section{Experiments Measuring Replication}
	
	So far, the mechanisms of DNA replication and a quick intoduction to modelling that process have been discussed.
	The question still remains:
	\emph{How can DNA replication be observed experimentally?}
	As it turns out, there are several techniques available to scientists that answer this question, including DNA combing, chip Seq, microarray, and sequencing experiments [\textbf{FIND SOURCE(S)}].
	Because they play an important role in this research, Microarray and sequencing experiments will be discussed in more detail here.
	
		\subsection{Microarray Experiments}
		
		\emph{[\textbf{READ ABOUT THIS}] Where did it originate? By what mechanism does it work? What set it apart when it was developed? What are the current pros and cons?}
		
		\subsection{Sequencing Experiments}
		
		\emph{Where did it originate? By What mechanism does it work? What set it apart when it was developed? What are the current pros and cons?}
		
		A sequencing experiment is any experiment that is able to identify the sequence of basepairs in a DNA molecule.
		
		
		
		
		
		
		
		
		