\chapter{Methods}
\label{ch:Methods}

In this chapter we outline our approach to investigating the question of how variability in initiation factor impacts the MIM.
The backbone of our investigation was the development and implementation of a Monte Carlo program that simulates DNA replication.
A great deal of consideration went into the details of the simulation program to ensure it was efficiently producing meaningful results:
We ensured the randomly generated numbers were distributed properly.
We utilized optimized algorithms to track the replicated fraction and used multiple programming languages to increase performance.
We calculated how to adequately produce measurements commensurate with current experimental standards.
We ran simple tests that show that the program produces expected results in well understood regimes.
After verifying the efficacy of the simulator, we set out to investigate the effect of small $n$ on the MIM.


	\section{MIM simulator}
	
	The MIM simulator takes, as inputs, an identical set of parameters to those defined by the MIM.
	Namely, there are four global inputs (The elapsed time since the start of S phase $t_\text{sim}$, the length of the genome $L$, the median firing time $t_{1/2}$, and the rate of increase of the cumulative probability distribution of firing times $r$) and two local parameters per origin (the position $x_i$ and the average number of unique chances to fire\footnote{For the remainder of this chapter, ``unique chances to fire'' will be referred to as ``initiators.''} $n_i$).
	The simulator uses these parameters to generate the replicated fraction, $f(x,t=t_\text{sim})$, over the entire genome at time $t_\text{sim}$.
	To increase efficiency, the simulation is able to do this over several sets of parameters that are nearly identical (with $t_\text{sim}$ changing by steps of 5 minutes), thereby creating sets of data comparable to those measured with sequencing experiments.
	
	The MIM simulator can be broken into three modules: the preparation module, that sets the randomly distributed parameters; the phantom nuclei module, that uses those parameters to calculate $f(x,t=t_\text{sim})$; and, the housing module, that tracks progress and executes the commands.
	The housing module was made for convenience only, and does not add any scientifically meaningful features to the program.
	However, both the preparation module and the phantom nuclei module will be discussed in more detail below.
	
	
		\subsection{The preparation module}
		
		The preparation module is, fundamentally, a Monte Carlo program.
		A Monte Carlo program can be identified by its use of pseudo-random numbers~[\textbf{find source}] (pseudo-random because computers are unable to generate purely-random numbers~[\textbf{find source}].
		In the case of the preparation module of the MIM simulator, there is are two sets of random numbers needed:
		First, the program requires a set of absolute numbers initiators, ${N_i}$, for all origins $i$.
		Second, the program requires a set of firing times, $t_i,j$, for each initiator $j$ at origin $i$.
		For both sets, care was taken to ensure that the generated values were properly distributed to match MIM theory (Sec.~\ref{sec:MIM}).
		
		In Sec.~\