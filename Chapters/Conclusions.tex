\chapter{Conclusions}
\label{ch:Conclusions}

In this thesis, we used the MIM simulator to explore the affect of variation in initiation factors on the efficacy of the analytical MIM.
We presented several investigations ranging in complexity from a single origin to Chromosome I of budding yeast.
From our investigations we concluded that the inferences made with the MIM remain accurate in the case that the number of initiators is lower than first assumed.

Our research was motivated by a recent experiment measuring low numbers of loaded initiators which contradicts the assumption made in the analytical MIM.
Naive interpretations of the MIM suggest that the MIM should fail when the number of initiators is low.
We started our investigations by analyzing simple single-origin genomes.
The results of these single-origin studies confirmed our suspicions:
Inferences made with the MIM become less accurate as the number of initiators decreases.
In contrast, simulations of Chromosome I of budding yeast showed that these inferences are accurate when the number of initiators is low.
To understand this contradiction, we continued our research by simulating sequences with multiple origins.
From these simulations we observed two trends that explain the transition from inaccuracy for single origins to accuracy for several origins.
The first trend is that the overall accuracy of MIM inferences increases with the number of origins.
The second trend is that the accuracy is greater for origins in the middle of a cluster of origins than origins at the edges (i.e., origins with two neighbours are handled better than origins with only one).
These two trends contribute to our conclusion that inferences made with the MIM are equally accurate for any number of initiators.

Our conclusion is that inferences made with the MIM maintain their accuracy for small number of initiators when there are many origins.
We have provided a qualitative analysis of multiple-origin simulations that shows that as the number of origins increases, so too does the accuracy.
The positive outcome of this research is that we now have increased confidence in the MIM approach to analyzing DNA replication data from experiments.

	\section{Future Considerations}
	
		\subsection{Quantitative Analysis}
		
		It is apparent that the next step in this research is to perform a quantitative analysis of how the contribution of multiple origins act to mitigate the inaccuracy in inferences made by the MIM due to fluctuations in initiation factors.
		Such research could explore several features of our results:
		first, a quantification of the accuracy as a function of number of origins and number of initiators;
		second, an exploration of how origin spacing and relative initiation factors between origins in a sequence affect inferences made by the MIM.
		A successful investigation along these lines would help future researchers quantify the accuracy of their measurements made with the MIM for a given genome.
		
		
		\subsection{Analysis of Other Organisms}
		
		The research presented here was based on the model organism \emph{Saccharomyces cerevisiae}.
		This was a deliberate choice: because the origins of replication in \emph{S.~cerevisiae} are localized to known locations, the complexity of the replication process is significantly reduced.
		However, as we discussed in Ch.~\ref{ch:Introduction}, \emph{S.~cerevisiae} is a special case, and it is far more common for origins to be located diffusely in a region.
		Therefore, research expanding the MIM to address stochastically located origins would dramatically increase the number of organisms it can analyze.
		
		
		\subsection{Toward a Biological Research Tool}
		
		We believe that in the long-term  the culmination of this research will be the development of a tool for biological research.
		If the studies described above of multiple origins and stochastically located origins are successful, the MIM could form the basis of a research tool for DNA replication.
		This tool would be used by researchers performing studies of DNA replication to quickly make inferences about the number of initiators loaded on the DNA.
		
		
		\subsection{Comments About the Simulator}
		
		In our investigations we created a modular program that simulates DNA replication.
		This simulation makes use of the KJMA framework discussed in Sec.~\ref{subsec:KJMA}.
		As we mentioned in our discussion of the KJMA, it is a mathematical framework that has a broad range of applications.
		Thus, the MIM simulator can address problems described by the KJMA in one dimension with the addition or replacement of modules within the program.
		Therefore, there are many new and peripherally related avenues of research that can be investigated with use of our MIM simulator.
		