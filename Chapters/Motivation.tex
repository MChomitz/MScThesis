\chapter{Motivation}
\label{ch:Motivation}

Previous research into DNA replication has investigated the timing of origin initiation in great detail \cite{ScottsPaper,StochasticTermination}.
In \cite{ScottsPaper}, Yang \emph{et al.} fit the replication fraction of an origin, $f(x=x_o,t)$, to a Sigmoidal Model to better understand origin firing-time [\textbf{Section reference}].
The results (discussed in detail below) were confirmed using a different model by Hawkins \emph{et al.} \cite{StochasticTermination} in 2013.

The work of Yang \emph{et al.} lead to the development of a second analytical model called the Multiple Initiator Model (MIM) [\textbf{Section reference}].
The MIM made some reasonable assumptions to simplify the math involved.
This simple model can then be used to analyze experimental data and draw conclusions about the physical mechanisms controlling DNA replication.

However, recent work performed in N. Rhind's lab [\textbf{Source this, somehow}] have shown that one of the assumptions used in the MIM may not be representative of the truth. [\textbf{Section reference}]
The purpose of this research is to discover the impact of this new information on the MIM.

The following sections of this chapter will expand on this story, filling in the details of the math behind the models, and explicitly stating the assumptions and measurements made.


	\section{The Sigmoidal Model}
	\label{sec:SigmoidalModel}
	
	