\chapter{Results}
\label{Ch:Results}

\textbf{Make an introductory paragraph(s) to outline the procedures and the results shown here.}

	\section{Single-Origin Investigations}
	\label{sec:SingleOrigin}
	
	Our early work consisted of single-origin simulations to motivate further study.
	Early in 2014, the results discussed in Sec.~\ref{sec:Experiments} from N.~Rhind's lab were sent to us~\cite{Rhind}.
	As mentioned, these results called into question the assumption that $n$ is large enough to ignore variations in $N$.
	Our first goal, therefore, was to show that simulations with small $n$ did not agree with the MIM predictions for small $n$.
	In this investigation, discussed further below, we developed ``the difference parameter,'' a metric to measure the difference between the simulated and predicted $f(n)$ of a single origin with average $n$ initiators.
	The investigation showed the difference parameter decreased as $n^{-1}$ and motivated the deeper research presented in this thesis.
	
	Two single-origin investigations that followed our preliminary work consisted of simulating $f(n)$ and fitting the MIM parameters to the result.
	With these investigations we refined the simulation program to run more efficiently, and to create data similar to that measured in sequencing experiments.
		
	\begin{figure}[tbh!]
		\begin{center}
			\includegraphics[width=\textwidth]{Images/DifferenceParameterGraph.pdf}
		\end{center}
			\caption[Difference Parameter]{\label{fig:DifferenceParameter} Schematic of the difference parameter calculations.
				Colours represent differen values of $n$. Triangles correspond to the parameter values in the corresponding inset graphs.
				\textbf{A} Replication fraction simulation and theory curves for $n=10$; $DP=0.037$.
				\textbf{B} Replication fraction simulation and theory curves for $n=5$; $DP = 0.062$.
				\textbf{C} Replication fraction simulation and theory curves for $n=1$; $DP = 0.264$.
				Also illustrated is the value $\Delta P(x)$ at the peak.
				Note that $\Delta P(x)$ is defined over the entire domain.
				}
	\end{figure}
	
	
		\subsection{The Difference Parameter}
		\label{subsec:earlywork}
		
		When this project started, we performed a quick investigation into the difference between $f_\text{sim}(n)$ and $f_\text{MIM}(n)$ for a single origin.
		To generate $f_\text{sim}(n)$, the MIM simulator calculated the average replicated fraction for about $10^6$ cells, producing data with very little statistical error.
		The global parameters used for the simulation were set equal to those measured previously by fitting the MIM to microarray measurements of budding yeast~\cite{Scottspaper}.
		To generate $f_\text{MIM}(n)$, the fitting function built into IGOR was used, where the parameters were fixed and set equal to those used in the simulation.
		The difference parameter was then defined as
		\begin{equation} \label{DifferenceParameter}
			DP = \max_{x} {\frac {\Delta P(x)} {P}} \text{ ,}
		\end{equation}
		where $\Delta P(x)$ is the difference between $f_\text{sim}(n)$ and $f_\text{MIM}(n)$ (see Fig.~\ref{fig:DifferenceParameter}), and $P$ is the peak value of $f_\text{MIM}(n)$.
		
		Figure~\ref{fig:DifferenceParameter} shows the analysis process of the difference parameter for a single origin.
		First, $f_\text{sim}(n)$ an $f_\text{MIM}(n)$ were calculated over several time steps and $n$ ranging from one to 128, examples are shown in the insets of Fig.~\ref{fig:DifferenceParameter}.
		From these data, we calculated $DP(n,t)$, the main image in Fig.~\ref{fig:DifferenceParameter}.
		Note the value for $DP$ ``saturates'' as $t$ increases, we call this value the ``saturated difference parameter.''
		Figure~\ref{fig:SaturatedDifferenceParameter} shows the saturated difference parameter\footnote{
		The reader may notice a change in the $n$ values displayed in the graphics. Figure~\ref{fig:DifferenceParameter} is illustrative and contains old data; accurate, but not useful.
		After producing that graph we used a slightly different set of parameters in the simulation and changed the range of $n$ simulated when producing Fig.~\ref{fig:SaturatedDifferenceParameter}}.
		This initial investigation showed the saturated difference parameter decreases as $n^{-1}$.
		
		\begin{figure}[tbh]
			\begin{center}
				\includegraphics[width=0.8\textwidth]{Images/SaturatedDifferenceParameterGraph.pdf}
			\end{center}
				\caption[Saturated Difference Parameter]{\label{fig:SaturatedDifferenceParameter} Saturated difference parameter vs. $n$.
				The large uncertainty in low $n$ comes the time-steps not going far enough to accurately measure the saturated difference parameter.
				The line is proportional to $n^{-1}$ ($\text{[Saturated Difference Parameter]} \approx \frac{0.3}{n}$).
				}
		\end{figure}
		
		From our initial investigation, we concluded that the difference parameter grows quickly with decreasing $n$.
		Therefore, we suspect that the MIM will not produce accurate results in the case that $n$ is small.
		These results motivated further research into the effect of small $n$ on the MIM.
		
		
		\subsection{Biased Fits}
		\label{subsec:BiasedFits}
		
		The results from investigating the difference parameter suggested that the question of how variability in initiation factor affects the replication profile was worthy of more research.
		However, the definition of the difference parameter does not scale to more than one origin for two reasons:
		First, the saturated difference parameter is measurable once the theory curve has reached a peak value near one, therefore near-neighbour origins will interfere with each other.
		Second the difference parameter is defined as a single value for the whole simulated genome, therefore we cannot infer anything about more than one origin.
		Therefore, we investigated how the parameters from fitting the MIM to a simulation of small $n$ may be biased.
		
		The process of this investigation was to calculate $f_\text{sim}(n)$ and to follow that by fitting the parameters of the MIM to the result.
		Thus, we have two parameters, $n_\text{sim}$ and $n_\text{fit}$.